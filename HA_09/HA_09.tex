% Created 2020-09-24 Do 23:45
% Intended LaTeX compiler: pdflatex
\documentclass[a4paper,11pt]{article}
\usepackage[latin1]{inputenc}
\usepackage[T1]{fontenc}
\usepackage{graphicx}
\usepackage{grffile}
\usepackage{longtable}
\usepackage{wrapfig}
\usepackage{rotating}
\usepackage[normalem]{ulem}
\usepackage{amsmath}
\usepackage{textcomp}
\usepackage{amssymb}
\usepackage{capt-of}
\usepackage{hyperref}
\usepackage{mathtools}
\DeclarePairedDelimiter{\ceil}{\lceil}{\rceil}
\DeclarePairedDelimiter{\floor}{\lfloor}{\rfloor}
\author{Moaz Haque, Felix Oechelhaeuser, Leo Pirker, Dennis Schulze}
\date{\today}
\title{Analysis I und Lineare Algebra f�r Ingenieurwissenschaften \large  \\ Hausaufgabe 09 - Geuter 29}
\hypersetup{
 pdfauthor={Moaz Haque, Felix Oechelhaeuser, Leo Pirker, Dennis Schulze},
 pdftitle={Analysis I und Lineare Algebra f�r Ingenieurwissenschaften \large  \\ Hausaufgabe 09 - Geuter 29},
 pdfkeywords={},
 pdfsubject={},
 pdfcreator={Emacs 27.1 (Org mode 9.4)}, 
 pdflang={English}}
\begin{document}

\maketitle
\tableofcontents

\pagebreak

\section{Aufgabe 1}
\label{sec:org70c0394}
\begin{align*}
    f(x) = \sqrt{x} = x^{\frac{1}{2}} \\
    f'(x) = \frac{1}{2}x^{-\frac{1}{2}} \\
    f''(x) = -\frac{1}{4}x^{-\frac{3}{2}} \\
    f'''(x) = \frac{3}{8}x^{-\frac{5}{2}} \\
    f^{(4)}(x) = -\frac{15}{16}x^{-\frac{7}{2}} \\
    f^{(5)}(x) = \frac{105}{32}x^{-\frac{9}{2}} \\
    T_2(x) = \frac{f(1)}{0!}(x-1)^0 + \frac{f'(1)}{1!}(x-1)^{1} + \frac{f''(1)}{2!}(x-1)^{2} \\
    = 1 + \frac{1}{2}(x-1) - \frac{1}{8}(x-1)^2 \\
    \intertext{$\xi$ muss zwischen dem Entwicklungspunkt $f(0)$ und x liegen, also: $\xi \in \left]1,x\right[ \cup \left]x,1\right[$} \\
    R_2(x) = \frac{f'''(\xi)}{3!}(x-1)^3 = \frac{1}{16}\xi^{-\frac{5}{2}}(x-1)^3 \\
    T_4(x) =
    \frac{f(1)}{0!}(x-1)^0
    + \frac{f'(1)}{1!}(x-1)^{1}
    + \frac{f''(1)}{2!}(x-1)^{2}
    + \frac{f'''(1)}{3!}(x-1)^{3}
    + \frac{f^{4}(1)}{3!}(x-1)^{4} \\
    = 1 + \frac{1}{2}(x-1) - \frac{1}{8}(x-1)^2 + \frac{3}{48}(x-1)^3 -\frac{5}{128}(x-1)^4 \\
    \intertext{F�r $x$ gilt:}
    | x - 1 | \leq \frac{1}{2}
    \intertext{F�r $x - 1 \geq 0 \Leftrightarrow x \geq 1$}
    x - 1 \leq \frac{1}{2}
    \Leftrightarrow x \leq \frac{3}{2}
    \intertext{F�r $x - 1 < 0 \Leftrightarrow x < 1$}
    x - 1 \geq \frac{1}{2}
    \Leftrightarrow x \geq \frac{1}{2} \\
    \Rightarrow L = \left[1,\frac{3}{2}\right]
    \intertext{Den Maximalen Fehler des Restglieds von $T_2$, absch�tzen. $x \in \left[1,\frac{3}{2}\right]$ und $\xi \in \left]1,\frac{3}{2}\right[$}
    R_2\left(\frac{3}{2}\right)
    = \frac{1}{16}\xi^{-\frac{5}{2}}\left(\frac{3}{2}-1\right)^3
    = \frac{1}{128}\xi^{-\frac{5}{2}}
    = \frac{1}{128 \sqrt{\xi^5}}
    \leq \frac{1}{128 \sqrt{1^5}}
    = \frac{1}{128} = 0,0078125 \\
    T_2(1,1) = 1,04875 \\
    T_4(1,1) = 1.048808594 \\
    \intertext{Taschenrechner $\sqrt{1,1} = 1,048808848$}
    \intertext{$T_4$ lieftert eine genauere Approximation von $\sqrt{1,1}$, n�mlich auf sechs Nachkommastellen genau, anstatt nur drei bei $T_2$}
\end{align*}

\section{Aufgabe 2}
\label{sec:orgc2eee53}
\subsection{a)}
\label{sec:org97c6510}
Zu zeigen ist

$$ f^{(n)}(x) = n! \left( \frac{1}{(3-x)^{n+1}} + (-1)^n \frac{1}{(3+x)^{n+1}} \right) $$

f�r alle \(n \in \mathbb{N}\).

IA: \\
Es gilt

\begin{align*}
  f^{(0)}(x) &= 0! \left( \frac{1}{(3-x)^{0+1}} + (-1)^0 \frac{1}{(3+x)^{0+1}} \right) \\
   &= \frac{1}{3-x} + \frac{1}{3+x} \\
   &= \frac{(3+x) + (3-x)}{(3+x)(3-x)} \\
   &= \frac{6}{9 - x^2} = f(x)
\end{align*}

IV: \\
Es gelte f�r ein beliebiges, festes \(n \in \mathbb{N}\)

$$ f^{(n)}(x) = n! \left( \frac{1}{(3-x)^{n+1}} + (-1)^n \frac{1}{(3+x)^{n+1}} \right) $$

IS: \\
Dann gilt

\begin{align*}
  f^{(n+1)}(x) &= (f^{(n)}(x))' \\
   &= \left( n! \left( \frac{1}{(3-x)^{n+1}} + (-1)^n \frac{1}{(3+x)^{n+1}} \right) \right)' \\
   &= n! \left( \left( \frac{1}{(3-x)^{n+1}} \right)' + (-1)^n \left( \frac{1}{(3+x)^{n+1}} \right)' \right) \\
   &= n! \left( \left( (3-x)^{-(n+1)} \right)' + (-1)^n \left( (3+x)^{-(n+1)} \right)' \right) \\
   &= n! \left( \left( -(n+1) (3-x)^{-(n+2)} (-1) \right) + (-1)^n \left( -(n+1) (3+x)^{-(n+2)} \right) \right) \\
   &= n! (n+1) \left( (-1) (3-x)^{-(n+2)} (-1) + (-1)^n (-1) (3+x)^{-(n+2)} \right) \\
   &= (n+1)! \left( (3-x)^{-(n+2)} + (-1)^{n+1} (3+x)^{-(n+2)} \right) \\
   &= (n+1)! \left( \frac{1}{(3-x)^{n+2}}  + (-1)^{n+1} \frac{1}{(3+x)^{n+2}} \right)
\end{align*}

Damit gilt

$$ f^{(n)}(x) = n! \left( \frac{1}{(3-x)^{n+1}} + (-1)^n \frac{1}{(3+x)^{n+1}} \right) $$

\subsection{b)}
\label{sec:org30e2323}
Es gilt f�r alle \(n \in \mathbb{N}\)

\begin{align*}
  f^{(n)}(0) &= n! \left( \frac{1}{3^{n+1}} + (-1)^n \frac{1}{3^{n+1}} \right) \\
   &= n! \left( \frac{1}{3^{n+1}} + \frac{(-1)^n}{3^{n+1}} \right) \\
   &= n! \left( \frac{3^{n+1}}{3^{n+1} 3^{n+1}} + \frac{(-1)^n 3^{n+1}}{3^{n+1} 3^{n+1}} \right) \\
   &= n! \left( \frac{3^{n+1} + (-1)^n 3^{n+1}}{3^{2n+2}} \right) \\
\end{align*}

dann gilt f�r ungerade \(n\)

$$ f^{(n)}(0) = n! \left( \frac{3^{n+1} - 3^{n+1}}{3^{2n+2}} \right) = 0 $$

und f�r gerade \(n\) gilt

\begin{align*}
  f^{(n)}(0) &= n! \left( \frac{2 \cdot 3^{n+1}}{3^{2n+2}} \right) \\
   &= \frac{2 n!}{3^{n+1}}
\end{align*}

Seien \(n, n_e, n_o \in \mathbb{N}\) mit \(n_o = \floor{\frac{n}{2}}\) und \(n_e = \ceil{\frac{n}{2}}\). \\
Dann gilt f�r \(T_n\) mit \(x_0 = 0\)

\begin{align*}
  T_n(x) &= \sum_{k = 0}^{n} \frac{f^{(k)}(0)}{k!} x^k \\
   &= \sum_{k = 0}^{n_e} \frac{f^{(2k)}(0)}{(2k)!} x^{2k} + \sum_{k = 0}^{n_o} \frac{f^{(2k+1)}(0)}{(2k+1)!} x^{2k+1} \\
   &= \sum_{k = 0}^{n_e} \frac{f^{(2k)}(0)}{(2k)!} x^{2k} + 0 \\
   &= \sum_{k = 0}^{n_e} \frac{2 (2k)!}{(2k)! 3^{2k+1}} x^{2k} \\
   &= \sum_{k = 0}^{n_e} \frac{2 x^{2k}}{3^{2k+1}}
\end{align*}

mit dem jeweils entsprechenden Restglied

\begin{align*}
  R_n(x) &= \frac{f^{(n+1)}(\xi)}{(n+1)!} x^{n+1} \\
   &= \frac{ (n+1)! \left( \frac{1}{(3-\xi)^{n+2}} + (-1)^{n+1} \frac{1}{(3+\xi)^{n+2}} \right) }{(n+1)!} x^{n+1} \\
   &= \frac{ (n+1)! \left( \frac{(3+\xi)^{n+2} + (-1)^{n+1} (3-\xi)^{n+1}}{(3-\xi)^{n+2} (3+\xi)^{n+2}} \right) }{(n+1)!} x^{n+1} \\
   &= \frac{ (3+\xi)^{n+2} + (-1)^n (3-\xi)^{n+2} }{(3-\xi)^{n+2} (3+\xi)^{n+2}} x^{n+1}
\end{align*}

\subsection{c)}
\label{sec:org8cc393b}
Zu zeigen ist f�r \(x \in ]-1, 1[\)

$$ \lim_{n \rightarrow \infty} R_n(x) = 0 $$

Es gilt

\begin{align*}
  \lim_{n \rightarrow \infty} R_n(x) &= \lim_{n \rightarrow \infty} \frac{ (3+\xi)^{n+2} + (-1)^{n+1} (3-\xi)^{n+2} }{(3-\xi)^{n+2} (3+\xi)^{n+2}} x^{n+1} \\
   &= \lim_{n \rightarrow \infty} \frac{ 4^{n+2} + (-1)^{n+1} 2^{n+2} }{ 2^{n+2} 4^{n+2}} x^{n+1}, \xi = 1 \\
   &= \lim_{n \rightarrow \infty} \frac{ 4^{n+2} + (-1)^{n+1} 2^{n+2} }{ 2^{n+2} 4^{n+2}} x^{n+1} \\
   &= \lim_{n \rightarrow \infty} \frac{ 2^{2n+4} + (-1)^{n+1} 2^{n+2} }{ 2^{n+2} 2^{2n+4}} x^{n+1} \\
   &= \lim_{n \rightarrow \infty} \frac{ 2^{n+2} (2^{n+2} + (-1)^{n+2}) }{ 2^{3(n+2)}} x^{n+1} \\
   &= \lim_{n \rightarrow \infty} \frac{ 2^{n+2} + (-1)^n }{2^{2(n+2)}} x^{n+1} \\
   &= \lim_{n \rightarrow \infty} \frac{ 2^{n+2}}{2^{2(n+1)}} x^{n+1} + \lim_{n \rightarrow \infty} \frac{(-1)^n }{2^{2(n+2)}} x^{n+1} \\
   &= \lim_{n \rightarrow \infty} \frac{ x^{n+1} }{2^{n+2}} + \lim_{n \rightarrow \infty} \frac{(-1)^{n+1} x^{n+1}}{2^{2(n+2)}} \\
   &= 0 + 0 = 0 \\
\end{align*}

nach \(n+2\) -maliger Anwendung der Regel von de l'Hospital. \\
Damit gilt f�r \(x \in ]-1, 1[\)

$$ \lim_{n \rightarrow \infty} R_n(x) = 0 $$

\section{Aufgabe 3}
\label{sec:orgdd6ecef}
F�r die n-te Ableitung von \(f\) gilt

\begin{align*}
  f^{(n)}(x) &= \frac{d^n}{dx^n} 5^x = \ln(5) \cdot \frac{d^{n-1}}{dx^{n-1}} 5^x \\
   &= (\ln(5))^n 5^x
\end{align*}

damit gilt f�r dessen Taylorreihe \(T_n\) mit \(x_0 = 0\)

\begin{align*}
  T_n(x) &= 1 + \frac{\ln(5) x}{1!} + \frac{(\ln(5) x)^2}{2!} + \frac{(\ln(5) x)^3}{3!} + \dots + \frac{(\ln(5) x)^n}{n!} \\
   &= \sum_{k = 0}^{n} \frac{(\ln(5) x)^k}{k!}
\end{align*}

mit dessen Restglied

\begin{align*}
  R_n(x) &= \frac{f^{(n+1)}(\xi)}{(n+1)!}x^{n+1} \\
   &= \frac{ (\ln(5))^{n+1} 5^{\xi} }{(n+1)!} x^{n+1} \\
   &= \frac{ (\ln(5))^{n+1} 5^{\xi} x^{n+1} }{(n+1)!} \\
   &= \frac{ (x \ln(5))^{n+1} 5^{\xi} }{(n+1)!} \\
\end{align*}

Es gilt

$$ \lim_{n \rightarrow \infty} R_n(x) = \frac{ (x \ln(5))^{n+1} 5^{\xi} }{(n+1)!} $$

Wenn man die Fakult�t f�r alle \(n > 1\) betrachtet, w�chst diese schneller als jede Exponentialfunktion.
Deshalb sollte gelten

$$ \lim_{n \rightarrow \infty} R_n(x) = 0 $$

\section{Aufgabe 4}
\label{sec:orgc624821}
\subsection{a)}
\label{sec:org406dc1f}
\begin{align*}
  \sqrt{4^{3(x-2)}} &= 16^{1-x} \\
\Leftrightarrow 4^{\frac{3}{2}(x-2)} &= (4^2)^{1-x} \\
\Leftrightarrow 4^{\frac{3}{2}(x-2)} &= 4^{2(1-x)} \\
\Leftrightarrow \frac{3}{2}(x-2) &= 2(1-x) \\
\Leftrightarrow 3(x-2) &= 4(1-x) \\
\Leftrightarrow 3x - 6 &= 4 - 4x \\
\Leftrightarrow 7x &= 10 \\
\Leftrightarrow x &= \frac{10}{7}
\end{align*}

\subsection{b)}
\label{sec:org662e293}
\begin{align*}
  e^{ix - \ln(2)} + \frac{1}{e^{ix + \ln(2)}} &= 1 \\
\Leftrightarrow e^{ix - \ln(2)} + e^{-ix - \ln(2)} &= 1 \\
\Leftrightarrow e^{ix} e^{-\ln(2)} + e^{-ix} e^{-\ln(2)} &= 1 \\
\Leftrightarrow \frac{e^{ix}}{2} + \frac{e^{-ix}}{2} &= 1 \\
\Leftrightarrow \frac{e^{ix} + e^{-ix}}{2} &= 1 \\
\Leftrightarrow \frac{(\cos(x) + i \sin(x)) + (\cos(-x) + i \sin(-x))}{2} &= 1 \\
\Leftrightarrow \frac{(\cos(x) + i \sin(x)) + (\cos(x) - i \sin(x))}{2} &= 1 \\
\Leftrightarrow \cos(x) &= 1 \\
\Rightarrow x = 2k\pi, k \in \mathbb{Z}
\end{align*}

\section{Aufgabe 5}
\label{sec:org7f72a54}
\subsection{a)}
\label{sec:orgd0fc8df}
\begin{align*}
    \intertext{F�r $x,y > 0, x,y,\in \mathbb{R}$ gilt:}
    0 \leq (x-y)^2 \\
    \Leftrightarrow 0 \leq x^2 -2xy+y^2 \\
    \Leftrightarrow 4xy \leq x^2 + 2xy + y^2 \\
    \Leftrightarrow xy \leq \frac{x^2+2xy+y^2}{4} = \frac{(x+y)^2}{4} \\
    \Leftrightarrow (xy)^{\frac{1}{2}} \leq \frac{x+y}{2} \\
    \Leftrightarrow \ln(xy^{\frac{1}{2}}) \leq \ln\left(\frac{x+y}{2}\right) \\
    \Leftrightarrow \frac{\ln(xy)}{2} = \frac{\ln(x) + \ln(y)}{2} \leq \ln\left(\frac{x+y}{2}\right)
\end{align*}

\subsection{b)}
\label{sec:orgef0b530}
\section{Aufgabe 6}
\label{sec:org0f854b6}
\subsection{a)}
\label{sec:org07dd623}
Es gilt

$$ \lim_{x \rightarrow \infty} \frac{\frac{\pi}{2} - \arctan(x)}{e^{-x}} = \frac{\frac{\pi}{2} - \frac{\pi}{2}}{0} = \frac{0}{0} $$

damit ist die Regel von de l'Hospital anwendbar und es gilt

\begin{align*}
  \lim_{x \rightarrow \infty} \frac{\frac{\pi}{2} - \arctan(x)}{e^{-x}} &= \lim_{x \rightarrow \infty} \frac{-\frac{1}{1 + x^2}}{-e^{-x}} \\
   &= \lim_{x \rightarrow \infty} -\frac{1}{(1 + x^2) (-e^{-x})} \\
   &= 0
\end{align*}

Damit existiert dessen Grenzwert.

\subsection{b)}
\label{sec:orgb60b260}
Es gilt

\begin{align*}
  \lim_{x \rightarrow 0} (1 + \arctan(x))^{\frac{1}{x}} &= \lim_{x \rightarrow 0} (1 + \arctan(x))^{x^{-1}} \\
   &= \lim_{x \rightarrow 0} e^{x^{-1} \ln(1 + \arctan(x))}
\end{align*}

und besitzt einen Grenzwert genau dann, wenn der Exponent einen Grenzwert besitzt. Also gilt

$$ \lim_{x \rightarrow 0} x^{-1} \ln(1 + \arctan(x)) = \lim_{x \rightarrow 0} \frac{\ln(1 + \arctan(x))}{x} = \frac{0}{0} $$

wo die bedingung f�r die Regel von de l'Hospital erf�llt ist. Damit gilt 

\begin{align*}
  \lim_{x \rightarrow 0} x^{-1} \ln(1 + \arctan(x)) &= \lim_{x \rightarrow 0} \frac{\frac{1}{1 + \arctan(x)} \frac{1}{1 + x^2}}{1} \\
   &= \lim_{x \rightarrow 0} \frac{1}{(1 + \arctan(x))(1 + x^2)} = 1
\end{align*}

Da dieser Grenzwert existiert, existiert auch der andere. Damit gilt

\begin{align*}
  \lim_{x \rightarrow 0} (1 + \arctan(x))^{\frac{1}{x}} = e
\end{align*}

mit \(e\) der Eulerschen Zahl.

\section{Aufgabe 7}
\label{sec:orgb214ded}
\subsection{a)}
\label{sec:orge7348a7}
\begin{align*}
    v(t) = s'(t) = \left(\frac{v^2}{g} \ln\left(\cosh\left(\frac{gt}{v}\right)\right)\right)'
    = \frac{v^2 g \sinh(\frac{gt}{v})}{vg \cosh(\frac{gt}{v})}
    = v \tanh\left(\frac{gt}{v}\right) \\
    a(t) = s'(t) = \left(v \tanh\left(\frac{gt}{v}\right)\right)'
    = v \cdot \frac{g}{\cosh(\frac{gt}{v})^2v}
    = \frac{g}{\cosh(\frac{gt}{v})^2}
\end{align*}

\subsection{b)}
\label{sec:org7d63204}
\begin{align*}
    s(5) = \frac{50^2 m/s}{9,81m/s^2} \ln\left(\cosh\left(\frac{9,81m/s^2 \cdot 5s}{50m/s}\right)\right)
    = 283,52m \\
    \Delta s = s'(t) \cdot \Delta t
    = 50m/s \cdot \tanh\left(\frac{9,81m/s^2 \cdot 5s}{50m/s}\right) \cdot \frac{1}{2} = \pm 18,84m \\
    |\Delta s| = 18,84m
\end{align*}
\end{document}