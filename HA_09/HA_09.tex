% Created 2020-09-24 Do 22:18
% Intended LaTeX compiler: pdflatex
\documentclass[a4paper,11pt]{article}
\usepackage[latin1]{inputenc}
\usepackage[T1]{fontenc}
\usepackage{graphicx}
\usepackage{grffile}
\usepackage{longtable}
\usepackage{wrapfig}
\usepackage{rotating}
\usepackage[normalem]{ulem}
\usepackage{amsmath}
\usepackage{textcomp}
\usepackage{amssymb}
\usepackage{capt-of}
\usepackage{hyperref}
\usepackage{mathtools}
\DeclarePairedDelimiter{\ceil}{\lceil}{\rceil}
\DeclarePairedDelimiter{\floor}{\lfloor}{\rfloor}
\author{Moaz Haque, Felix Oechelhaeuser, Leo Pirker, Dennis Schulze}
\date{\today}
\title{Analysis I und Lineare Algebra f�r Ingenieurwissenschaften \large  \\ Hausaufgabe 09 - Geuter 29}
\hypersetup{
 pdfauthor={Moaz Haque, Felix Oechelhaeuser, Leo Pirker, Dennis Schulze},
 pdftitle={Analysis I und Lineare Algebra f�r Ingenieurwissenschaften \large  \\ Hausaufgabe 09 - Geuter 29},
 pdfkeywords={},
 pdfsubject={},
 pdfcreator={Emacs 27.1 (Org mode 9.4)}, 
 pdflang={English}}
\begin{document}

\maketitle
\tableofcontents

\pagebreak

\section{Aufgabe 1}
\label{sec:org32274b3}
\section{Aufgabe 2}
\label{sec:org4356d5c}
\subsection{a)}
\label{sec:org4b0f815}
Zu zeigen ist

$$ f^{(n)}(x) = n! \left( \frac{1}{(3-x)^{n+1}} + (-1)^n \frac{1}{(3+x)^{n+1}} \right) $$

f�r alle \(n \in \mathbb{N}\).

IA: \\
Es gilt

\begin{align*}
  f^{(0)}(x) &= 0! \left( \frac{1}{(3-x)^{0+1}} + (-1)^0 \frac{1}{(3+x)^{0+1}} \right) \\
   &= \frac{1}{3-x} + \frac{1}{3+x} \\
   &= \frac{(3+x) + (3-x)}{(3+x)(3-x)} \\
   &= \frac{6}{9 - x^2} = f(x)
\end{align*}

IV: \\
Es gelte f�r ein beliebiges, festes \(n \in \mathbb{N}\)

$$ f^{(n)}(x) = n! \left( \frac{1}{(3-x)^{n+1}} + (-1)^n \frac{1}{(3+x)^{n+1}} \right) $$

IS: \\
Dann gilt

\begin{align*}
  f^{(n+1)}(x) &= (f^{(n)}(x))' \\
   &= \left( n! \left( \frac{1}{(3-x)^{n+1}} + (-1)^n \frac{1}{(3+x)^{n+1}} \right) \right)' \\
   &= n! \left( \left( \frac{1}{(3-x)^{n+1}} \right)' + (-1)^n \left( \frac{1}{(3+x)^{n+1}} \right)' \right) \\
   &= n! \left( \left( (3-x)^{-(n+1)} \right)' + (-1)^n \left( (3+x)^{-(n+1)} \right)' \right) \\
   &= n! \left( \left( -(n+1) (3-x)^{-(n+2)} (-1) \right) + (-1)^n \left( -(n+1) (3+x)^{-(n+2)} \right) \right) \\
   &= n! (n+1) \left( (-1) (3-x)^{-(n+2)} (-1) + (-1)^n (-1) (3+x)^{-(n+2)} \right) \\
   &= (n+1)! \left( (3-x)^{-(n+2)} + (-1)^{n+1} (3+x)^{-(n+2)} \right) \\
   &= (n+1)! \left( \frac{1}{(3-x)^{n+2}}  + (-1)^{n+1} \frac{1}{(3+x)^{n+2}} \right)
\end{align*}

Damit gilt

$$ f^{(n)}(x) = n! \left( \frac{1}{(3-x)^{n+1}} + (-1)^n \frac{1}{(3+x)^{n+1}} \right) $$

\subsection{b)}
\label{sec:orgf9bb922}
Es gilt f�r alle \(n \in \mathbb{N}\)

\begin{align*}
  f^{(n)}(0) &= n! \left( \frac{1}{3^{n+1}} + (-1)^n \frac{1}{3^{n+1}} \right) \\
   &= n! \left( \frac{1}{3^{n+1}} + \frac{(-1)^n}{3^{n+1}} \right) \\
   &= n! \left( \frac{3^{n+1}}{3^{n+1} 3^{n+1}} + \frac{(-1)^n 3^{n+1}}{3^{n+1} 3^{n+1}} \right) \\
   &= n! \left( \frac{3^{n+1} + (-1)^n 3^{n+1}}{3^{2n+2}} \right) \\
\end{align*}

dann gilt f�r ungerade \(n\)

$$ f^{(n)}(0) = n! \left( \frac{3^{n+1} - 3^{n+1}}{3^{2n+2}} \right) = 0 $$

und f�r gerade \(n\) gilt

\begin{align*}
  f^{(n)}(0) &= n! \left( \frac{2 \cdot 3^{n+1}}{3^{2n+2}} \right) \\
   &= \frac{2 n!}{3^{n+1}}
\end{align*}

Seien \(n, n_e, n_o \in \mathbb{N}\) mit \(n_o = \floor{\frac{n}{2}}\) und \(n_e = \ceil{\frac{n}{2}}\). \\
Dann gilt f�r \(T_n\) mit \(x_0 = 0\)

\begin{align*}
  T_n(x) &= \sum_{k = 0}^{n} \frac{f^{(k)}(0)}{k!} x^k \\
   &= \sum_{k = 0}^{n_e} \frac{f^{(2k)}(0)}{(2k)!} x^{2k} + \sum_{k = 0}^{n_o} \frac{f^{(2k+1)}(0)}{(2k+1)!} x^{2k+1} \\
   &= \sum_{k = 0}^{n_e} \frac{f^{(2k)}(0)}{(2k)!} x^{2k} + 0 \\
   &= \sum_{k = 0}^{n_e} \frac{2 (2k)!}{(2k)! 3^{2k+1}} x^{2k} \\
   &= \sum_{k = 0}^{n_e} \frac{2 x^{2k}}{3^{2k+1}}
\end{align*}

mit dem jeweils entsprechenden Restglied

\begin{align*}
  R_n(x) &= \frac{f^{(n+1)}(\xi)}{(n+1)!} x^{n+1} \\
   &= \frac{ (n+1)! \left( \frac{1}{(3-\xi)^{n+2}} + (-1)^{n+1} \frac{1}{(3+\xi)^{n+2}} \right) }{(n+1)!} x^{n+1} \\
   &= \frac{ (n+1)! \left( \frac{(3+\xi)^{n+2} + (-1)^{n+1} (3-\xi)^{n+1}}{(3-\xi)^{n+2} (3+\xi)^{n+2}} \right) }{(n+1)!} x^{n+1} \\
   &= \frac{ (3+\xi)^{n+2} + (-1)^n (3-\xi)^{n+2} }{(3-\xi)^{n+2} (3+\xi)^{n+2}} x^{n+1}
\end{align*}

\subsection{c)}
\label{sec:org74b61ed}
Zu zeigen ist f�r \(x \in ]-1, 1[\)

$$ \lim_{n \rightarrow \infty} R_n(x) = 0 $$

Es gilt

\begin{align*}
  \lim_{n \rightarrow \infty} R_n(x) &= \lim_{n \rightarrow \infty} \frac{ (3+\xi)^{n+2} + (-1)^{n+1} (3-\xi)^{n+2} }{(3-\xi)^{n+2} (3+\xi)^{n+2}} x^{n+1} \\
   &= \lim_{n \rightarrow \infty} \frac{ 4^{n+2} + (-1)^{n+1} 2^{n+2} }{ 2^{n+2} 4^{n+2}} x^{n+1}, \xi = 1 \\
   &= \lim_{n \rightarrow \infty} \frac{ 4^{n+2} + (-1)^{n+1} 2^{n+2} }{ 2^{n+2} 4^{n+2}} x^{n+1} \\
   &= \lim_{n \rightarrow \infty} \frac{ 2^{2n+4} + (-1)^{n+1} 2^{n+2} }{ 2^{n+2} 2^{2n+4}} x^{n+1} \\
   &= \lim_{n \rightarrow \infty} \frac{ 2^{n+2} (2^{n+2} + (-1)^{n+2}) }{ 2^{3(n+2)}} x^{n+1} \\
   &= \lim_{n \rightarrow \infty} \frac{ 2^{n+2} + (-1)^n }{2^{2(n+2)}} x^{n+1} \\
   &= \lim_{n \rightarrow \infty} \frac{ 2^{n+2}}{2^{2(n+1)}} x^{n+1} + \lim_{n \rightarrow \infty} \frac{(-1)^n }{2^{2(n+2)}} x^{n+1} \\
   &= \lim_{n \rightarrow \infty} \frac{ x^{n+1} }{2^{n+2}} + \lim_{n \rightarrow \infty} \frac{(-1)^{n+1} x^{n+1}}{2^{2(n+2)}} \\
   &= 0 + 0 = 0 \\
\end{align*}

nach \(n+2\) -maliger Anwendung der Regel von de l'Hospital. \\
Damit gilt f�r \(x \in ]-1, 1[\)

$$ \lim_{n \rightarrow \infty} R_n(x) = 0 $$

\section{Aufgabe 3}
\label{sec:orgb5459d1}
F�r die n-te Ableitung von \(f\) gilt

\begin{align*}
  f^{(n)}(x) &= \frac{d^n}{dx^n} 5^x = \ln(5) \cdot \frac{d^{n-1}}{dx^{n-1}} 5^x \\
   &= (\ln(5))^n 5^x
\end{align*}

damit gilt f�r dessen Taylorreihe \(T_n\) mit \(x_0 = 0\)

\begin{align*}
  T_n(x) &= 1 + \frac{\ln(5) x}{1!} + \frac{(\ln(5) x)^2}{2!} + \frac{(\ln(5) x)^3}{3!} + \dots + \frac{(\ln(5) x)^n}{n!} \\
   &= \sum_{k = 0}^{n} \frac{(\ln(5) x)^k}{k!}
\end{align*}

mit dessen Restglied

\begin{align*}
  R_n(x) &= \frac{f^{(n+1)}(\xi)}{(n+1)!}x^{n+1} \\
   &= \frac{ (\ln(5))^{n+1} 5^{\xi} }{(n+1)!} x^{n+1} \\
   &= \frac{ (\ln(5))^{n+1} 5^{\xi} x^{n+1} }{(n+1)!} \\
   &= \frac{ (x \ln(5))^{n+1} 5^{\xi} }{(n+1)!} \\
\end{align*}

Es gilt

$$ \lim_{n \rightarrow \infty} R_n(x) &= \frac{ (x \ln(5))^{n+1} 5^{\xi} }{(n+1)!} $$

Wenn man die Fakult�t f�r alle \(n > 1\) betrachtet, w�chst diese schneller als jede Exponentialfunktion.
Deshalb sollte gelten

$$ \lim_{n \rightarrow \infty} R_n(x) = 0 $$

\section{Aufgabe 4}
\label{sec:org1c97458}
\subsection{a)}
\label{sec:org663dec6}
\begin{align*}
  \sqrt{4^{3(x-2)}} &= 16^{1-x} \\
\Leftrightarrow 4^{\frac{3}{2}(x-2)} &= (4^2)^{1-x} \\
\Leftrightarrow 4^{\frac{3}{2}(x-2)} &= 4^{2(1-x)} \\
\Leftrightarrow \frac{3}{2}(x-2) &= 2(1-x) \\
\Leftrightarrow 3(x-2) &= 4(1-x) \\
\Leftrightarrow 3x - 6 &= 4 - 4x \\
\Leftrightarrow 7x &= 10 \\
\Leftrightarrow x &= \frac{10}{7}
\end{align*}

\subsection{b)}
\label{sec:org8826b4b}
\begin{align*}
  e^{ix - \ln(2)} + \frac{1}{e^{ix + \ln(2)}} &= 1 \\
\Leftrightarrow e^{ix - \ln(2)} + e^{-ix - \ln(2)} &= 1 \\
\Leftrightarrow e^{ix} e^{-\ln(2)} + e^{-ix} e^{-\ln(2)} &= 1 \\
\Leftrightarrow \frac{e^{ix}}{2} + \frac{e^{-ix}}{2} &= 1 \\
\Leftrightarrow \frac{e^{ix} + e^{-ix}}{2} &= 1 \\
\Leftrightarrow \frac{(\cos(x) + i \sin(x)) + (\cos(-x) + i \sin(-x))}{2} &= 1 \\
\Leftrightarrow \frac{(\cos(x) + i \sin(x)) + (\cos(x) - i \sin(x))}{2} &= 1 \\
\Leftrightarrow \cos(x) &= 1 \\
\Rightarrow x = 2k\pi, k \in \mathbb{Z}
\end{align*}

\section{Aufgabe 5}
\label{sec:org07b0cd1}
\subsection{a)}
\label{sec:org31d9101}
\subsection{b)}
\label{sec:org209fa88}
\section{Aufgabe 6}
\label{sec:orgf73c344}
\subsection{a)}
\label{sec:org9943966}
\subsection{b)}
\label{sec:org50ee310}
\section{Aufgabe 7}
\label{sec:org93945c3}
\subsection{a)}
\label{sec:org8793747}
\subsection{b)}
\label{sec:org8a851d1}
\end{document}