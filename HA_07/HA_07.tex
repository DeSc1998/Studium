% Created 2020-09-15 Di 19:31
% Intended LaTeX compiler: pdflatex
\documentclass[a4paper,11pt]{article}
\usepackage[latin1]{inputenc}
\usepackage[T1]{fontenc}
\usepackage{graphicx}
\usepackage{grffile}
\usepackage{longtable}
\usepackage{wrapfig}
\usepackage{rotating}
\usepackage[normalem]{ulem}
\usepackage{amsmath}
\usepackage{textcomp}
\usepackage{amssymb}
\usepackage{capt-of}
\usepackage{hyperref}
\author{Moaz Haque, Felix Oechelhaeuser, Leo Pirker, Dennis Schulze}
\date{\today}
\title{Analysis I und Lineare Algebra f�r Ingenieurwissenschaften \large  \\ Hausaufgabe 07 - Geuter 29}
\hypersetup{
 pdfauthor={Moaz Haque, Felix Oechelhaeuser, Leo Pirker, Dennis Schulze},
 pdftitle={Analysis I und Lineare Algebra f�r Ingenieurwissenschaften \large  \\ Hausaufgabe 07 - Geuter 29},
 pdfkeywords={},
 pdfsubject={},
 pdfcreator={Emacs 27.1 (Org mode 9.3.6)}, 
 pdflang={English}}
\begin{document}

\maketitle
\tableofcontents

\pagebreak

\section{Aufgabe 1}
\label{sec:org42095c7}
\subsection{a)}
\label{sec:org626a67d}
Es gilt

\begin{align*}
  \frac{1}{b} + \frac{1}{g} &= \frac{1}{f} \\
  \Leftrightarrow \frac{1}{b} &= \frac{g - f}{fg} \\
  \Leftrightarrow b &= \frac{fg}{g - f}
\end{align*}

Damit ist b betrachtet als Funktion

$$ b(g) = \frac{fg}{g - f} $$

\subsection{b)}
\label{sec:org1551f1f}
Da der Bruch nur f�r alle \(g \neq f\) definiert ist, \\
ist b nur auf \(\mathbb{R}\) $\backslash$ \{\(f\)\} stetig.

\subsection{c)}
\label{sec:orgc9ddaa3}
Es gilt f�r den ersten Term

\begin{align*}
  \lim_{g \rightarrow \infty} b(g) &= \lim_{g \rightarrow \infty} \frac{fg}{g - f} \\
  &= \lim_{g \rightarrow \infty} \frac{f}{1 - \frac{f}{g}} \\
  &= \frac{f}{1 - 0} \\
  &= f
\end{align*}

Es gilt f�r den zweiten Term

\begin{align*}
  \lim_{g \rightarrow f} b(g) &= \lim_{g \rightarrow f} \frac{fg}{g - f} \\
  &= \frac{\lim_{g \rightarrow f} fg}{\lim_{g \rightarrow f} (g - f)} \\
  &= \frac{f^2}{f - f} \\
  &\Rightarrow \lim_{g \rightarrow f} b(g) = \infty
\end{align*}

\section{Aufgabe 2}
\label{sec:orgbbd6495}
\subsection{a)}
\label{sec:org3d251e4}
\subsection{b)}
\label{sec:org20fa916}
\section{Aufgabe 3}
\label{sec:org37fd289}
\subsection{a)}
\label{sec:orgaf8fe37}
\subsection{b)}
\label{sec:org4abf784}
\section{Aufgabe 4}
\label{sec:org3b8543a}
\subsection{a)}
\label{sec:org01ebe1f}
\subsection{b)}
\label{sec:orge9e9ec8}
\section{Aufgabe 5}
\label{sec:org4297757}
\section{Aufgabe 6}
\label{sec:orgafc5df6}
\subsection{a)}
\label{sec:orgd909f71}
\subsection{b)}
\label{sec:orgdff701f}
\end{document}