% Created 2020-11-14 Sa 15:43
% Intended LaTeX compiler: pdflatex
\documentclass[a4paper, 11pt]{article}
\usepackage[utf8]{inputenc}
\usepackage[T1]{fontenc}
\usepackage{graphicx}
\usepackage{grffile}
\usepackage{longtable}
\usepackage{wrapfig}
\usepackage{rotating}
\usepackage[normalem]{ulem}
\usepackage{amsmath}
\usepackage{textcomp}
\usepackage{amssymb}
\usepackage{capt-of}
\usepackage{hyperref}
\usepackage{braket}
\author{Daniel Geinets (), Christopher Neumann (), Dennis Schulze (458415)}
\date{\today}
\title{Analysis I und Lineare Algebra für Ingenieurwissenschaften \large  \\ Hausaufgabe 02 - Al-Maweri 13}
\hypersetup{
 pdfauthor={Daniel Geinets (), Christopher Neumann (), Dennis Schulze (458415)},
 pdftitle={Analysis I und Lineare Algebra für Ingenieurwissenschaften \large  \\ Hausaufgabe 02 - Al-Maweri 13},
 pdfkeywords={},
 pdfsubject={},
 pdfcreator={Emacs 26.3 (Org mode 9.4)}, 
 pdflang={English}}
\begin{document}

\maketitle
\tableofcontents

\setcounter{secnumdepth}{0}
\newcommand{\tuple}[1]{\left(#1\right)}
\newcommand{\R}{\mathbb{R}}
\newcommand{\Q}{\mathbb{Q}}
\newcommand{\N}{\mathbb{N}}
\newcommand{\C}{\mathbb{C}}

\pagebreak

\section{Aufgabe 1}
\label{sec:org0ae6b3c}
\subsection{a)}
\label{sec:org2c1632a}
\subsection{b)}
\label{sec:org168152a}
$$ \prod_{k=2}^{n} \tuple{1 - \frac{1}{k^2}} = \frac{n + 1}{2n}, n \in \N, n \geq 2 $$

IA:
  Es gilt

\begin{align*}
    \prod_{k = 2}^{2} \tuple{1 - \frac{1}{k^2}} = \frac{2 + 1}{4}
    = \frac{3}{4}
    = 1 - \frac{1}{4}
\end{align*}

IV:
  Es gelte für ein beliebiges n \(\in\) \(\N\) mit n > 1
$$ \prod_{k=2}^{n} \tuple{1 - \frac{1}{k^2}} = \frac{n + 1}{2n} $$

IB:
  Dann gilt
$$ \prod_{k=2}^{n + 1} \tuple{1 - \frac{1}{k^2}} = \frac{n + 2}{2(n + 1)} $$

IS:
  Es gilt

\begin{align*}
    \prod_{k = 2}^{n + 1} \tuple{1 - \frac{1}{k^2}} &= \prod_{k = 2}^{n} \tuple{1 - \frac{1}{k^2}} \cdot \tuple{1 - \frac{1}{(n + 1)^2}} \\
    &= \frac{n + 1}{2n} \cdot \tuple{1 - \frac{1}{(n + 1)^2}} \\
    &= \frac{n + 1}{2n} \cdot \tuple{\frac{(n+1)^2}{(n+1)^2} - \frac{1}{(n + 1)^2}} \\
    &= \frac{n + 1}{2n} \cdot \frac{(n+1)^2 - 1}{(n+1)^2} \\
    &= \frac{n + 1}{2n} \cdot \frac{((n+1) - 1)((n+1) + 1)}{(n + 1)^2} \\
    &= \frac{n + 1}{2n} \cdot \frac{n(n + 2)}{(n + 1)(n + 1)} \\
    &= \frac{(n + 1) \cdot n(n + 2)}{2n(n + 1)(n + 1)}
    = \frac{n + 2}{2(n + 1)}
\end{align*}



Damit gilt für alle n \(\in\) \(\N\) mit n > 1
$$ \prod_{k=2}^{n} \tuple{1 - \frac{1}{k^2}} = \frac{n + 1}{2n} $$

\subsection{c)}
\label{sec:org462334e}
$$ \forall n \in \N, n \geq 4, \text{ gilt } 2^n \geq n^2 $$

IA:
  Es gilt für n = 4

\begin{align*}
    2^4 &\geq 4^2 \\
    \Leftrightarrow 16 &\geq 16
\end{align*}

IV:
  Es gelte für ein beliebiges \(n \in \N\) mit n \(\ge\) 4
$$ 2^n \geq n^2 $$

IB:
  Dann gilt
$$ 2^{n + 1} \geq (n + 1)^2 $$

IS:
  Es gilt

\begin{align*}
    2^{n + 1} &\geq (n + 1)^2 \\
    \Leftrightarrow 2 \cdot 2^n &\geq 2 \cdot n^2 = n^2 + 2n + 1 \\
    \Rightarrow n^2 &= 2n + 1 \\
    \Leftrightarrow 0 &= n^2 - 2n - 1 \\
    \Rightarrow n_1 &= 1 - \sqrt{2} \text{ und } n_2 = 1 + \sqrt{2} \\
    \Rightarrow n_1 &< 4 \text{ und } n_2 < 4 \text{ (siehe IV.)} \\
    \Rightarrow n^2 &> 2n + 1 \\
    \Rightarrow 2 \cdot n^2 &> n^2 + 2n + 1 = (n + 1)^2 \\
    \Rightarrow 2 \cdot 2^n &> (n + 1)^2 \\
    \Leftrightarrow 2^{n + 1} &> (n + 1)^2 \\
\end{align*}


Damit gilt für alle \(n \in \N\) mit n \(\ge\) 4
$$ 2^n \geq n^2 $$

\section{Aufgabe 2}
\label{sec:orgf8f8866}
\subsection{a)}
\label{sec:org37ae618}
\subsection{b)}
\label{sec:orgd9eeabf}
\section{Aufgabe 3}
\label{sec:orgc855dac}
\subsection{a)}
\label{sec:org966af4f}
\subsection{b)}
\label{sec:org571a5d9}
\subsection{c)}
\label{sec:orgb3ff4f5}
\subsection{d)}
\label{sec:org203563a}
\section{Aufgabe 4}
\label{sec:org6060963}
\subsection{a)}
\label{sec:orgbba321e}
\subsection{b)}
\label{sec:orgad293ef}
\subsection{c)}
\label{sec:orgba9b438}
\end{document}
