% Created 2021-02-03 Mi 13:52
% Intended LaTeX compiler: pdflatex
\documentclass[a4paper, 11pt]{article}
\usepackage[utf8]{inputenc}
\usepackage[T1]{fontenc}
\usepackage{graphicx}
\usepackage{grffile}
\usepackage{longtable}
\usepackage{wrapfig}
\usepackage{rotating}
\usepackage[normalem]{ulem}
\usepackage{amsmath}
\usepackage{textcomp}
\usepackage{amssymb}
\usepackage{capt-of}
\usepackage{hyperref}
\usepackage{braket}
\usepackage[germanb]{babel}
\usepackage[dvipsnames]{xcolor}
\definecolor{BG}{RGB}{28, 20, 8}
\definecolor{FG}{RGB}{60, 140, 0}
\pagecolor{BG}
\color{FG}
\author{Daniel Geinets (453843), Christopher Neumann (409098), Dennis Schulze (458415)}
\date{\today}
\title{Analysis I und Lineare Algebra für Ingenieurwissenschaften \large  \\ Hausaufgabe 10 - Al-Maweri 13}
\hypersetup{
 pdfauthor={Daniel Geinets (453843), Christopher Neumann (409098), Dennis Schulze (458415)},
 pdftitle={Analysis I und Lineare Algebra für Ingenieurwissenschaften \large  \\ Hausaufgabe 10 - Al-Maweri 13},
 pdfkeywords={},
 pdfsubject={},
 pdfcreator={Emacs 27.1 (Org mode 9.5)}, 
 pdflang={Germanb}}
\begin{document}

\maketitle
\tableofcontents

\setcounter{secnumdepth}{0}
\newcommand{\tuple}[1]{\left(#1\right)}
\newcommand{\R}{\mathbb{R}}
\newcommand{\Z}{\mathbb{Z}}
\newcommand{\Q}{\mathbb{Q}}
\newcommand{\N}{\mathbb{N}}
\newcommand{\C}{\mathbb{C}}

\makeatletter
\renewcommand*\env@matrix[1][*\c@MaxMatrixCols c]{%
\hskip -\arraycolsep
\let\@ifnextchar\new@ifnextchar
\array{#1}}
\makeatother

\pagebreak

\section{Aufgabe 1}
\label{sec:org88bbaf0}
\subsection{a)}
\label{sec:org86509a3}
z.z.:

\begin{equation*}
    \forall n \in \N, \sum_{k = 1}^{n} 2 k^3 = \frac{n^2(n+1)^2}{2}
\end{equation*}

IA: Es gilt für \(n = 0\)

\begin{equation*}
    \sum_{k = 1}^{0} 2 k^3 = 0 = \frac{0^2(0+1)^2}{2}
\end{equation*}

IV: Es gelte für ein festes beliebiges \(n \in \N\)

\begin{equation*}
    \sum_{k = 1}^{n} 2 k^3 = \frac{n^2(n+1)^2}{2}
\end{equation*}

IB: Dann gilt

\begin{equation*}
    \sum_{k = 1}^{n+1} 2 k^3 = \frac{(n+1)^2(n+2)^2}{2}
\end{equation*}

IS: Es gilt

\begin{align*}
    \sum_{k = 1}^{n+1} 2 k^3 &= \sum_{k = 1}^{n} 2 k^3 + 2(n+1)^3 \\
    &= \frac{n^2(n+1)^2}{2} + 2(n+1)^3 \\
    &= \frac{n^2(n+1)^2 + 4(n+1)^3}{2} \\
    &= \frac{n^2(n+1)^2 + (4n + 4)(n+1)^2}{2} \\
    &= \frac{(n^2 + 4n + 4)(n+1)^2}{2} \\
    &= \frac{(n + 2)^2(n+1)^2}{2}
\end{align*}

Damit gilt für alle \(n \in \N\)

\begin{equation*}
    \sum_{k = 1}^{n} 2 k^3 = \frac{n^2(n+1)^2}{2}
\end{equation*}

\subsection{b)}
\label{sec:org66c7fa5}
Es gilt

\begin{align*}
    \int_{0}^{a} 2 x^3 dx &=
        \lim_{n \rightarrow \infty}
            \sum_{i=1}^{n} 2\tuple{(i-1)\frac{a}{n}}^3 \frac{a}{n} \\
    &= \lim_{n \rightarrow \infty}
            \tuple{\frac{a}{n}}^4 \sum_{i=1}^{n} 2(i-1)^3 \\
    &\overset{i=k+1}{=} \lim_{n \rightarrow \infty}
            \tuple{\frac{a}{n}}^4 \sum_{k=0}^{n} 2k^3 \\
    &= \lim_{n \rightarrow \infty}
            \tuple{\frac{a}{n}}^4 \sum_{k=1}^{n} 2k^3 \\
    &= \lim_{n \rightarrow \infty}
            \tuple{\frac{a}{n}}^4 \cdot \frac{n^2(n+1)^2}{2} \\
    &= \lim_{n \rightarrow \infty}
            \frac{a^4 (n^4 + 2n^3 + n^2)}{2 n^4} \\
    &= \frac{a^4}{2} \lim_{n \rightarrow \infty}
            \frac{n^4 + 2n^3 + n^2}{n^4}
        \overset{\text{GWS}}{=} \frac{a^4}{2}
\end{align*}

\section{Aufgabe 2}
\label{sec:orge538fc3}
\subsection{a)}
\label{sec:orgb8e479e}
\begin{equation*}
	\int (3x^{-2} + 4e^{-3x}) dx = -3x^{-1} - \frac{4}{3}e^{-3x} + c, \forall c \in \mathbb{R}
\end{equation*}

\subsection{b)}
\label{sec:org97e29c4}
\begin{align*}
	\int_1^e x\ln(x) dx &= \frac{1}{2}x^2 \cdot \ln(x) - \int \frac{1}{2}x^2 \cdot \frac{1}{x} dx \\
				    &= \frac{1}{2}x^2 \cdot \ln(x) - \int \frac{1}{2}x dx \\
				    &= \frac{1}{2}x^2 \cdot \ln(x) - \frac{1}{4}x^2 \; \bigg|_{1}^{e} \\
				    &= \left(\frac{1}{2}e^2 \cdot 1 - \frac{1}{4}e^2\right) - \left(\frac{1}{2} \cdot 0 - \frac{1}{4}\right) \\
				    &= \frac{1}{4}e^2 + \frac{1}{4}
\end{align*}

\subsection{c)}
\label{sec:org54c996c}
\begin{align*}
	\int e^x \sin(x) dx &= e^x \sin(x) - \int e^x \cos(x) dx \\
				   &= e^x \sin(x) - e^x \cos(x) - \int e^x \sin(x) dx \\
	\Leftrightarrow 2\int e^x \sin(x) dx &= e^x \sin(x) - e^x \cos(x) + c \\
	\Leftrightarrow \int e^x \sin(x) dx &= \frac{1}{2}e^x (\sin(x) - \cos(x)) + \frac{1}{2}c \forall c \in \mathbb{R}
\end{align*}

\section{Aufgabe 3}
\label{sec:orgb90769c}
\subsection{a)}
\label{sec:orge817eb6}
\begin{align*}
    \int 2\cot(x)dx &= 2 \int \cot(x)dx \\
        &= \int \frac{\cos(x)}{\sin(x)}dx
\end{align*}

\begin{math}
    \text{Substituiere } u = \sin(x) \rightarrow \frac{du}{dx} = \cos(x) \rightarrow dx = \frac{1}{\cos(x)}du:
\end{math}

\begin{equation*}
    = \int \frac{1}{u}du = \ln(u)
\end{equation*}

\begin{math}
    \text{Ruecksubstitution von } u = \sin(x) \rightarrow \ln(\sin(x))
\end{math}

\begin{equation*}
    \int 2\cot(x) dx = 2\ln|\sin(x)| + c, c \in \R
\end{equation*}

\subsection{b)}
\label{sec:org9752cba}

\begin{equation*}
    \int x^3e^{-x^4+1}dx
\end{equation*}

\begin{math}
    \text{Substituiere } u = -x^4+1 \rightarrow \frac{du}{dx} = -4x^3 \rightarrow dx = -\frac{1}{4x^3}du:
\end{math}

\begin{equation*}
    = -\frac{1}{4} \int e^u dx = - \frac{1}{4}e^u
\end{equation*}

\begin{math}
    \text{Ruecksubstitution von } u = -x^4+1 \rightarrow -\frac{1}{4}e^{-x^4+1}
\end{math}

\begin{equation*}
    \int x^3e^{-x^4+1}dx = - \frac{e^{-x^4+1}}{4}+c, c \in \R
\end{equation*}

\subsection{c)}
\label{sec:org3914fda}
\begin{equation*}
\int \frac{1+\ln(x)}{x^x}dx
\end{equation*}

Substitution:

\begin{align*}
    u &= \frac{1}{x^x} = x^{-x} = e^{-x\ln(x)} \\
    \frac{du}{dx} &= e^{-x\ln(x)} (-\ln(x) - x \frac{1}{x}) \\
    &= e^{-x\ln(x)} (-1 - \ln(x)) \\
    &= \frac{-(\ln(x)+1)}{x^x} \\
    \Leftrightarrow dx &= \frac{x^x}{-\ln(x)-1}du:
\end{align*}

Damit gilt

\begin{equation*}
    \int \frac{1+\ln(x)}{x^x}dx = - \int du = -u
\end{equation*}

\begin{math}
\text{Ruecksubstitution von  } u = \frac{1}{x^x}:
\end{math}

\begin{equation*}
\int \frac{1+\ln(x)}{x^x}dx = -\frac{1}{x^x}+c, c \in \R
\end{equation*}
\end{document}
