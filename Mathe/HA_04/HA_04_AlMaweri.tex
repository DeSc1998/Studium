% Created 2020-12-09 Mi 13:10
% Intended LaTeX compiler: pdflatex
\documentclass[a4paper, 11pt]{article}
\usepackage[utf8]{inputenc}
\usepackage[T1]{fontenc}
\usepackage{graphicx}
\usepackage{grffile}
\usepackage{longtable}
\usepackage{wrapfig}
\usepackage{rotating}
\usepackage[normalem]{ulem}
\usepackage{amsmath}
\usepackage{textcomp}
\usepackage{amssymb}
\usepackage{capt-of}
\usepackage{hyperref}
\usepackage{braket}
\usepackage[germanb]{babel}
\author{Daniel Geinets (453843), Christopher Neumann (409098), Dennis Schulze (458415)}
\date{\today}
\title{Analysis I und Lineare Algebra für Ingenieurwissenschaften \large  \\ Hausaufgabe 04 - Al-Maweri 13}
\hypersetup{
 pdfauthor={Daniel Geinets (453843), Christopher Neumann (409098), Dennis Schulze (458415)},
 pdftitle={Analysis I und Lineare Algebra für Ingenieurwissenschaften \large  \\ Hausaufgabe 04 - Al-Maweri 13},
 pdfkeywords={},
 pdfsubject={},
 pdfcreator={Emacs 26.3 (Org mode 9.4)}, 
 pdflang={Germanb}}
\begin{document}

\maketitle
\tableofcontents

\setcounter{secnumdepth}{0}
\newcommand{\tuple}[1]{\left(#1\right)}
\renewcommand{\cfrac}[3]{#1 \tuple{\frac{#2}{#3}}}
\newcommand{\R}{\mathbb{R}}
\newcommand{\Z}{\mathbb{Z}}
\newcommand{\Q}{\mathbb{Q}}
\newcommand{\N}{\mathbb{N}}
\newcommand{\C}{\mathbb{C}}

\pagebreak

\section{Aufgabe 1}
\label{sec:org7eb57be}
Es gilt

$$ x^3 + 3x^2 + 3x + 1 = (x + 1)^3 $$

daruas folgt dann der Ansatz mit \(A, B, C \in \C\)

$$ \frac{1}{(x + 1)^3} = \frac{A}{x+1} + \frac{B}{(x+1)^2} + \frac{C}{(x+1)^3} $$

wo für \(C\) mit \(x = -1\) gilt

$$ 1 = C $$

damit gilt

\begin{align*}
    \frac{1}{(x + 1)^3} &= \frac{A}{x+1} + \frac{B}{(x+1)^2} + \frac{1}{(x+1)^3} \\
    &= \frac{A(x+1)^2}{(x+1)^3} + \frac{B(x+1)}{(x+1)^3} + \frac{1}{(x+1)^3} \\
    &= \frac{Ax^2 + 2Ax + A + Bx + B + 1}{(x+1)^3}
\end{align*}

daraus ergibt sich das Gleichungssystem

\begin{align*}
    1 &= A + B + 1 \Rightarrow A = -B \\
    0 &= 2A + B \\
    0 &= A \Rightarrow B = 0
\end{align*}

daraus folgt, dass der Ausgangsterm seine eigene Partialbruchzerlegung ist sowohl
im reellen als auch im komplexen.

\pagebreak

\section{Aufgabe 2}
\label{sec:orgbcbe81c}
\subsection{a)}
\label{sec:org8a715b9}
Es gilt

\begin{align*}
    T_1 &= \left\{ \begin{bmatrix} x_1 \\ x_2 \end{bmatrix} \in \R^2 \Big| 2 x_1 = x_2 \right\} \\
    &= \left\{ \begin{bmatrix} x \\ 2x \end{bmatrix} \Big| x \in \R \right\} \\
    &= \left\{ x \begin{bmatrix} 1 \\ 2 \end{bmatrix} \Big| x \in \R \right\} \\
    &= \text{span} \left\{ \begin{bmatrix} 1 \\ 2 \end{bmatrix} \right\} \\
\end{align*}

dann gilt mit \(v, w \in T_1\) und \(r, x, y, a, b \in \R\)

\begin{align*}
    0 \begin{bmatrix} 1 \\ 2 \end{bmatrix} &= \begin{bmatrix} 0 \\ 0 \end{bmatrix} \in T_1 \\
    r v &= r \begin{bmatrix} x \\ y \end{bmatrix} = \begin{bmatrix} rx \\ ry \end{bmatrix} \\
    &= \begin{bmatrix} rx \\ 2rx \end{bmatrix} \in T_1 \\
    v + w &= \begin{bmatrix} x \\ y \end{bmatrix} + \begin{bmatrix} a \\ b \end{bmatrix}
    = \begin{bmatrix} x \\ 2x \end{bmatrix} + \begin{bmatrix} a \\ 2a \end{bmatrix}
    = \begin{bmatrix} x + a \\ 2(x + a) \end{bmatrix} \in T_1
\end{align*}

daraus folgt

$$ T_1 \subseteq \R^2 $$

\pagebreak

Bei \(T_2\) gilt mit \(v, w \in T_1\) und \(r, x, y, a, b \in \R\)

\begin{align*}
    0 \begin{bmatrix} x \\ y \end{bmatrix} &= \begin{bmatrix} 0 \\ 0 \end{bmatrix} \in T_2 \\
    r v &= r \begin{bmatrix} x \\ y \end{bmatrix} = \begin{bmatrix} rx \\ ry \end{bmatrix} \not\in T_2 \\
    \text{gilt, weil es für } 2 \begin{bmatrix} 1 \\ 0 \end{bmatrix} &= \begin{bmatrix} 2 \\ 0 \end{bmatrix} \text{ zum beispiel nicht erfüllt ist. } \\
    v + w &= \begin{bmatrix} x \\ y \end{bmatrix} + \begin{bmatrix} a \\ b \end{bmatrix}
    = \begin{bmatrix} x \\ 2x \end{bmatrix} + \begin{bmatrix} a \\ 2a \end{bmatrix}
    = \begin{bmatrix} x + a \\ 2(x + a) \end{bmatrix} \not\in T_2 \\
    \text{gilt, weil es für } \begin{bmatrix} 1 \\ 0 \end{bmatrix} + \begin{bmatrix} 1 \\ 0 \end{bmatrix}
    &= \begin{bmatrix} 2 \\ 0 \end{bmatrix} \text{ zum beispiel nicht erfüllt ist. }
\end{align*}

daraus folgt

$$ T_2 \not\subseteq \R^2 $$

Die Menge \(T_3\) ist keine Teilmenge von \(\R^2\),
da \(T_3 = T_1 \cap T_2\) gilt und \(T_2\) keine Teilmenge von \(\R^2\) ist.

\subsection{b)}
\label{sec:org41719d4}
Es gilt für \(f, g \in T\) und \(r, x \in \R\)

\begin{align*}
    0 \cdot f(x) &= 0 \in T \\
    r \cdot f(1) &= r \cdot 0 = 0 \Rightarrow r \cdot f \in T \\
    f(1) + g(1) &= 0 + 0 = 0 \Rightarrow f + g \in T
\end{align*}

damit gilt

$$ T \subseteq V $$

\section{Aufgabe 3}
\label{sec:orgadeda94}
\subsection{a)}
\label{sec:orgfdad9e6}
$$ \overrightarrow{v} = \begin{bmatrix} 1 \\ 0 \\ -1\end{bmatrix} = \overrightarrow{v_1} + \overrightarrow{v_2} - \overrightarrow{v_3} =
\begin{bmatrix} 1 \\ 0 \\ 0\end{bmatrix} + \begin{bmatrix} 0 \\ 1 \\ 2\end{bmatrix} - \begin{bmatrix} 0 \\ 1 \\ 3\end{bmatrix} =  \begin{bmatrix} 1 \\ 0 \\ -1\end{bmatrix} $$

\subsection{b)}
\label{sec:org7ac08d5}
\begin{align*}
    p(z) &= z^2 + 2z + 1 + i \\
    &= 0,5(2z^2 - 1) + 2z + 1,5 + i \\
    &= 0,5T_2(z) + 2T_1(z) + (1,5 + i)T_0(z)
\end{align*}

\subsection{c)}
\label{sec:orgd7e906d}
$$ \cos(x + \pi) = \cos(x)\cos(\pi) - \sin(x)\sin(\pi) = -\cos(x) $$

\section{Aufgabe 4}
\label{sec:org25f4c60}
\subsection{a)}
\label{sec:orge55b814}
Nein, denn
$$ \lambda_1\begin{bmatrix} 1 \\ 0\end{bmatrix} + \lambda_2\begin{bmatrix} 2 \\ 0\end{bmatrix} \ne \begin{bmatrix} 0 \\ 1\end{bmatrix} \in \mathbb{R}^2 \text{ mit } \lambda_1, \lambda_2 \in \R $$

\subsection{b)}
\label{sec:orgf4eaaec}
$$ \text{Basis}(\mathbb{R}^2 ) = \left\{ \begin{bmatrix} 1 \\ 0\end{bmatrix}, \begin{bmatrix} 0 \\ 1\end{bmatrix} \right\}
\text{ mit } \begin{bmatrix} 1 \\ 0\end{bmatrix} =
\begin{bmatrix} 1 \\ 1\end{bmatrix} - \begin{bmatrix} 0 \\ 1\end{bmatrix}, \begin{bmatrix} 0 \\ 1\end{bmatrix} = \begin{bmatrix} 0 \\ 1\end{bmatrix} $$
\subsection{c)}
\label{sec:org21afae1}
$$ \left\{ \begin{bmatrix} 1 \\ 1\end{bmatrix}, \begin{bmatrix} 0 \\ 1 \end{bmatrix} \right\} \text{ ist EZS von } \R^2 $$
\end{document}
