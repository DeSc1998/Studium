% Created 2021-01-06 Mi 16:58
% Intended LaTeX compiler: pdflatex
\documentclass[a4paper, 11pt]{article}
\usepackage[utf8]{inputenc}
\usepackage[T1]{fontenc}
\usepackage{graphicx}
\usepackage{grffile}
\usepackage{longtable}
\usepackage{wrapfig}
\usepackage{rotating}
\usepackage[normalem]{ulem}
\usepackage{amsmath}
\usepackage{textcomp}
\usepackage{amssymb}
\usepackage{capt-of}
\usepackage{hyperref}
\usepackage{braket}
\usepackage[germanb]{babel}
\author{Daniel Geinets (453843), Christopher Neumann (409098), Dennis Schulze (458415)}
\date{\today}
\title{Analysis I und Lineare Algebra für Ingenieurwissenschaften \large  \\ Hausaufgabe 05 - Al-Maweri 13}
\hypersetup{
 pdfauthor={Daniel Geinets (453843), Christopher Neumann (409098), Dennis Schulze (458415)},
 pdftitle={Analysis I und Lineare Algebra für Ingenieurwissenschaften \large  \\ Hausaufgabe 05 - Al-Maweri 13},
 pdfkeywords={},
 pdfsubject={},
 pdfcreator={Emacs 27.1 (Org mode 9.5)}, 
 pdflang={Germanb}}
\begin{document}

\maketitle
\tableofcontents

\setcounter{secnumdepth}{0}
\newcommand{\tuple}[1]{\left(#1\right)}
\renewcommand{\cfrac}[3]{#1 \tuple{\frac{#2}{#3}}}
\newcommand{\R}{\mathbb{R}}
\newcommand{\Z}{\mathbb{Z}}
\newcommand{\Q}{\mathbb{Q}}
\newcommand{\N}{\mathbb{N}}
\newcommand{\C}{\mathbb{C}}

\makeatletter
\renewcommand*\env@matrix[1][*\c@MaxMatrixCols c]{%
\hskip -\arraycolsep
\let\@ifnextchar\new@ifnextchar
\array{#1}}
\makeatother

\pagebreak

\section{Aufgabe 1}
\label{sec:org8725fd7}
\subsection{a)}
\label{sec:orgfdd8f53}
$$f_{B_1,B_1} = \left[\text{K}_{B_1}(f(b_1)), \text{K}_{B_1}(f(b_2)), \text{K}_{B_1}(f(b_3))\right]$$
1.Spalte:
$$\text{K}_{B_1}(f(b_1)) = \text{K}_{B_1}\left(\begin{bmatrix} 2 \\ 0 \\ -2 \end{bmatrix}\right) = 2 \cdot e_1 + 0 \cdot e_2 - 2 \cdot e_3
= \begin{bmatrix} 2 \\ 0 \\ -2 \end{bmatrix} = v_1$$

2.Spalte:
$$\text{K}_{B_1}(f(b_2)) = \text{K}_{B_1}\left(\begin{bmatrix} -4 \\ -2 \\ 2 \end{bmatrix}\right) = - 4 \cdot e_1 - 2 \cdot e_2 + 2 \cdot e_3
= \begin{bmatrix} -4 \\ -2 \\ 2 \end{bmatrix}$$

3.Spalte(siehe 1.Spalte):
$$\text{K}_{B_1}(f(b_1)) = \text{K}_{B_1}\left(\begin{bmatrix} -2 \\ 1 \\ 4 \end{bmatrix}\right) = v_3$$

$$f_{B_1,B_1} = \begin{bmatrix} 2 & -4 & -2 \\ 0 & -2 & 1 \\ -2 & 2 & 4 \end{bmatrix}$$

\subsection{b)}
\label{sec:org92cfec4}
allg:
$$\text{id}_{B_2,B_1} = \text{K}_{B_1} \circ \text{K}_{B_2}^{-1} = \text{K}_{B_1}(\text{K}_{B_2}^{-1}(e_i)) = \text{K}_{B_1}(v_j)$$

1.Spalte:
$$\text{K}_{B_1}(\text{K}_{B_2}^{-1}(e_1)) =  \text{K}_{B_1}(v_1) =  \text{K}_{B_1}\left(\begin{bmatrix} 2 \\ 0 \\ -2 \end{bmatrix}\right) = \text{K}_{B_1}(2 \cdot e_1 + 0 \cdot e_2 + (-2) \cdot e_3)$$
$$\Rightarrow \begin{bmatrix} 2 \\ 0 \\ -2 \end{bmatrix} = v_1$$

2.Spalte:
$$\text{K}_{B_1}(\text{K}_{B_2}^{-1}(e_2)) =  \text{K}_{B_1}(v_2) = \begin{bmatrix} 4 \\ 2 \\ -2 \end{bmatrix} = v_2$$

3.Spalte:
$$\text{K}_{B_1}(\text{K}_{B_2}^{-1}(e_3)) =  \text{K}_{B_1}(v_3) = \begin{bmatrix} -2 \\ 1 \\ 4 \end{bmatrix} = v_3$$

$$\text{id}_{B_2,B_1} = \begin{bmatrix} 2 & 4 & -2 \\ 0 & 2 & 1 \\ -2 & -2 & 4 \end{bmatrix}$$

\subsection{c)}
\label{sec:org07f7b46}
allg:
$$\text{id}_{B_1,B_2} = \text{K}_{B_2} \circ \text{K}_{B_1}^{-1} = \text{K}_{B_2}(\text{K}_{B_1}^{-1}(e_i)) = \text{K}_{B_2}(b_j)$$

1.Spalte:
$$ \text{K}_{B_2}(b_1) =  \text{K}_{B_2}\left(\begin{bmatrix} 1 \\ 0 \\ 0 \end{bmatrix}\right)
= \text{K}_{B_2}\left(\lambda_1\begin{bmatrix} 2 \\ 0 \\ -2 \end{bmatrix} +
    \lambda_2\begin{bmatrix} 4 \\ 2 \\ -2 \end{bmatrix} +
    \lambda_3\begin{bmatrix} -2 \\ 1 \\ 4 \end{bmatrix}\right) $$

LGS:
$$\text{I: } 1 = 2\lambda_1 + 4\lambda_2 - 2\lambda_3 $$
$$\text{II: } 0 = 0\lambda_1 + 2\lambda_2 + \lambda_3 \Leftrightarrow \lambda_3 = -2\lambda_2 $$
$$\text{III: } 0 = -2\lambda_1 - 2\lambda_2 + 4\lambda_3 $$
Also:
$$\text{III: } 0 = -2\lambda_1 - 2\lambda_2 - 8\lambda_2 \Leftrightarrow \lambda_1 = -5\lambda-2$$
$$\Rightarrow \text{I: } 1 = -10\lambda_2 + 4\lambda_2 + 4\lambda_2$$
$$2\lambda_2 = -1 \Leftrightarrow \lambda_2 = -\frac{1}{2} \Rightarrow \lambda_1 = \frac{5}{2} \Rightarrow \lambda_3 = 1$$
$$\Rightarrow \text{K}_{B_2}\left(\frac{5}{2}\begin{bmatrix} 2 \\ 0 \\ -2   \end{bmatrix} -
    \frac{1}{2}\begin{bmatrix} 4 \\ 2 \\ -2 \end{bmatrix} +
    1\begin{bmatrix} -2 \\ 1 \\ 4 \end{bmatrix}\right) =
\begin{bmatrix} \frac{5}{2} \\ -\frac{1}{2} \\ 1 \end{bmatrix}$$

2.Spalte (auf die selbe Weise, wie 1.Spalte):
$$\lambda_1 = -3, \lambda_2 = 1, \lambda_3 = -1$$
$$\Rightarrow \text{K}_{B_2}\left(\begin{bmatrix} 0 \\ 1 \\ 0 \end{bmatrix}\right) =
\text{K}_{B_2}\left(- 3\begin{bmatrix} 2 \\ 0 \\ -2 \end{bmatrix} +
    1\begin{bmatrix} 4 \\ 2 \\ -2 \end{bmatrix} -
    1\begin{bmatrix} -2 \\ 1 \\ 4 \end{bmatrix}\right) =
    \begin{bmatrix} -3 \\ 1 \\ -1 \end{bmatrix}$$

\pagebreak

3.Spalte (siehe 1.Spalte):
$$\lambda_1 = 2, \lambda_2 = -\frac{1}{2}, \lambda_3 = 1$$
$$\Rightarrow \text{K}_{B_2}\left(\begin{bmatrix} 0 \\ 0 \\ 1 \end{bmatrix}\right) =
\text{K}_{B_2}\left(2\begin{bmatrix} 2 \\ 0 \\ -2 \end{bmatrix} -
\frac{1}{2}\begin{bmatrix} 4 \\ 2 \\ -2 \end{bmatrix} +
1\begin{bmatrix} -2 \\ 1 \\ 4 \end{bmatrix}\right)
= \begin{bmatrix} 2 \\ -\frac{1}{2} \\ 1 \end{bmatrix}$$

$$\text{id}_{B_1,B_2} = \begin{bmatrix} \frac{5}{2} & -3 & 2 \\ -\frac{1}{2} & 1 & -\frac{1}{2} \\ 1 & -1 & 1 \end{bmatrix}$$

\subsection{d)}
\label{sec:org3b23af9}
\begin{align*}
f_{B_2,B_2} &= \text{id}_{B_1, B_2} \cdot f_{B_1,B_1} \cdot \text{id}_{B_2, B_1} \\
&= \begin{bmatrix} \frac{5}{2} & -3 & 2 \\ -\frac{1}{2} & 1 & -\frac{1}{2} \\ 1 & -1 & 1 \end{bmatrix} \cdot \begin{bmatrix} 2 & -4 & -2 \\ 0 & -2 & 1 \\ -2 & 2 & 4 \end{bmatrix} \cdot \begin{bmatrix} 2 & 4 & -2 \\ 0 & 2 & 1 \\ -2 & -2 & 4 \end{bmatrix} \\
&= \begin{bmatrix} 1 & 0 & 0 \\ 0 & -1 & 0 \\ 0 & 0 & 1 \end{bmatrix} \cdot
\begin{bmatrix} 2 & 4 & -2 \\ 0 & 2 & 1 \\ -2 & -2 & 4 \end{bmatrix}
= \begin{bmatrix} 2 & 4 & -2 \\ 0 & -2 & -1 \\ -2 & -2 & 4 \end{bmatrix}
= f_{B_2,B_2}
\end{align*}

\section{Aufgabe 2}
\label{sec:org23aaed3}
\subsection{a)}
\label{sec:org99edcaa}
Da \(f\) linear ist gelten \(f(x+y) = f(x) + f(y)\) und
\(f(r \cdot x) = r \cdot f(x)\). Damit gilt

\begin{align*}
    f(2x) &= 2f(x + 3) - 6 f(1) \\
    &= 2(-2x + 1) - 6(x + 2) \\
    &= -4x + 2 - 6x - 12 \\
    &= -10x - 10
\end{align*}

\subsection{b)}
\label{sec:orgd01c7ba}
\begin{align*}
    f(1) &= x + 2 \\
    f(x) &= -5x -5
\end{align*}

\pagebreak

Seien \(a, b \in \R\), dann gilt

\begin{align*}
    f(ax + b) &= f(ax) + f(b) = a f(x) + b f(1) \\
    &= a(-5x - 5) + b(x + 2) \\
    &= -5ax - 5a + bx + 2b \\
    &= (b - 5a)x + (2b - 5a)
\end{align*}

also gilt

$$ f: \begin{bmatrix} a \\ b \end{bmatrix} \mapsto
    \begin{bmatrix} b - 5a \\ 2b - 5a \end{bmatrix} $$

\(f\) lässt als Matrix wie folgt darstellen

$$ f = \begin{bmatrix} -5 & 1 \\ -5 & 2 \end{bmatrix} $$

Überprüfung auf lineare Abhängigkeit mit \(r_1, r_2 \in \R\)

$$ \begin{bmatrix} 0 \\ 0 \end{bmatrix} =
    r_1 \begin{bmatrix} -5 \\ -5 \end{bmatrix} +
    r_2 \begin{bmatrix} 1 \\ 2 \end{bmatrix} $$

daraus ergibt sich ein LGS, dass zu einem Widerspruch führt

\begin{align*}
    \text{(I) } 0 &= r_2 - 5r_1 \\
    \text{(II) } 0 &= 2r_2 - 5r_1 \\
\end{align*}

daraus folgt, dass die Spalten linear unabhängig sind, Woraus folgt

$$ \dim(\text{Bild}(f)) = 2 $$

\subsection{c)}
\label{sec:orgcfe6267}
Überführen der Matrix

\begin{align*}
    \begin{bmatrix}[cc|c]
        -5 & 1 & 0 \\
        -5 & 2 & 0
    \end{bmatrix}
    &\xrightarrow{\text{II} - \text{I}}
    \begin{bmatrix}[cc|c]
        -5 & 1 & 0 \\
        0 & 1 & 0
    \end{bmatrix} \\
    \xrightarrow{\text{I} - \text{II}}
    \begin{bmatrix}[cc|c]
        -5 & 0 & 0 \\
        0 & 1 & 0
    \end{bmatrix}
    &\xrightarrow{ \frac{-1}{5}\text{I}}
    \begin{bmatrix}[cc|c]
        1 & 0 & 0 \\
        0 & 1 & 0
    \end{bmatrix}
\end{align*}

Für Kern(\(f\)) gibt es also nur eine Lösung, weshalb gilt

$$ \text{Kern}(f) = \left\{ \begin{bmatrix} 0 \\ 0 \end{bmatrix} \right\} $$

daraus folgt

$$ B_{\text{Kern}(f)} = \{\} $$

\subsection{d)}
\label{sec:orgf380e64}
$$ f_{B_1,B_2} = \left[\text{K}_{B_2}(f(b_1)), \text{K}_{B_2}(f(b_2))\right]$$

1.Spalte:
$$ \text{K}_{B_2}(f(b_1)) = \text{K}_{B_2}(- 2x + 1) = 1 \cdot b_{2_1} + 0 \cdot b_{2_0} = \begin{bmatrix} 1 \\ 0 \end{bmatrix}$$

2.Spalte:
$$ \text{K}_{B_2}(f(b_2)) = \text{K}_{B_2}(x + 2) = 0 \cdot b_{2_1} + 1 \cdot b_{2_0} = \begin{bmatrix} 0 \\ 1 \end{bmatrix}$$

$$ f_{B_1,B_2} = \begin{bmatrix} 1 & 0 \\ 0 & 1 \end{bmatrix}$$

\section{Aufgabe 3}
\label{sec:orgd9b7002}
Es gilt

$$ (3k + 1)(3k + 4) =
    9 \tuple{k + \frac{1}{3}} \tuple{k + \frac{4}{3}} $$

daraus ergibt sich dann der folgende Ansatz

$$ \frac{\frac{1}{3}}{(k + \frac{1}{3})(k + \frac{4}{3})} =
    \frac{A}{k + \frac{1}{3}} +
    \frac{B}{k + \frac{4}{3}} $$

für \(A\) mit \(k = -\frac{1}{3}\) gilt

$$ \frac{1}{3} \cdot \frac{1}{-\frac{1}{3} + \frac{4}{3}} = A = \frac{1}{3} $$

und für \(B\) mit \(k = -\frac{4}{3}\) gilt

$$ \frac{1}{3} \cdot \frac{1}{-\frac{4}{3} + \frac{1}{3}} = B = -\frac{1}{3} $$

\pagebreak

daraus folgt

\begin{align*}
    \sum_{k = 0}^{n} \frac{3}{(3k + 1)(3k + 4)} &=
        \sum_{k = 0}^{n} \tuple{ \frac{1}{3(k + \frac{1}{3})} -
            \frac{1}{3(k + \frac{4}{3})} } \\
    &= \sum_{k = 0}^{n} \tuple{ \frac{1}{3k + 1} - \frac{1}{3k + 4} } \\
    &= \tuple{\frac{1}{1} - \frac{1}{4}} +
        \tuple{\frac{1}{4} - \frac{1}{7}} +
        \tuple{\frac{1}{7} - \frac{1}{10}} + \dots \\
        &+ \tuple{ \frac{1}{3n - 2} - \frac{1}{3n + 1} } +
        \tuple{ \frac{1}{3n + 1} - \frac{1}{3n + 4} } \\
    &= 1 - \frac{1}{3n + 4}
\end{align*}

Für den Grenzwert gilt dann

\begin{align*}
    \lim_{n \Leftrightarrow \infty} \tuple{1 - \frac{1}{3n + 4}} &=
        \lim_{n \rightarrow \infty} 1 -
            \lim_{n \rightarrow \infty} \frac{1}{3n + 4} \\
    &= 1 - \frac{\lim_{n \rightarrow \infty} 1}{\lim_{n \rightarrow \infty} 3n +
            \lim_{n \rightarrow \infty} 4} \\
    &= 1 - \frac{1}{\infty + 4} \overset{\text{GWS}}{=} 1 - 0 = 1
\end{align*}

damit gilt

$$ \lim_{n \rightarrow \infty} \sum_{k = 0}^{n} \frac{3}{(3k + 1)(3k + 4)} = 1 $$

\section{Aufgabe 4}
\label{sec:org71c0e09}
\subsection{a)}
\label{sec:org8bb9735}
IA: Es gilt

$$ x_0 = 1 < \frac{4}{3} $$

IV: Es gelte für ein beliebiges \(n \in \N\)

$$ x_n < \frac{4}{3} $$

IB: Damit gilt dann

$$ x_{n+1} < \frac{4}{3} $$

IS: Es gilt

\begin{align*}
    x_{n+1} &< \frac{4}{3} \\
    \Leftrightarrow \frac{x_n}{4} + 1 &< \frac{4}{3} \\
    \Leftrightarrow \frac{x_n}{4} &< \frac{1}{3} \\
    \Leftrightarrow x_n &< \frac{4}{3} \\
\end{align*}

Folglich gilt dann für alle \(n \in \N\)

$$ x_n < \frac{4}{3} $$

\subsection{b)}
\label{sec:org1addee1}
Zu Zeigen ist für alle \(n \in \N\)

$$ x_n < x_{n+1} $$

IA: Es gilt

$$ x_0 = 1 < x_1 = \frac{1}{4} + 1 = \frac{5}{4} $$

IV: Es gelte für ein beliebiges \(n \in \N\)

$$ x_n < x_{n+1} $$

IB: Damit gilt dann

$$ x_{n+1} < x_{n+2} $$

IS: Es gilt

\begin{align*}
    x_{n+1} &< x_{n+2} \\
    \Leftrightarrow \frac{x_n}{4} + 1 &< \frac{x_{n+1}}{4} + 1 \\
    \Leftrightarrow \frac{x_n}{4} &< \frac{x_{n+1}}{4} \\
    \Leftrightarrow x_n &< x_{n+1} \\
\end{align*}

\pagebreak

Damit gilt für alle \(n \in \N\)

$$ x_n < x_{n+1} $$

Damit ist die Folge \(x_n\) streng monoton wachsend.

\subsection{c)}
\label{sec:org96cd89a}
Die Folge \(x_n\) ist nach oben beschränkt mit \(M = \frac{4}{3}\) (siehe a) \\

Die Folge \(x_n\) ist streng monoton wachsend (siehe b) \\

Daraus folgt, dass die Folge \(x_n\) auch nach unten beschränkt ist mit \(m = 1\) \\

Daraus folgt die Folge \(x_n\) konvergiert.

\subsection{d)}
\label{sec:orgfb37b5c}
Es gilt für die ersten Folgeglieder

\begin{align*}
    x_0 &= 1 \\
    x_1 &= \frac{x_0}{4} + 1 = \frac{1}{4} + 1 \\
    x_2 &= \frac{x_1}{4} + 1 = \frac{1}{16} + \frac{1}{4} + 1 \\
    \vdots
\end{align*}

Man kann die Folge also auch wie folgt schreiben

$$ x_n = \sum_{k = 0}^{n} \frac{1}{4^k} $$

dann gilt

\begin{align*}
    x_n &= \sum_{k = 0}^{n} \frac{1}{4^k}
    = \sum_{k = 0}^{n} \tuple{\frac{1}{4}}^k \\
    &= \frac{1 - \tuple{\frac{1}{4}}^{n+1}}{1 - \frac{1}{4}}
    = \frac{4 - \tuple{\frac{1}{4}}^n}{3}
\end{align*}

\pagebreak

dann gilt für den Grenzwert

\begin{align*}
    \lim_{n \rightarrow \infty} \frac{4 - \tuple{\frac{1}{4}}^n}{3} &=
        \lim_{n \rightarrow \infty} \frac{4}{3} -
        \lim_{n \rightarrow \infty} \frac{\tuple{\frac{1}{4}}^n}{3} \\
    &= \frac{4}{3} - \lim_{n \rightarrow \infty} \frac{1}{3 \cdot 4^n} \\
    & \overset{\text{GWS}}{=} \frac{4}{3} - 0 = \frac{4}{3}
\end{align*}
\end{document}
