% Created 2020-12-16 Mi 14:19
% Intended LaTeX compiler: pdflatex
\documentclass[a4paper, 11pt]{article}
\usepackage[utf8]{inputenc}
\usepackage[T1]{fontenc}
\usepackage{graphicx}
\usepackage{grffile}
\usepackage{longtable}
\usepackage{wrapfig}
\usepackage{rotating}
\usepackage[normalem]{ulem}
\usepackage{amsmath}
\usepackage{textcomp}
\usepackage{amssymb}
\usepackage{capt-of}
\usepackage{hyperref}
\usepackage{braket}
\usepackage[germanb]{babel}
\author{Daniel Geinets (453843), Christopher Neumann (409098), Dennis Schulze (458415)}
\date{\today}
\title{Analysis I und Lineare Algebra für Ingenieurwissenschaften \large  \\ Hausaufgabe 05 - Al-Maweri 13}
\hypersetup{
 pdfauthor={Daniel Geinets (453843), Christopher Neumann (409098), Dennis Schulze (458415)},
 pdftitle={Analysis I und Lineare Algebra für Ingenieurwissenschaften \large  \\ Hausaufgabe 05 - Al-Maweri 13},
 pdfkeywords={},
 pdfsubject={},
 pdfcreator={Emacs 26.3 (Org mode 9.4)}, 
 pdflang={Germanb}}
\begin{document}

\maketitle
\tableofcontents

\setcounter{secnumdepth}{0}
\newcommand{\tuple}[1]{\left(#1\right)}
\renewcommand{\cfrac}[3]{#1 \tuple{\frac{#2}{#3}}}
\newcommand{\R}{\mathbb{R}}
\newcommand{\Z}{\mathbb{Z}}
\newcommand{\Q}{\mathbb{Q}}
\newcommand{\N}{\mathbb{N}}
\newcommand{\C}{\mathbb{C}}

\makeatletter
\renewcommand*\env@matrix[1][*\c@MaxMatrixCols c]{%
\hskip -\arraycolsep
\let\@ifnextchar\new@ifnextchar
\array{#1}}
\makeatother

\pagebreak

\section{Aufgabe 1}
\label{sec:org42b4481}
\subsection{a)}
\label{sec:org440bc7a}
Es gilt

\begin{align*}
    [A|b] = 
    & \left[ \begin{array}{@{}cccc|c@{}}
        -2 & 2i & 0 & 2 & -4 \\
        -4 & 4i & -i & 7 & -8 \\
        1 & i & 2 & -1 & 4
    \end{array} \right] \\
    \xrightarrow{\text{II} - 2\text{I}}
    & \left[ \begin{array}{@{}cccc|c@{}}
        -2 & 2i & 0 & 2 & -4 \\
        0 & 0 & -i & 3 & 0 \\
        1 & i & 2 & -1 & 4
    \end{array} \right] \\
    \xrightarrow{2\text{III} + \text{I}}
    & \left[ \begin{array}{@{}cccc|c@{}}
        -2 & 2i & 0 & 2 & -4 \\
        0 & 0 & -i & 3 & 0 \\
        0 & 4i & 4 & 0 & 4
    \end{array} \right] \\
    \xrightarrow{\text{II} \leftrightarrow \text{III}}
    & \left[ \begin{array}{@{}cccc|c@{}}
        -2 & 2i & 0 & 2 & -4 \\
        0 & 4i & 4 & 0 & 4 \\
        0 & 0 & -i & 3 & 0
    \end{array} \right] \\
    \xrightarrow{\text{II} - 4i\text{III}}
    & \left[ \begin{array}{@{}cccc|c@{}}
        -2 & 2i & 0 & 2 & -4 \\
        0 & 4i & 0 & -12i & 4 \\
        0 & 0 & -i & 3 & 0
    \end{array} \right] \\
    \xrightarrow{\text{I} - \frac{1}{2}\text{II}}
    & \left[ \begin{array}{@{}cccc|c@{}}
        -2 & 0 & 0 & 2+6i & -6 \\
        0 & 2i & 0 & -6i & 2 \\
        0 & 0 & -i & 3 & 0
    \end{array} \right] \\
    \xrightarrow{\frac{-1}{2}\text{I}, \frac{-i}{2}\text{II}, i\text{II}}
    & \left[ \begin{array}{@{}cccc|c@{}}
        1 & 0 & 0 & -1-3i & -3 \\
        0 & 1 & 0 & -3 & i \\
        0 & 0 & 1 & 3i & 0
    \end{array} \right] \\
\end{align*}

\subsection{b)}
\label{sec:org838b339}
Aus dem LGS aus a) folgt mit \(x_1, x_2, x_3, x_4 \in \C\)

\begin{align*}
    \text{(I)  } 0 &= x_1 + (-1 - 3i)x_4 \Rightarrow x_1 = (1 + 3i)x_4 \\
    \text{(II)  } 0 &= x_2 - 3 x_4 \Rightarrow             x_2 = 3 x_4 \\
    \text{(III)  } 0 &= x_3 + 3i x_4 \Rightarrow         x_3 = -3i x_4 \\
\end{align*}

damit gilt dann für den Kern(\(A\))

\begin{align*}
    \text{Kern}(A) &= \left\{ \begin{bmatrix} (1 + 3i)z \\ 3 z \\ -3i z \\ z \end{bmatrix} \Bigg| z \in \C \right\} \\
    &= \text{span} \left\{ \begin{bmatrix} 1 + 3i \\ 3 \\ -3i \\ 1 \end{bmatrix} \right\} \\
    &= \mathbb{L}(A, 0)
\end{align*} 

\subsection{c)}
\label{sec:org360b326}
Es gilt

\begin{align*}
    \begin{bmatrix}
        1 & 0 & 0 & -1-3i \\
        0 & 1 & 0 & -3 \\
        0 & 0 & 1 & 3i
    \end{bmatrix} \cdot
    \begin{bmatrix}
        3 \\
        -i \\
        0 \\
        0
    \end{bmatrix} = \begin{bmatrix} 3 \\ -i \\ 0 \end{bmatrix}
\end{align*}

Vergleiche mit dem LGS aus a)

$$ \begin{bmatrix} 3 \\ -i \\ 0 \\ 0 \end{bmatrix} \in \mathbb{L}(A, b) $$

desweiteren gilt

\begin{align*}
    \begin{bmatrix}
        1 & 0 & 0 & -1-3i \\
        0 & 1 & 0 & -3 \\
        0 & 0 & 1 & 3i
    \end{bmatrix} \cdot
    \begin{bmatrix}
        1 \\
        0 \\
        1 \\
        0
    \end{bmatrix} = \begin{bmatrix} 1 \\ 0 \\ 1 \end{bmatrix} \neq \begin{bmatrix} 3 \\ -i \\ 0 \end{bmatrix}
\end{align*}

Vergleiche mit dem LGS aus a)

$$ \begin{bmatrix} 1 \\ 0 \\ 1 \\ 0 \end{bmatrix} \not\in \mathbb{L}(A, b) $$

\subsection{d)}
\label{sec:orga02fdc8}
aus b):
$$\mathbb{L} = \left\{\begin{bmatrix} (1 + 3i)t \\ 3t \\ 3it \\ t\end{bmatrix}, t \in \mathbb{C}\right\}$$
\newline
aus c):
$$\mathbb{L}(A, b) = \begin{bmatrix} 3 \\ -i \\ 0 \\ 0\end{bmatrix} + \mathbb{L}(A, 0)$$

\section{Aufgabe 2}
\label{sec:org12eb77c}
\subsection{a)}
\label{sec:orge9c695b}
Rang(\(A_1\)) = 3, Rang(\(A_2\)) = 3, Rang(\(A_3\)) = 2, Rang(\(A_4\)) = 2

\subsection{b)}
\label{sec:org0175cf6}

\begin{equation*}
    A_1: \text{span} \left\{ \begin{bmatrix} 2 \\ 0 \\ 0\end{bmatrix}, \begin{bmatrix} 1 \\ 1 \\ 0\end{bmatrix}, \begin{bmatrix} 2 \\ 0 \\ 1\end{bmatrix}\\
    \begin{bmatrix} -2 \\ 4 \\ 3\end{bmatrix} \right\} \neq B(A_1)
\end{equation*}


da:

\begin{equation*}
    \begin{bmatrix} -2 \\ 4 \\ 3\end{bmatrix} = 4\begin{bmatrix} 1 \\ 1 \\ 0\end{bmatrix} +
    3\begin{bmatrix} 2 \\ 0 \\ 1\end{bmatrix} - 6\begin{bmatrix} 2 \\ 0 \\ 0\end{bmatrix}
\end{equation*}

$$A_2: \text{span}\left\{\begin{bmatrix} 1 \\ 0 \\ 0 \\ 0\end{bmatrix}, \begin{bmatrix} 2 \\ 4 \\ 0 \\ 0\end{bmatrix}, \\
\begin{bmatrix} 3 \\ 5 \\ 6 \\ 0\end{bmatrix}\right\} = B(A_2)$$


da:

$$\lambda_1\begin{bmatrix} 1 \\ 0 \\ 0 \\ 0\end{bmatrix} \neq \lambda_2\begin{bmatrix} 2 \\ 4 \\ 0 \\ 0\end{bmatrix} \neq \\
\lambda_3\begin{bmatrix} 3 \\ 5 \\ 6 \\ 0\end{bmatrix}$$

$$A_3: \text{span}\left\{\begin{bmatrix} 11 \\ 0\end{bmatrix}, \begin{bmatrix} 12 \\ 0\end{bmatrix}, \begin{bmatrix} 0 \\ 2\end{bmatrix}\right\} \\
\neq B(A_3)$$


da:

$$\begin{bmatrix} 12 \\ 0\end{bmatrix} = \frac{12}{11}\begin{bmatrix} 11 \\ 0\end{bmatrix}$$

$$A_4: \text{span} \left\{\begin{bmatrix} 3 \\ 0\end{bmatrix}, \begin{bmatrix} 4 \\
2\end{bmatrix}\right\} = B(A_3)$$

da:

$$\lambda_1\begin{bmatrix} 3 \\ 0\end{bmatrix} \neq \lambda_2\begin{bmatrix} 4 \\ 2\end{bmatrix}$$

\subsection{c)}
\label{sec:org4cbcb43}
allg: \(\dim(\text{Kern}(A)) = n - r = n - \text{Rang}(A)\)
$$ \dim(\text{Kern}(A_1)) = 4 - 3 = 1 $$
$$ \dim(\text{Kern}(A_2)) = 3 - 3 = 0 $$

\subsection{d)}
\label{sec:org81092a5}

\begin{equation*}
    \begin{bmatrix} 3 & 4 \\ 0 & 2\end{bmatrix} = 3\begin{bmatrix} 1 & 0 \\ 0 & -2\end{bmatrix} +
    2\begin{bmatrix} 0 & 2 \\ 0 & 1\end{bmatrix} + 6\begin{bmatrix} 0 & 0 \\ 0 & 1\end{bmatrix} +
    0\begin{bmatrix} 0 & 0 \\ 1 & 0\end{bmatrix}
\end{equation*}


\begin{equation*}
    \Rightarrow K_B(A_4) = \begin{bmatrix} 6 \\ 2 \\ 3 \\ 0\end{bmatrix}
\end{equation*}

\section{Aufgabe 3}
\label{sec:orgc6b698c}
Gegeben seien
$$f: \mathbb{C}^3 \rightarrow \mathbb{C}^2, \begin{bmatrix} a \\ b \\ c\end{bmatrix} \mapsto \begin{bmatrix} ia + b \\ 2ic\end{bmatrix}, \\
g: \mathbb{C}^3 \rightarrow \mathbb{C}^2, \begin{bmatrix} a \\ b \\ c\end{bmatrix} \mapsto \begin{bmatrix} a \\ c + 1\end{bmatrix}$$

\subsection{a)}
\label{sec:org38f0458}
f:
$$\overrightarrow{v} + \overrightarrow{w} \in \mathbb{C}^3, \overrightarrow{v} =   \begin{bmatrix} a_1 \\ b_ 1\\ c_1\end{bmatrix}, \\
\overrightarrow{w} = \begin{bmatrix} a_2 \\ b_2 \\ c_2\end{bmatrix}$$

\begin{align*}
    f(\overrightarrow{v} + \overrightarrow{w}) &= f \left( \begin{bmatrix} a_1 + a_2 \\ b_ 1 + b_2\\ c_1 + c_2\end{bmatrix} \right) \\
    &= \begin{bmatrix} i(a_1 + a_2) + b_1 + b_2 \\ 2i(c_ 1 + c_2)\end{bmatrix} =
    \begin{bmatrix} ia_1 + b_1 \\ 2ic_1 \end{bmatrix} + \begin{bmatrix} ia_2 + b_2 \\ 2ic_2 \end{bmatrix} \\
    &= f(\overrightarrow{v}) + f(\overrightarrow{w})
\end{align*}

$$\Rightarrow\text{f ist geschlossen in Addition}$$

\begin{align*}
    \lambda \in \mathbb{R}, f(\lambda\overrightarrow{v}) &= f \left( \lambda\begin{bmatrix} a_1 \\ b_ 1\\ c_1\end{bmatrix} \right) \\
    &= \begin{bmatrix} \lambda ia_1 + \lambda b_1 \\ \lambda 2ic_ 1 \end{bmatrix} = \begin{bmatrix} \lambda(ia_1 + b_1) \\ \lambda 2ic_ 1 \end{bmatrix} \\
    &= \lambda f(\overrightarrow{v})
\end{align*}

$$\Rightarrow\text{f ist homogen}$$
$$\Rightarrow\text{f ist additiv und homogen}\Rightarrow\text{f ist linear}$$

g:
$$\lambda \in \mathbb{R}, g(\lambda\overrightarrow{v}) = g \left( \lambda\begin{bmatrix} a_1 \\ b_ 1\\ c_1\end{bmatrix} \right)  \\
= \begin{bmatrix} \lambda a_1 \\ \lambda c_1 + 1\end{bmatrix} \neq \begin{bmatrix} \lambda a_1 \\ \lambda(c_1 + 1)\end{bmatrix} = \\
\lambda\begin{bmatrix} a_1 \\ c_1 + 1\end{bmatrix} =
\lambda g(\overrightarrow{v})$$
$$\Rightarrow\text{nicht homogen}$$
$$\Rightarrow\text{g ist nicht linear}$$


\subsection{b)}
\label{sec:org8aa4275}
$$f\left(\begin{bmatrix} a \\ b \\ c\end{bmatrix}\right) = A\begin{bmatrix} a \\ b \\ c\end{bmatrix} \text{für alle} \begin{bmatrix} a \\ b \\ c\end{bmatrix} \in \mathbb{C}^3$$
$$f: \begin{bmatrix} a \\ b \\ c\end{bmatrix} \mapsto \begin{bmatrix} ia + b\\ 2ic\end{bmatrix} \Rightarrow \lambda_1 = i, \lambda_2 = 1, \lambda_3 = 2i$$
$$\Rightarrow A = \begin{bmatrix} i & 1 & 0 \\ 0 & 0 & 2i\end{bmatrix}$$

\subsection{c)}
\label{sec:org3d0c9fa}
Kern(f) = Kern(A)
Also:
$$\text{Kern}(A): A\overrightarrow{x} = \overrightarrow{0}$$
$$\Rightarrow \begin{bmatrix}[ccc|c] i & 1 & 0 & 0\\ 0 & 0 & 2i & 0\end{bmatrix} \xrightarrow{-i\text{I}} \\
\begin{bmatrix}[ccc|c] 1 & -i & 0 & 0\\ 0 & 0 & 2i & 0\end{bmatrix}  \xrightarrow{-\frac{1}{2}i\text{II}} \\
\begin{bmatrix}[ccc|c] 1 & -i & 0 & 0\\ 0 & 0 & 1 & 0\end{bmatrix}$$
$$\Rightarrow x_1 = it, x_2 = t, x_3 = 0$$
$$\text{Kern}(A) = \text{span}\left\{\begin{bmatrix} it \\ t\\ 0\end{bmatrix}\right\}$$
Für \(x_2 = t = 1\):
$$\Rightarrow \begin{bmatrix} i \\ 1\\ 0\end{bmatrix} \Rightarrow \left\{\begin{bmatrix} i \\ 1\\ 0\end{bmatrix}\right\} = \text{B}(\text{Kern}(f))$$
\(\left\{\begin{bmatrix} i \\ 1\\ 0\end{bmatrix}\right\}\) ist eine Basis von Kern(f)


\subsection{d)}
\label{sec:org0fce944}
Ist f injektiv/surjektiv?
\newline
\newline
f ist nicht injektiv, da: 
$$\dim(\text{Kern}(f)) = 1 \neq 0$$
f ist surjektiv, da: 
$$\dim(\C^3) - \dim(\text{Kern}(f)) = 3 -1 = 2 = \dim(\text{Bild}(f)) = \dim(\C^2)$$
\end{document}
