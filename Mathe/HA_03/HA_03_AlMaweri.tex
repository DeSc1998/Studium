% Created 2020-12-02 Mi 13:29
% Intended LaTeX compiler: pdflatex
\documentclass[a4paper, 11pt]{article}
\usepackage[utf8]{inputenc}
\usepackage[T1]{fontenc}
\usepackage{graphicx}
\usepackage{grffile}
\usepackage{longtable}
\usepackage{wrapfig}
\usepackage{rotating}
\usepackage[normalem]{ulem}
\usepackage{amsmath}
\usepackage{textcomp}
\usepackage{amssymb}
\usepackage{capt-of}
\usepackage{hyperref}
\usepackage{braket}
\usepackage[germanb]{babel}
\author{Daniel Geinets (453843), Christopher Neumann (409098), Dennis Schulze (458415)}
\date{\today}
\title{Analysis I und Lineare Algebra für Ingenieurwissenschaften \large  \\ Hausaufgabe 03 - Al-Maweri 13}
\hypersetup{
 pdfauthor={Daniel Geinets (453843), Christopher Neumann (409098), Dennis Schulze (458415)},
 pdftitle={Analysis I und Lineare Algebra für Ingenieurwissenschaften \large  \\ Hausaufgabe 03 - Al-Maweri 13},
 pdfkeywords={},
 pdfsubject={},
 pdfcreator={Emacs 26.3 (Org mode 9.4)}, 
 pdflang={Germanb}}
\begin{document}

\maketitle
\tableofcontents

\setcounter{secnumdepth}{0}
\newcommand{\tuple}[1]{\left(#1\right)}
\renewcommand{\cfrac}[3]{#1 \tuple{\frac{#2}{#3}}}
\newcommand{\R}{\mathbb{R}}
\newcommand{\Z}{\mathbb{Z}}
\newcommand{\Q}{\mathbb{Q}}
\newcommand{\N}{\mathbb{N}}
\newcommand{\C}{\mathbb{C}}

\pagebreak

\section{Aufgabe 1}
\label{sec:org857a317}
\subsection{a)}
\label{sec:orga9c13bc}
\subsubsection{i)}
\label{sec:org6610eb2}
$$ z^3 = -27i = 27 e^{-\frac{\pi}{2}i} $$

Es gilt für den Betrag \(r\) und das Argument \(\phi_n\) mit \(n \in \set{0, 1, 2}\)
von \(z\)

\begin{align*}
    r &= \sqrt[3]{27} = 3 \\
    \phi_n &= -\frac{\pi}{2 \cdot 3} + \frac{2n\pi}{3} \\
    &= \frac{-\pi + 4n \pi}{6} = \frac{\pi(4n - 1)}{6}
\end{align*}

Dann gilt

$$ z = 3 e^{\frac{\pi(4n - 1)}{6}i} \text{ mit } n = 0, 1, 2 $$

\subsubsection{ii)}
\label{sec:org3aad591}
$$ 2z^4 + 4(\sqrt{12} - 2i)z^2 + 16 - 8\sqrt{12}i = 0 $$

Es gilt für den ersten Koeffizienten mit \(r, \phi \in \R\)

\begin{align*}
    r &= \sqrt{12 + (-2)^2} = 4 \\
    \tan(\phi) &= \frac{-2}{\sqrt{12}} = \frac{-1}{\sqrt{3}} \\
    &= \frac{-1}{\sqrt{3}} \\
    \Leftrightarrow \phi &= \arctan \tuple{\frac{-1}{\sqrt{3}}} = -\frac{\pi}{6}
\end{align*}

damit gilt

$$ \sqrt{12} - 2i = 4 e^{-\frac{\pi}{6}i} $$

analog gilt dann für den zweiten Koeffizienten

$$ 16 - 8\sqrt{12}i = 32 e^{-\frac{\pi}{3}i} $$

\pagebreak

dann gilt

\begin{align*}
    2z^4 + 4(\sqrt{12} - 2i)z^2 + 16 - 8\sqrt{12}i &= 0 \\
    \Leftrightarrow 2z^4 + 16 e^{-\frac{\pi}{6}i}z^2 + 32 e^{-\frac{\pi}{3}i} &= 0 \\
    \Leftrightarrow z^4 + 8 e^{-\frac{\pi}{6}i}z^2 + 16 e^{-\frac{\pi}{3}i} &= 0 \\
    \Leftrightarrow (z^2 + 4 e^{-\frac{\pi}{6}i})^2 &= 0 \\
\end{align*}

daraus folgt dann die Gleichung

$$ z^2 = -4 e^{-\frac{\pi}{6}i} = 4 e^{\frac{5\pi}{6}i} $$

für \(z\) gilt dann mit \(s, \phi_k \in \R\)

\begin{align*}
    s &= \sqrt{4} = 2 \\
    \phi_k &= \tuple{\frac{5 \pi}{6} + 2k\pi} \frac{1}{2} \text{ mit } k = 0, 1 \\
    &= \tuple{\frac{5 \pi}{6} + \frac{12k\pi}{6}} \frac{1}{2}
    = \tuple{\frac{\pi (5 + 12k)}{6}} \frac{1}{2} \\
    &= \frac{\pi (5 + 12k)}{12}
\end{align*}

damit gilt

$$ z = 2 e^{\frac{\pi (5 + 12k)}{12}i} \text{ mit } k = 0, 1 $$

\textbf{\uline{NOTE:}} In diesem Fall sind das beides doppelte Nullstellen.

\subsection{b)}
\label{sec:orga070297}
Es gilt

\begin{align*}
    i e^{\frac{5\pi}{12}i} &= i e^{\frac{2\pi + 3\pi}{4 \cdot 3}i} \\
    &= i e^{\frac{\pi}{4}i + \frac{\pi}{6}i} \\
    &= i e^{\frac{\pi}{4}i} e^{\frac{\pi}{6}i} \\
    &= i \tuple{\cfrac{\cos}{\pi}{4} + i\cfrac{\sin}{\pi}{4}}
        \tuple{\cfrac{\cos}{\pi}{6} + i\cfrac{\sin}{\pi}{6}} \\
    &= i \tuple{\cfrac{\cos}{\pi}{4}\cfrac{\cos}{\pi}{6} + i \cfrac{\cos}{\pi}{4}\cfrac{\sin}{\pi}{6}
        +i \cfrac{\sin}{\pi}{4}\cfrac{\cos}{\pi}{6} - \cfrac{\sin}{\pi}{4}\cfrac{\sin}{\pi}{6}} \\
    &= i \tuple{\frac{\sqrt{3}}{2\sqrt{2}} + \frac{1}{2\sqrt{2}}i +
        \frac{\sqrt{3}}{2\sqrt{2}}i - \frac{1}{2\sqrt{2}}} \\
    &= \frac{-1 - \sqrt{3}}{2\sqrt{2}} + \frac{\sqrt{3} - 1}{2\sqrt{2}}i
\end{align*}

\section{Aufgabe 2}
\label{sec:orgc4ab504}
\subsection{a)}
\label{sec:orgd3df44d}
$$ p(i) = (i)^4 - 1 = 1 - 1 = 0 $$

\subsection{b)}
\label{sec:orgfac6d1e}
Polynomdivision:

\begin{align*}
    (z^4 - 1) / (z^2 + 1) &= \\
    (-z^2 - 1) / (z^2 + 1) &= z^2 \\
    (0) / (z^2 + 1) &= z^2 - 1
\end{align*}

demnach gilt

$$ \frac{p(z)}{q(z)} = z^2 - 1 $$

\subsection{c)}
\label{sec:org48a6b17}
\(p\) lässt sich im Komplexen wie folgt zerlegen

\begin{align*}
    p(z) = z^4 - 1 &= (z^2 + 1)(z^2 - 1) \\
    &= (z^2 + 1)(z + 1)(z - 1) \\
    &= (z + i)(z - i)(z + 1)(z - 1) \\
\end{align*}

\subsection{d)}
\label{sec:org0c498dc}
\(p\) lässt sich im Reellen wie folgt zerlegen

$$ p(z) = z^4 - 1 = (z^2 + 1)(z + 1)(z - 1) $$

\pagebreak

\section{Aufgabe 3}
\label{sec:orgb3171c6}
\subsection{a)}
\label{sec:orgfe2dec5}
Für den Nenner \(p\) gilt

\begin{align*}
    p(x) &= x^2 - 2x + 5 \\
    &= (x - (1 + 2i))(x - (1 - 2i))
\end{align*}

damit gilt für die Zerlegung

$$ \frac{x}{x^2 - 2x + 5} = \frac{A}{x - (1 + 2i)} + \frac{B}{x - (1 - 2i)} $$

\subsection{b)}
\label{sec:org5016542}
Es gilt

\begin{align*}
    \frac{i}{x - (1 + 2i)} + \frac{-i}{x - (1 - 2i)} &=
       \frac{i(x - 1 + \sqrt{2}i) - i(x - 1 - \sqrt{2}i)}{x^2 - 2x + 5} \\
    &= \frac{-2\sqrt{2}}{x^2 - 2x + 5}
\end{align*}

damit gilt für die reelle Zerlegung

$$ f(x) = \frac{-2\sqrt{2}}{x^2 - 2x + 5} $$

\section{Aufgabe 4}
\label{sec:orga5bb5d6}
\subsection{a)}
\label{sec:org13b630f}
Sei \(p\) das Zahlerpolynom und \(q\) das Nennerpolynom.

Für den \(p\) gilt

$$ p(x) = x^2 - 6x + 9 = (x - 3)^2 $$

Es muss eine Polynomdivision durchgeführt werden, da gilt \(\deg(p) > \deg(q)\).
Es folgt eine Polynomdivision:

\begin{align*}
    (x^3 - 6x^2 + 10x - 1) / (x^2 - 6x + 9) &= \\
    (x - 1) / (x^2 - 6x + 9) &= x 
\end{align*}

\pagebreak

damit gilt

$$ \frac{p(x)}{q(x)} = x + \frac{x - 1}{(x - 3)^2} $$

daraus ergibt sich der komplexe Ansatz der Zerlegung mit \(A, B \in \C\)

$$ \frac{x - 1}{(x - 3)^2} = \frac{A}{x - 3} + \frac{B}{(x - 3)^2} $$

damit gilt für \(B\) mit \(x = 3\)

$$ (3) - 1 = 2 = B $$

und für \(A\) gilt

$$ x - 1 = Ax - 3A + 2 \Leftrightarrow x - 1 = Ax - 3A + 2 $$

daraus ergeben sich

\begin{align*}
    1 &= A \\
    -1 &= -3A + 2 \\
    \\
    \Rightarrow A &= 1
\end{align*}

Da \(A\) und \(B\) Elemente der reellen und komplexen Zahlen sind, ist die folgende Zerlegung
sowohl reel als auch komplex

$$ \frac{p(x)}{q(x)} = x + \frac{1}{x - 3} + \frac{2}{(x - 3)^2} $$

\subsection{b)}
\label{sec:orge9b1c24}
Sei \(p\) das Zahlerpolynom und \(q\) das Nennerpolynom.

Da \(\deg(p) < \deg(q)\) gilt, muss keine Polynomdivision durchgeführt werden. \\
Für \(q\) gilt


\begin{align*}
    q(x) = x^3 + 2x = x(x^2 + 2) = x(x + \sqrt{2}i)(x - \sqrt{2}i)
\end{align*}

daraus ergibt sich dann folgender Ansatz mit \(A, B, C \in \C\)

$$ \frac{2x^2 - 2x - 1}{x(x + \sqrt{2}i)(x - \sqrt{2}i)} = \frac{A}{x} +
    \frac{B}{x + \sqrt{2}i} + \frac{C}{x - \sqrt{2}i} $$

\pagebreak

damit gilt für \(A\) mit \(x = 0\)

$$ \frac{-1}{2} = A $$

und es gilt für \(B\) mit \(x = -\sqrt{2}i\)

$$ \frac{-4 + 2\sqrt{2}i - 1}{(-\sqrt{2}i)(-2\sqrt{2}i)} = B = \frac{5 - 2\sqrt{2}i}{4} $$

und es gilt für \(C\) mit \(x = \sqrt{2}i\)

$$ \frac{-4 - 2\sqrt{2}i - 1}{(\sqrt{2}i)(2\sqrt{2}i)} = C = \frac{5 + 2\sqrt{2}i}{4} $$

damit gilt für die komplexe Zerlegung

$$ \frac{p(x)}{q(x)} = \frac{\frac{-1}{2}}{x} + \frac{\frac{5 - 2\sqrt{2}i}{4}}{x + \sqrt{2}i} +
    \frac{\frac{5 + 2\sqrt{2}i}{4}}{x - \sqrt{2}i} $$

es gilt

\begin{align*}
    \frac{\frac{5 - 2\sqrt{2}i}{4}}{x + \sqrt{2}i} + \frac{\frac{5 + 2\sqrt{2}i}{4}}{x - \sqrt{2}i} &=
        \frac{\frac{1}{4}((5 - 2\sqrt{2}i)(x - \sqrt{2}i) + (5 + 2\sqrt{2}i)(x + \sqrt{2}i))}{x^2 + 2} \\
    &= \frac{\frac{1}{4}(5x - 5\sqrt{2}i - 2\sqrt{2}xi - 4 + 5x + 5\sqrt{2}i + 2\sqrt{2}xi - 4)}{x^2 + 2} \\
    &= \frac{\frac{1}{2}(5x - 4)}{x^2 + 2}
\end{align*}

damit gilt für die reelle Zerlegung

$$ \frac{p(x)}{q(x)} = \frac{\frac{-1}{2}}{x} + \frac{\frac{1}{2}(5x - 4)}{x^2 + 2} $$
\end{document}
