% Created 2021-01-13 Mi 14:59
% Intended LaTeX compiler: pdflatex
\documentclass[a4paper, 11pt]{article}
\usepackage[utf8]{inputenc}
\usepackage[T1]{fontenc}
\usepackage{graphicx}
\usepackage{grffile}
\usepackage{longtable}
\usepackage{wrapfig}
\usepackage{rotating}
\usepackage[normalem]{ulem}
\usepackage{amsmath}
\usepackage{textcomp}
\usepackage{amssymb}
\usepackage{capt-of}
\usepackage{hyperref}
\usepackage{braket}
\usepackage[germanb]{babel}
\usepackage[dvipsnames]{xcolor}
\definecolor{BG}{RGB}{28, 20, 8}
\definecolor{FG}{RGB}{60, 140, 0}
\pagecolor{BG}
\color{FG}
\author{Daniel Geinets (453843), Christopher Neumann (409098), Dennis Schulze (458415)}
\date{\today}
\title{Analysis I und Lineare Algebra für Ingenieurwissenschaften \large  \\ Hausaufgabe 07 - Al-Maweri 13}
\hypersetup{
 pdfauthor={Daniel Geinets (453843), Christopher Neumann (409098), Dennis Schulze (458415)},
 pdftitle={Analysis I und Lineare Algebra für Ingenieurwissenschaften \large  \\ Hausaufgabe 07 - Al-Maweri 13},
 pdfkeywords={},
 pdfsubject={},
 pdfcreator={Emacs 27.1 (Org mode 9.5)}, 
 pdflang={Germanb}}
\begin{document}

\maketitle
\tableofcontents

\setcounter{secnumdepth}{0}
\newcommand{\tuple}[1]{\left(#1\right)}
\renewcommand{\cfrac}[3]{#1 \tuple{\frac{#2}{#3}}}
\newcommand{\R}{\mathbb{R}}
\newcommand{\Z}{\mathbb{Z}}
\newcommand{\Q}{\mathbb{Q}}
\newcommand{\N}{\mathbb{N}}
\newcommand{\C}{\mathbb{C}}

\makeatletter
\renewcommand*\env@matrix[1][*\c@MaxMatrixCols c]{%
\hskip -\arraycolsep
\let\@ifnextchar\new@ifnextchar
\array{#1}}
\makeatother

\pagebreak

\section{Aufgabe 1}
\label{sec:orga811001}

Für f mit \(z \in \Z\):
\newline

$$ f(x) =
    \begin{cases}
        z   & \text{, für } x = z \;\;(1) \\
        z   & \text{, für } z \leq x < z + 1 \;\;(2)
    \end{cases} $$
\newline
\newline
(1) Sprungstellen von \(z\) auf \(z + 1\)
    in ganz \(\Z\) und damit unstetig \newline
(2) konstant und damit stetig in [z, z + 1[
    für alle \(z \in \Z\) \newline
\(\Rightarrow\) damit ist f unstetig \newline
\newline
Für g:
\newline
\newline
\begin{math}
    g(x) = \frac{2}{2 + e^{-2x}}
    \text{, exp-Funktion positiv, also keine Polarstelle und stetig in ganz }\mathbb{R}
\end{math}
\newline
\newline
Für \(f \circ g\):
\newline
\newline
\begin{math}
    f(g(x)) = f\left(\frac{2}{2 + e^{-2x}}\right) =
        \lfloor \frac{2}{2 + e^{-2x}} \rfloor \text{, mit }
    e^{-2x} \mapsto \; ]0, \infty[ \\
    \Rightarrow \lfloor \frac{2}{2 + e^{-2x}} \rfloor = 0, \forall x \in \R \\
    \Rightarrow \text{ somit ist } f \circ g \text{ auf ganz } \R
    \text{ konstant und damit stetig}
\end{math}

\section{Aufgabe 2}
\label{sec:orgd486413}
Sei \(a \in \R\) und sei da eine beliebige Folge \(x_n\) mit
\(\lim_{n \rightarrow \infty} x_n = a\), dann gilt

$$ \lim_{n \rightarrow \infty} \sin(2\pi x_n) + 3 =
    \lim_{x \rightarrow a} \sin(2\pi x) + 3 = \sin(2\pi a) + 3 $$

daraus folgt, \(\sin(2\pi x) + 3\) ist stetig. \newline
Ebenso gilt

$$ \lim_{n \rightarrow \infty} (x_n)^3 - 1 =
    \lim_{x \rightarrow a} x^3 - 1 = a^3 - 1 $$

daraus folgt, \(x^3 - 1\) ist stetig. \newline
Desweiteren gilt

\begin{align*}
    \lim_{x \rightarrow 1} \sin(2\pi x) + 3 &= \sin(2\pi \cdot 1) + 3 =
        \sin(2\pi) + 3 = 3 \\
    \lim_{x \rightarrow 1} (x^3 - 1) &= (1)^3 - 1 = 0 \neq 3
\end{align*}

daraus folgt, \(f\) ist nicht stetig in \(x = 1\).

\section{Aufgabe 3}
\label{sec:org0e61ea9}
\subsection{a)}
\label{sec:org768c34b}
\subsubsection{(a)}
\label{sec:org8bb911b}
Es gilt

\begin{align*}
    \lim_{x \rightarrow x_0} \frac{\sqrt{2x} - \sqrt{2x_0}}{x - x_0} &=
        \lim_{x \rightarrow x_0}
            \frac{2(x - x_0)}{(x - x_0)(\sqrt{2x} + \sqrt{2x_0})} \\
    &= \lim_{x \rightarrow x_0} \frac{2}{\sqrt{2x} + \sqrt{2x_0}} \\
    & \overset{\text{GWS}}{=}
        \frac{2}{2 \sqrt{2x_0}} = \frac{1}{\sqrt{2x_0}} \\
    \Rightarrow f'(x) = \frac{1}{\sqrt{2x}}
\end{align*}

\subsubsection{(b)}
\label{sec:org0c5d25a}
Es gilt

$$ f(x) = (x-1)|x-1| \Leftrightarrow f(x) =
    \begin{cases}
        (x-1)^2 &, x \geq 1 \\
        -(x-1)^2 &, x < 1
    \end{cases} $$

dann gilt

\begin{align*}
    \lim_{x \rightarrow x_0} \frac{(x - 1)^2 - (x_0 - 1)^2}{x - x_0} &=
        \lim_{x \rightarrow x_0} \frac{(x + x_0 - 2)(x - x_0)}{x - x_0} \\
    &= \lim_{x \rightarrow x_0} (x + x_0 - 2)
        \overset{\text{GWS}}{=} 2x_0 - 2
\end{align*}

desweiteren gilt

\begin{align*}
    \lim_{x \rightarrow x_0} \frac{-(x - 1)^2 + (x_0 - 1)^2}{x - x_0} &=
        \lim_{x \rightarrow x_0} \frac{(x_0 + x - 2)(x_0 - x)}{x - x_0} \\
    &= \lim_{x \rightarrow x_0} \frac{-(x_0 + x - 2)(x - x_0)}{x - x_0} \\
    &= \lim_{x \rightarrow x_0} -(x + x_0 - 2)
        \overset{\text{GWS}}{=} -2x_0 + 2
\end{align*}

daraus folgt

$$ f'(x) =
    \begin{cases}
        2(x-1) &, x \geq 1 \\
        -2(x-1) &, x < 1
    \end{cases} $$

\subsection{b)}
\label{sec:org5dd23f9}
\subsubsection{(a)}
\label{sec:orgb250665}
Es gilt

$$ (5x^6 - \cos(6x))' = 30x^5 + 6\sin(6x) $$

\subsubsection{(b)}
\label{sec:org548a63a}
Es gilt

\begin{align*}
    \tuple{\frac{\sin(2x^2 - 2\pi)}{\cos(5\pi - 2x^2)}}' &=
        \tuple{\frac{\sin(2x^2)}{\cos(\pi - 2x^2)}}' =
        \tuple{\frac{\sin(2x^2)}{-\cos(-2x^2)}}' =
        \tuple{\frac{\sin(2x^2)}{-\cos(2x^2)}}' \\
    &= \tuple{\sin(2x^2)\tuple{{-\cos(2x^2)}}^{-1}}' \\
    &= -\cos(2x^2) 4x (\cos(2x^2))^{-1} -
            \sin(2x^2) \frac{1}{\cos^2(2x^2)} \sin(2x^2) 4x \\
    &= -4x - 4x \frac{\sin^2(2x^2)}{\cos^2(2x^2)} = -4x(\tan^2(2x^2) + 1)
\end{align*}

\subsubsection{(c)}
\label{sec:org1eb343f}
Es gilt

$$ \tuple{e^{(4x - 2)(3x + 3)}}' = \tuple{e^{12x^2 + 6x - 6}}' =
    e^{12x^2 + 6x - 6}(24x + 6) $$

\section{Aufgabe 4}
\label{sec:org3ad8b5f}
\subsection{a)}
\label{sec:orgb69403c}
\begin{math}
    \lim \limits_{x \nearrow 1} f(x) =
        \lim \limits_{x \nearrow 1} (ax + b - 3)
        \overset{\text{GWS}}{=} a + b - 3
    \newline
    \newline
    \lim \limits_{x \searrow 1} f(x) =
        \lim \limits_{x \searrow 1} (3x^2) \overset{\text{GWS}}{=} 3
    \newline
    \newline
    \text{Also: }\lim \limits_{x \nearrow 1} f(x) =
        a + b - 3 = 3 = \lim \limits_{x \searrow 1} f(x)
    \newline
    \text{$\Rightarrow$ $f$ ist stetig in ganz } \R
        \text{, wenn } a + b = 6 \text{ gilt.}
\end{math}

\pagebreak

\subsection{b)}
\label{sec:org8e305b1}
\begin{math}
    \text{obere Teilfunktion: }
    \newline
    \lim \limits_{x \to x_0} f(x) =
        \lim \limits_{x \to x_0} \frac{(ax + b - 3) -
            (ax_0 + b - 3)}{x - x_0} = \frac{a(x - x_0)}{x - x_0} = a
    \newline
    \newline
    \text{untere Teilfunktion: }
    \newline
    f'(x) = 6x
    \newline
    \text{für: }f'(1) = 6 = a = \lim \limits_{x \to x_0} f(x)
    \newline
    \newline
    \text{Somit ist f differenzierbar für: } a = 6 \text{ und alle } b \in \mathbb{R}
\end{math}

\subsection{c)}
\label{sec:orgaf5a11b}
\begin{math}
    \text{obere Teilfunktion: }
    \newline
    \lim \limits_{x \nearrow 1} f(x) =
        \lim \limits_{x \nearrow 1} (6x + 3) \overset{\text{GWS}}{=} 6 + 3 = 9
    \newline
    \newline
    \text{Also: }\lim \limits_{x \searrow 1} f(x) =
        3 \neq 9 \lim \limits_{x \nearrow 1} f(x)
    \newline
    \newline
    \text{Somit ist f in } x = 1
    \text{ nicht stetig und damit auch nicht differenzierbar}
\end{math}
\end{document}
