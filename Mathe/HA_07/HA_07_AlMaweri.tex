% Created 2021-01-13 Mi 10:58
% Intended LaTeX compiler: pdflatex
\documentclass[a4paper, 11pt]{article}
\usepackage[utf8]{inputenc}
\usepackage[T1]{fontenc}
\usepackage{graphicx}
\usepackage{grffile}
\usepackage{longtable}
\usepackage{wrapfig}
\usepackage{rotating}
\usepackage[normalem]{ulem}
\usepackage{amsmath}
\usepackage{textcomp}
\usepackage{amssymb}
\usepackage{capt-of}
\usepackage{hyperref}
\usepackage{braket}
\usepackage[germanb]{babel}
\usepackage[dvipsnames]{xcolor}
\definecolor{BG}{RGB}{28, 20, 8}
\definecolor{FG}{RGB}{60, 140, 0}
\pagecolor{BG}
\color{FG}
\author{Daniel Geinets (453843), Christopher Neumann (409098), Dennis Schulze (458415)}
\date{\today}
\title{Analysis I und Lineare Algebra für Ingenieurwissenschaften \large  \\ Hausaufgabe 07 - Al-Maweri 13}
\hypersetup{
 pdfauthor={Daniel Geinets (453843), Christopher Neumann (409098), Dennis Schulze (458415)},
 pdftitle={Analysis I und Lineare Algebra für Ingenieurwissenschaften \large  \\ Hausaufgabe 07 - Al-Maweri 13},
 pdfkeywords={},
 pdfsubject={},
 pdfcreator={Emacs 27.1 (Org mode 9.5)}, 
 pdflang={Germanb}}
\begin{document}

\maketitle
\tableofcontents

\setcounter{secnumdepth}{0}
\newcommand{\tuple}[1]{\left(#1\right)}
\renewcommand{\cfrac}[3]{#1 \tuple{\frac{#2}{#3}}}
\newcommand{\R}{\mathbb{R}}
\newcommand{\Z}{\mathbb{Z}}
\newcommand{\Q}{\mathbb{Q}}
\newcommand{\N}{\mathbb{N}}
\newcommand{\C}{\mathbb{C}}

\makeatletter
\renewcommand*\env@matrix[1][*\c@MaxMatrixCols c]{%
\hskip -\arraycolsep
\let\@ifnextchar\new@ifnextchar
\array{#1}}
\makeatother

\pagebreak

\section{Aufgabe 1}
\label{sec:orgf6a5124}

\section{Aufgabe 2}
\label{sec:org665e8b4}
Sei \(a \in \R\) und sei da eine beliebige Folge \(x_n\) mit
\(\lim_{n \rightarrow \infty} x_n = a\), dann gilt

$$ \lim_{n \rightarrow \infty} \sin(2\pi x_n) + 3 =
    \lim_{x \rightarrow a} \sin(2\pi x) + 3 = \sin(2\pi a) + 3 $$

daraus folgt, \(\sin(2\pi x) + 3\) ist stetig. \newline
Ebenso gilt

$$ \lim_{n \rightarrow \infty} (x_n)^3 - 1 =
    \lim_{x \rightarrow a} x^3 - 1 = a^3 - 1 $$

daraus folgt, \(x^3 - 1\) ist stetig. \newline
Desweiteren gilt

\begin{align*}
    \lim_{x \rightarrow 1} \sin(2\pi x) + 3 &= \sin(2\pi \cdot 1) + 3 =
        \sin(2\pi) + 3 = 3 \\
    \lim_{x \rightarrow 1} (x^3 - 1) &= (1)^3 - 1 = 0 \neq 3
\end{align*}

daraus folgt, \(f\) ist nicht stetig in \(x = 1\).

\section{Aufgabe 3}
\label{sec:org9cd7a94}
\subsection{a)}
\label{sec:org8365333}
\subsubsection{(a)}
\label{sec:org629d593}
\subsubsection{(b)}
\label{sec:org0efba22}
\subsection{b)}
\label{sec:org99b6063}
\subsubsection{(a)}
\label{sec:org4a6de2e}
\subsubsection{(b)}
\label{sec:org8f877d8}
\subsubsection{(c)}
\label{sec:org2284fe0}
\section{Aufgabe 4}
\label{sec:org3887b6e}
\end{document}
