% Created 2021-01-20 Mi 14:18
% Intended LaTeX compiler: pdflatex
\documentclass[a4paper, 11pt]{article}
\usepackage[utf8]{inputenc}
\usepackage[T1]{fontenc}
\usepackage{graphicx}
\usepackage{grffile}
\usepackage{longtable}
\usepackage{wrapfig}
\usepackage{rotating}
\usepackage[normalem]{ulem}
\usepackage{amsmath}
\usepackage{textcomp}
\usepackage{amssymb}
\usepackage{capt-of}
\usepackage{hyperref}
\usepackage{braket}
\usepackage[germanb]{babel}
\usepackage[dvipsnames]{xcolor}
\definecolor{BG}{RGB}{28, 20, 8}
\definecolor{FG}{RGB}{60, 140, 0}
\pagecolor{BG}
\color{FG}
\author{Daniel Geinets (453843), Christopher Neumann (409098), Dennis Schulze (458415)}
\date{\today}
\title{Analysis I und Lineare Algebra für Ingenieurwissenschaften \large  \\ Hausaufgabe 08 - Al-Maweri 13}
\hypersetup{
 pdfauthor={Daniel Geinets (453843), Christopher Neumann (409098), Dennis Schulze (458415)},
 pdftitle={Analysis I und Lineare Algebra für Ingenieurwissenschaften \large  \\ Hausaufgabe 08 - Al-Maweri 13},
 pdfkeywords={},
 pdfsubject={},
 pdfcreator={Emacs 27.1 (Org mode 9.5)}, 
 pdflang={Germanb}}
\begin{document}

\maketitle
\tableofcontents

\setcounter{secnumdepth}{0}
\newcommand{\tuple}[1]{\left(#1\right)}
\newcommand{\R}{\mathbb{R}}
\newcommand{\Z}{\mathbb{Z}}
\newcommand{\Q}{\mathbb{Q}}
\newcommand{\N}{\mathbb{N}}
\newcommand{\C}{\mathbb{C}}

\makeatletter
\renewcommand*\env@matrix[1][*\c@MaxMatrixCols c]{%
\hskip -\arraycolsep
\let\@ifnextchar\new@ifnextchar
\array{#1}}
\makeatother

\pagebreak

\section{Aufgabe 1}
\label{sec:orga3882f8}
\begin{math}
    \text{Zeige, dass  }|\tan(x)+\tan(y)| \geq |x+y|
        \text{  für alle  }x,y \in \left]-\frac{\pi}{2}, \frac{\pi}{2} \right[
    \newline
    \newline
\end{math}

\begin{align*}
    \frac{|\tan(x)+\tan(y)|}{|x+y|} &= f'(\varepsilon) = 1+\frac{1}{\cos(\varepsilon)^2} \\
    \Leftrightarrow   \frac{|\tan(x)+\tan(y)|}{|x+y|} &= 1+\frac{1}{\cos(\varepsilon)^2} \\
    \Leftrightarrow |\tan(x)+\tan(y)| &= (|x+y|)\left(1+\frac{1}{\cos(\varepsilon)^2}\right)\\
    \Leftrightarrow |\tan(x)+\tan(y)| &= |x+y|+\frac{|x+y|}{\cos(\varepsilon)^2}\\
\end{align*}

\begin{math}
    \newline
    \cos(\varepsilon)^2 \text{  auf  } \left]-\frac{\pi}{2}, \frac{\pi}{2} \right[
        \geq 0 \text{  , also}
    \newline
\end{math}

\begin{align*}
    \frac{|x+y|}{\cos(\varepsilon)^2} &\geq 0 \\
    \Leftrightarrow|x+y|+\frac{|x+y|}{\cos(\varepsilon)^2} &\geq |x+y| \\
    \Leftrightarrow|\tan(x)+\tan(y)| &\geq |x+y|
\end{align*}

\section{Aufgabe 2}
\label{sec:org033ce5e}
Es gilt

\begin{align*}
    |f(x) - f(y)| &\leq |x - y|^2 \\
    \Leftrightarrow \frac{|f(x) - f(y)|}{|x - y|} &\leq |x - y|
\end{align*}

Die Funktion \(f\) muss zumindest einmal differenzierbar sein, dann gilt mit \(\xi \in ]y, x[\)

\begin{equation*}
    |f'(\xi)| \leq |x - y|
\end{equation*}

Behauptung: \(f'(x) = 0, \forall x \in \R\)

Dann gilt

\begin{equation*}
    |f'(\xi)| = |0| = 0 \leq |x - y|
\end{equation*}

Daraus folgt

\begin{align*}
    |f(x) - f(y)| &= |f(x) - f(x)| \\
    &= |0| = 0 \\
    &\leq |x - y|^2
\end{align*}

\section{Aufgabe 3}
\label{sec:org78f30ce}
\begin{math}
    \text{Berechne für  }f(x) = \sqrt{2+x}, T_2(x)\text{ und }
        T_4(x) \text{ mit  }x_0 = 0
    \newline
\end{math}

\begin{align*}
    f(x) &= \sqrt{2+x} \\
    f'(x) &= \frac{1}{2\sqrt{2+x}} \\
    f''(x) &= -\frac{1}{4\sqrt{2+x}^3} \\
    f'''(x) &= \frac{3}{8\sqrt{2+x}^5} \\
    f^4(x) &= \frac{15}{16\sqrt{2+x}^7} \\
    \Rightarrow f^{(n)}(x) &= (-1)^n \cdot \frac{(2n-2)!}{2^{2n-1}(n-1)!} \cdot (x+2)^{\frac{1}{2}-n}
\end{align*}

\begin{math}
    \newline
\end{math}

\begin{align*}
    T_2(x) &= \sqrt{2} + \frac{\frac{1}{2\sqrt{2}}}{1}x - \frac{\frac{1}{8\sqrt{2}^3}}{2!}x^2 \\
    &= \sqrt{2} + \frac{1}{2\sqrt{2}}x - \frac{1}{16\sqrt{2}}x^2
\end{align*}

\begin{align*}
    T_4(x) &= T_2(x) + \frac{\frac{3}{8\sqrt{2}^5}}{3!}x^3 - \frac{\frac{15}{16\sqrt{2}^7}}{4!}x^4 \\
    &= \sqrt{2} + \frac{1}{2\sqrt{2}}x -\frac{1}{8\sqrt{2}}x^2 + \frac{1}{64\sqrt{2}}x^3 - \frac{5}{1024\sqrt{2}}x^4
\end{align*}
\newline
\begin{math}
    \text{Abschätzen des  }R_2(x) \text{ ,für } |x| < \frac{1}{2}
    \newline
    |R_2(x)| = \left|\frac{ \frac{3}{8 \sqrt{\varepsilon+2}^5}}{3!}x^3 \right|
        = \left| \frac{1}{16\sqrt{\varepsilon+2}^5} \right| \cdot |x|^3
    \newline
    \text{Es gilt mit }\varepsilon+2 > \frac{1}{2} \Leftrightarrow \frac{1}{\varepsilon+2} < 2
    \newline
    \Rightarrow \left| \frac{1}{16\sqrt{\varepsilon+2}^5} \right| \cdot |x|^3
        < \frac{1}{16} \cdot \sqrt{2}^5 \cdot \frac{1}{2^3}
        = \frac{\sqrt{2}}{32} \approx 0.0442
\end{math}

\section{Aufgabe 4}
\label{sec:org8bdb58f}
Es gilt

\begin{align*}
    f(x) &= e^x \cos(x) \\
    f'(x) &= e^x \cos(x) - e^x \sin(x) \\
        &= e^x (\cos(x) - \sin(x)) \\
    f''(x) &= e^x (\cos(x) - \sin(x)) + e^x (-\sin(x) - \cos(x)) \\
        &= -2 e^x \sin(x) \\
    f'''(x) &= -2 e^x \sin(x) -2 e^x \cos(x) \\
        &= -2 e^x (\cos(x) + \sin(x)) \\
    f''''(x) &= -2 e^x (\cos(x) + \sin(x)) - 2 e^x (-\sin(x) + \cos(x)) \\
        &= -4 e^x \cos(x)
\end{align*}

desweiteren gilt

\begin{align*}
    f(0)     &= 1 \\
    f'(0)    &= 1 \\
    f''(0)   &= 0 \\
    f'''(0)  &= -2
\end{align*}

dann gilt für \(T_3\)

\begin{align*}
    T_3(x) &= f(x_0) + \frac{f'(x_0)}{1!} x +
        \frac{f''(x_0)}{2!} x^2 + \frac{f'''(x_0)}{3!} x^3 \\
    &= 1 + x + 0 \cdot x^2 + (-\frac{1}{3}) x^3 \\
    &= 1 + x - \frac{1}{3} x^3
\end{align*}

dann gilt für Restglied \(R_3\)

$$ R_3(x) = \frac{f''''(\xi)}{4!} x^4 = \frac{-4 e^{\xi} \cos(\xi)}{4!} x^4
    = \frac{- e^{\xi} \cos(\xi)}{3!} x^4 $$

dann gilt mit \(|x| = 10^{-2}\)

\begin{align*}
    |R_3(x)| &\leq 10^{-8} \\
    \Leftrightarrow \left| \frac{- e^{\xi} \cos(\xi)}{3!} x^4 \right| &\leq 10^{-8} \\
    \Leftrightarrow \left| \frac{- e^{\xi} \cos(\xi)}{3!} \right| |x^4| &\leq 10^{-8} \\
    \Leftrightarrow \left| \frac{- e^{\xi} \cos(\xi)}{3!} \right| |x|^4 &\leq 10^{-8} \\
    \Leftrightarrow \left| \frac{- e^{\xi} \cos(\xi)}{3!} \right| (10^{-2})^4 &\leq 10^{-8} \\
    \Leftrightarrow \left| \frac{- e^{\xi} \cos(\xi)}{3!} \right| &\leq 1 \\
    \Leftrightarrow \frac{e^{\xi} \cos(\xi)}{3!} &\leq 1 \\
    \Leftrightarrow e^{\xi} \cos(\xi) &\leq 6 \\
\end{align*}

das gilt für alle \(\xi \in ]-\frac{1}{100}, \frac{1}{100}[ \setminus \{0\}\). Daraus folgt

$$ \forall x \in [-\frac{1}{100}, \frac{1}{100}], |R_3(x)| \leq 10^{-8} $$
\end{document}
