% Created 2020-09-21 Mo 23:58
% Intended LaTeX compiler: pdflatex
\documentclass[a4paper,11pt]{article}
\usepackage[latin1]{inputenc}
\usepackage[T1]{fontenc}
\usepackage{graphicx}
\usepackage{grffile}
\usepackage{longtable}
\usepackage{wrapfig}
\usepackage{rotating}
\usepackage[normalem]{ulem}
\usepackage{amsmath}
\usepackage{textcomp}
\usepackage{amssymb}
\usepackage{capt-of}
\usepackage{hyperref}
\author{Moaz Haque, Felix Oechelhaeuser, Leo Pirker, Dennis Schulze}
\date{\today}
\title{Analysis I und Lineare Algebra f�r Ingenieurwissenschaften \large  \\ Hausaufgabe 08 - Geuter 29}
\hypersetup{
 pdfauthor={Moaz Haque, Felix Oechelhaeuser, Leo Pirker, Dennis Schulze},
 pdftitle={Analysis I und Lineare Algebra f�r Ingenieurwissenschaften \large  \\ Hausaufgabe 08 - Geuter 29},
 pdfkeywords={},
 pdfsubject={},
 pdfcreator={Emacs 27.1 (Org mode 9.4)}, 
 pdflang={English}}
\begin{document}

\maketitle
\tableofcontents

\pagebreak

\section{Aufgabe 1}
\label{sec:orgee68005}
\subsection{a)}
\label{sec:orgd5355ca}
Da der Grenzwert des Terms bei direktem Einsetzten nicht definiert ist ("\(\frac{0}{0}\)"),
muss die Regel von de l'Hospital angewand werden. Also gilt

\begin{align*}
  \lim_{x \rightarrow 0} \frac{e^x - 1}{\sin(2x)} &= \lim_{x \rightarrow 0} \frac{e^x}{2\cos(2x)} \\
  &= \frac{\lim_{x \rightarrow 0}  e^x}{\lim_{x \rightarrow 0}  2\cos(2x)} \\
  &= \frac{1}{2}
\end{align*}

Damit existiert dessen Grenzwert.

\subsection{b)}
\label{sec:org0aeee09}
Siehe a) \\
Es gilt

\begin{align*}
  \lim_{x \rightarrow 0} \frac{\cos(x) - 1}{x^2} &= \lim_{x \rightarrow 0} \frac{-\sin(x)}{2x)} \\
  &= \lim_{x \rightarrow 0} \frac{-\cos(x)}{2} \\
  &= \frac{\lim_{x \rightarrow 0}  -\cos(x)}{\lim_{x \rightarrow 0} 2} \\
  &= \frac{-1}{2}
\end{align*}

Damit existiert dessen Grenzwert nach zweifacher Anwendung der Regel.

\section{Aufgabe 2}
\label{sec:org1c7ac84}
\subsection{a) und c)}
\label{sec:org5503949}
\begin{align*}
    x_1 & = \underline{1},2 \\
    x_2 & = \underline{1,1}43575834 \\
    x_3 & = \underline{1,134}909462 \\
    x_4 & = \underline{1,134724}221 \\
    x_5 & = \underline{1,134724138} \\
    x_6 & = \underline{1,134724138} \\
\end{align*}

\subsection{b)}
\label{sec:org0910fc9}
\begin{align*}
    \intertext{aus HA07-3: $a_6 = \frac{9}{8}$ $b_6 = \frac{73}{64}$}
    \\
    y_6 & = \frac{a_6 + b_6}{2} = \frac{\frac{9}{8}+\frac{73}{64}}{2} = \frac{145}{128} = 1,1328125
    \\
    x_6 & = 1,134724138 > 1,1328125 = y_6
    \\
    \intertext{Das Bisektionsverfahren liefert mit 1,132 einen kleineren Funktionswert,
    als das Newton-Verfahren mit 1,134}
\end{align*}

\section{Aufgabe 3}
\label{sec:orgca42545}
\begin{align*}
    \intertext{Mittelwertsatz: $\frac{f(x) - f(y)}{x - y} = f'(\xi)$ wenn $y < x$}
    f'(\xi) = \tan'(\xi)
    & = \frac{\sin'(\xi)}{\cos'(\xi)}
    = \frac{\cos^2(\xi)+\sin^2(\xi)}{cos^2(\xi)}
    = \frac{1}{cos^2(\xi)}
    \intertext{Da sich der Nenner des Bruchs zwischen $\left]0,1\right]$ bewegt, gilt: $\frac{1}{cos^2(\xi)} \geq 1$}
    \frac{\tan(x) - \tan(0)}{x - 0} 
    = \frac{\tan(x)}{x} 
    & = \frac{1}{cos^2(\xi)} \geq 1
    \\
    | \tan(x) | & \geq |x|
    \intertext{Analog dazu muss also auch $| \tan(y) | \geq |y|$ sein,
    wodurch auch $| \tan(x) + \tan(y) | \geq | x + y |$ gelten muss.}
\end{align*}

\section{Aufgabe 4}
\label{sec:org1c0cad3}
\begin{align*}
    f(x) & = \frac{x-2}{(3-x)^2} \\
    f'(x) & = \frac{(3-x)^2 - (x-2)(2x-6)}{((3-x)^2)^2}
    = \frac{(9+x^2-6x)-(2x^2-6x-4x+12)}{(3-x)^4}
    \\
    & = \frac{-x^2+4x-3}{(3-x)^4} = \frac{(3-x)(x-1)}{(3-x)^4} = \frac{x-1}{(3-x)^2}
    \intertext{Kanidaten: $x_1 = 0$ (Randpunkt), $x_2 = 1$ (Nullstelle von $f'(x)$),
    $x_3 = 3$ (Randpunkt)}
    f(0) & = \frac{0-2}{(3-0)^2} = -\frac{2}{9} \\
    f(1) & = \frac{1-2}{(3-1)^2} = -\frac{1}{4} \\
    \intertext{$f(3)$ ist aus dem Definitionsbereich ausgeschlossen (und nicht definiert), also mit dem Grenzwert ann�hern:}
    f(3) & = \lim_{x \rightarrow 3} \frac{x-2}{(3-x)^2} = \frac{1}{0} = \infty
    \intertext{Also befindet sich das globale Maximum bei $x_3 = 3$,
    bei $x_1 = 0$ ein lokales Maximum und bei $x_2 = 1$ ein globales Minimum.}
\end{align*}

\section{Aufgabe 5}
\label{sec:org19ef50d}
\subsection{a)}
\label{sec:org855c274}
Es gilt f�r die Ableitung von \(g\)

\begin{align*}
  g'(t) &= \left( (s'(t))^2 + (\omega s(t))^2 \right)' \\
   &= ((s'(t))^2)' + ((\omega s(t))^2)' \\
   &= 2 s'(t) s''(t) + 2 (\omega s(t)) (\omega s'(t)) \\
   &= 2 s'(t) s''(t) + 2 \omega^2 s(t) s'(t) \\
   &= 2 s'(t) (s''(t) + \omega^2 s(t))
\end{align*}

Da \(s''(t) = -\omega^2 s(t)\) gilt, folgt daraus

\begin{align*}
  g'(t) &= 2 s'(t) ((-\omega^2 s(t)) + \omega^2 s(t)) \\
   &= 2 s'(t) \cdot 0 = 0
\end{align*}

Da \(g'(t) \geq 0\) und \(g'(t) \leq 0\) mit \(t \in \mathbb{R}\) gelten, folgt daraus,
dass \(g(t) = c\), mit \(c \in \mathbb{R}\) und f�r alle \(t \in \mathbb{R}\). \\
Daraus folgt, dass \((s'(t))^2 + (\omega s(t))^2\) ebenfalls konstant sein muss.
Daraus folgt wiederum, dass \((s'(t))^2\) und \((\omega s(t))^2\) konstant sind.
Daraus folgt dann, dass \(\omega s(t)\) konstant ist und damit gilt \(s'(t) = 0\) f�r alle \(t \in \mathbb{R}\),
woraus wiederum aus der urspr�ngilchen Annahme folgt \(s(t) = 0\) f�r alle \(t \in \mathbb{R}\).
Damit gilt dann auch \(g(t) = 0\), f�r alle \(t \in \mathbb{R}\).

\subsection{b)}
\label{sec:org8e9e712}
Es gilt f�r die erste und zweite Ableitung von \(h\)

\begin{align*}
  h'(t) &= \left( s(t) - s(0)\cos(\omega t) - \frac{s'(0)}{\omega}\sin(\omega t) \right)' \\
   &= s'(t) + s(0)\omega \sin(\omega t) - s'(0) \cos(\omega t) \\
  \\
  h''(t) &= \left( s'(t) + s(0)\omega \sin(\omega t) - s'(0) \cos(\omega t) \right)' \\
   &= s''(t) + s(0)\omega^2 \cos(\omega t) + s'(0)\omega \sin(\omega t) \\
   &= -\omega^2 s(t) + s(0)\omega^2 \cos(\omega t) + s'(0)\omega \sin(\omega t) \\
   &= -\omega^2 \left( s(t) - s(0)\cos(\omega t) - \frac{s'(0)}{\omega}\sin(\omega t) \right) \\
   &= -\omega^2 h(t)
\end{align*}

Was in 5a) gezeigt wurde, kann analog f�r \(h\) gezeigt werden. Damit gilt

\begin{align*}
  h(t) = 0 &= s(t) - s(0)\cos(\omega t) - \frac{s'(0)}{\omega}\sin(\omega t) \\
\Leftrightarrow s(t) &= s(0)\cos(\omega t) + \frac{s'(0)}{\omega}\sin(\omega t)
\end{align*}

was \textbf{eine} L�sung der urspr�nglichen Differenzialgleichung ist.

\pagebreak

\section{Aufgabe 6}
\label{sec:org56832ec}
Es gelten f�r die n�tigen Ableitungen von \(f\)

\begin{align*}
  f'(x) &= \left( e^x \cos(x) \right)' \\
   &= e^x \cos(x) + e^x (-\sin(x)) \\
   &= e^x (\cos(x) - \sin(x)) \\
  \\
  f''(x) &= \left( e^x (\cos(x) - \sin(x)) \right)' \\
   &= e^x (\cos(x) - \sin(x)) + e^x (-\sin(x) - \cos(x)) \\
   &= -2 e^x \sin(x) \\
  \\
  f'''(x) &= \left( -2 e^x \sin(x) \right)' \\
   &= -2 ( e^x \sin(x) + e^x \cos(x) ) \\
   &= -2 e^x ( \sin(x) + \cos(x) ) \\
  \\
  f''''(x) &= \left( -2 e^x ( \sin(x) + \cos(x) ) \right)' \\
   &= -2 ( e^x ( \sin(x) + \cos(x) ) + e^x ( \cos(x) - \sin(x) ) ) \\
   &= -4 e^x \cos(x)
\end{align*}

und die n�tigen Funktionswerte bei \(x_0 = 0\) f�r \(T_3\)

\begin{align*}
  f(0) &= e^0 \cos(0) = 1\\
  f'(0) &= e^0 (\cos(0) - \sin(0)) = 1 \\
  f''(0) &= -2 e^0 \sin(0) = 0 \\
  f'''(0) &= -2 e^0 ( \sin(0) + \cos(0) ) = -2
\end{align*}

Dann gilt f�r die Aproximation von \(f\) bei \(x_0 = 0\) mit \(T_3\)

\begin{align*}
  T_3(x) &= \sum_{k = 0}^{3} \frac{f^{(k)}(0)}{k!}x^k \\
   &= \frac{f(0)}{0!}x^0 + \frac{f'(0)}{1!}x^1 + \frac{f''(0)}{2!}x^2 + \frac{f'''(0)}{3!}x^3 \\
   &= \frac{1}{0!} + \frac{1}{1!}x + \frac{0}{2!}x^2 + \frac{-2}{3!}x^3 \\
   &= 1 + x - \frac{1}{3}x^3
\end{align*}

Da \(f\) n+1-mal ableitbar ist, gilt f�r das Restglied mit Absch�tzung

\begin{align*}
  R_3(x) &= \frac{f^{(4)}(\xi)}{4!}x^4 = \frac{-4 e^{\xi} \cos(\xi)}{24}x^4 \\
   &\geq \frac{-4 x^4}{24}
\end{align*}

Es soll gelten

$$ |f(x) - T_3(x)| = |R_3(x)| \leq 10^{-8} $$

mit der Absch�tzung gilt dann

\begin{align*}
  \left| \frac{-4 x^4}{24} \right| &\leq 10^{-8}, \forall x \in \left[ -\frac{1}{100}, \frac{1}{100} \right] \\
\Leftrightarrow \frac{4 x^4}{24} &\leq 10^{-8} \\
\Leftrightarrow x^4 &\leq 6 \cdot 10^{-8} \\
\Leftrightarrow x &\leq \sqrt[4]{6} \cdot 10^{-2}
\end{align*}

\section{Aufgabe 7}
\label{sec:org066f831}
Es gelten f�r die n�tigen Ableitungen von \(f\)

\begin{align*}
  f'(x) &= (\ln(1 + x))' = \frac{1}{1 + x} \\
  f''(x) &= (\frac{1}{1 + x})' = \frac{-1}{(1 + x)^2} \\
  f'''(x) &= (\frac{-1}{(1 + x)^2})' = \frac{2}{(1 + x)^3}
\end{align*}

und die n�tigen Funktionswerte bei \(x = 0\)

\begin{align*}
  f(0) &= \ln(1 + 0) = 0 \\
  f'(0) &= \frac{1}{1 + 0} = 1 \\
  f''(0) &= \frac{-1}{(1 + 0)^2} = -1 \\
  f'''(0) &= \frac{2}{(1 + 0)^3} = 2
\end{align*}

\pagebreak

F�r die Approximation \(T_1\) gilt dann

$$ T_1(x) = \frac{f(0)}{0!}x^0 + \frac{f'(0)}{1!}x^1 = 0 + x = x $$

mit dessen Restglied \(R_1\)

$$ f(x) - T_1(x) = R_1(x) = \frac{f''(\xi)}{2!}x^2 = \frac{-x^2}{2(1 + \xi)^2} $$

und f�r die Approximation \(T_2\) gilt dann

$$ T_2(x) = \frac{f(0)}{0!}x^0 + \frac{f'(0)}{1!}x^1 + \frac{f''(0)}{2!}x^2 = 0 + x - \frac{x^2}{2} = x - \frac{x^2}{2} $$

mit dessen Restglied \(R_2\)

$$ f(x) - T_2(x) = R_2(x) = \frac{f'''(\xi)}{3!}x^3 = \frac{x^3}{3(1 + \xi)^3} $$


F�r das Restglied \(R_1\) gilt

$$ R_1(x) > 0, \forall x \in ]0, \infty[ \text{ mit } \xi = 0 $$

Daraus folgt \(T_1(x) > f(x)\) f�r alle \(x\) \(\in\) ]0, \(\infty\)[.

und es gilt f�r das Restglied \(R_2\)

$$ R_2(x) < 0, \forall x \in ]0, \infty[ \text{ mit } \xi = 0 $$

daraus folgt wiederum \(T_2(x) < f(x)\) f�r alle \(x\) \(\in\) ]0, \(\infty\)[.

Damit gilt

\begin{align*}
  & T_1(x) > f(x) > T_2(x) \\
\Leftrightarrow & x > \ln(1 + x) > x - \frac{x^2}{2}, \text{ f�r } x \in ]0, \infty[
\end{align*}
\end{document}