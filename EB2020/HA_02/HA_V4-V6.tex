% Created 2020-08-20 Do 22:56
% Intended LaTeX compiler: pdflatex
\documentclass[11pt]{article}
\usepackage[latin1]{inputenc}
\usepackage[T1]{fontenc}
\usepackage{graphicx}
\usepackage{grffile}
\usepackage{longtable}
\usepackage{wrapfig}
\usepackage{rotating}
\usepackage[normalem]{ulem}
\usepackage{amsmath}
\usepackage{textcomp}
\usepackage{amssymb}
\usepackage{capt-of}
\usepackage{hyperref}
\author{Moaz Haque, Felix Oechelhaeuser, Leo Pirker, Dennis Schulze}
\date{\today}
\title{Analysis I und Lineare Algebra f�r Ingenieurwissenschaften \large  \\ Hausaufgabe 02}
\hypersetup{
 pdfauthor={Moaz Haque, Felix Oechelhaeuser, Leo Pirker, Dennis Schulze},
 pdftitle={Analysis I und Lineare Algebra f�r Ingenieurwissenschaften \large  \\ Hausaufgabe 02},
 pdfkeywords={},
 pdfsubject={},
 pdfcreator={Emacs 26.3 (Org mode 9.3.6)}, 
 pdflang={English}}
\begin{document}

\maketitle
\tableofcontents


\section{Aufgabe 1}
\label{sec:orgd37cbcd}
\subsection{a)}
\label{sec:orga8586f3}
$$\sum_{k=1}^{n} k^2 = \frac{n(n + 1)(2n + 1)}{6}$$

IA:
  Es gilt
$$
  \sum_{k = 1}^{1} k^2 = \frac{1(1 + 1)(2 \cdot 1 + 1)}{6}
  = \frac{1 \cdot 2 \cdot 3}{6}
  = 1
  = 1^2
$$

IV:
  Es gelte f�r ein beliebiges n \(\in\) \(\mathbb{N}\) mit n > 0
$$\sum_{k = 1}^{n} k^2 = \frac{n(n + 1)(2n + 1)}{6}$$

IB:
  Dann gilt
$$\sum{k = 1}^{n + 1} k^2 = \frac{(n + 1)(n + 2)(2n + 3)}{6}$$

IS:
  Es gilt
$$
  \sum_{k = 1}^{n + 1} k^2 = \sum_{k = 1}^{n} k^2 + (n + 1)^2
  = \frac{n(n + 1)(2n + 1)}{6} + (n + 1)^2
$$
$$= \frac{n(n + 1)(2n + 1)}{6} + \frac{6(n + 1)^2}{6}$$
$$= \frac{n(n + 1)(2n + 1) + 6(n + 1)^2}{6}$$
$$= \frac{n(n + 1)(2n + 1) + 6(n + 1)(n + 1)}{6}$$
$$= \frac{n(n + 1)(2n + 1) + 3(n + 1)(2n + 2)}{6}$$
$$= \frac{n(n + 1)(2n + 1) + 3(n + 1)(2n + 1) + 3(n + 1)}{6}$$
$$= \frac{(n + 2)(n + 1)(2n + 1) + (n + 1)(2n + 1) + 3(n + 1)}{6}$$
$$= \frac{(n + 2)(n + 1)(2n + 1) + (n + 1)(2n + 4)}{6}$$
$$= \frac{(n + 2)(n + 1)(2n + 1) + 2(n + 1)(n + 2)}{6}$$
$$= \frac{(n + 2)(n + 1)(2n + 3)}{6}$$

Damit gilt f�r alle n \(\in\) \(\mathbb{N}\) mit n > 0

$$\sum_{k = 1}^{n} k^2 = \frac{n(n + 1)(2n + 1)}{6}$$

\subsection{b)}
\label{sec:orgf440103}
$$\prod_{k=2}^{n} 1 - \frac{1}{k^2}) = \frac{n + 1}{2n}$$

IA:
  Es gilt
$$\prod_{k = 2}^{2} 1 - \frac{1}{k^2} = \frac{2 + 1}{4}$$
$$= \frac{3}{4}$$
$$= 1 - \frac{1}{4}$$

IV:
  Es gelte f�r ein beliebiges n \(\in\) \(\mathbb{N}\) mit n > 1
$$\prod_{k=2}^{n} 1 - \frac{1}{k^2} = \frac{n + 1}{2n}$$

IB:
  Dann gilt
$$\prod_{k=2}^{n + 1} 1 - \frac{1}{k^2} = \frac{n + 2}{2(n + 1)}$$

IS:
  Es gilt
$$
  \prod_{k = 2}^{n + 1} 1 - \frac{1}{k^2}
  = \prod_{k = 2}^{n} 1 - \frac{1}{k^2} \cdot (1 - \frac{1}{(n + 1)^2}
$$
$$= \frac{n + 1}{2n} \cdot (1 - \frac{1}{(n + 1)^2})$$
$$= \frac{n + 1}{2n} \cdot (\frac{(n+1)^2}{(n+1)^2} - \frac{1}{(n + 1)^2})$$
$$= \frac{n + 1}{2n} \cdot \frac{(n+1)^2 - 1}{(n+1)^2}$$
$$= \frac{n + 1}{2n} \cdot \frac{((n+1) - 1)((n+1) + 1)}{(n + 1)^2}$$
$$= \frac{n + 1}{2n} \cdot \frac{n(n + 2)}{(n + 1)(n + 1)}$$
$$
  = \frac{(n + 1) \cdot n(n + 2)}{2n(n + 1)(n + 1)}
  = \frac{n + 2}{2(n + 1)}
$$


Damit gilt f�r alle n \(\in\) \(\mathbb{N}\) mit n > 1
$$\prod_{k=2}^{n} 1 - \frac{1}{k^2} = \frac{n + 1}{2n}$$

\subsection{c)}
\label{sec:org064204d}
\(\forall\) n \(\in\) \(\mathbb{N}\), n \(\ge\) 4, gilt 2\textsuperscript{n} \(\ge\) n\textsuperscript{2}

IA:
  Es gilt f�r n = 4

$$2^4 \ge 4^2$$
$$\Leftrightarrow 16  \ge 16$$

IV:
  Es gelte f�r ein beliebiges n \(\in\) \(\mathbb{N}\) mit n \(\ge\) 4
$$2^n \ge n^2$$

IB:
  Dann gilt
$$2^{n + 1} \ge (n + 1)^2$$

IS:
  Es gilt

$$2^{n + 1} \ge (n + 1)^2$$
$$\Leftrightarrow 2 \cdot 2^n \ge 2 \cdot n^2 = n^2 + 2n + 1$$
$$\Rightarrow n^2 = 2n + 1$$
$$\Leftrightarrow 0 = n^2 - 2n - 1$$
$$\Rightarrow n_1 = 1 - \sqrt{2} \wedge n_2 = 1 - \sqrt{2}$$
$$\Rightarrow n_1 < 4 \wedge n_2 < 4$$
$$\Rightarrow n^2 > 2n + 1$$
$$\Rightarrow 2 \cdot n^2 > n^2 + 2n + 1 = (n + 1)^2$$
$$\Rightarrow 2 \cdot 2^n > (n + 1)^2$$
$$\Leftrightarrow 2^{n + 1} > (n + 1)^2$$


Damit gilt \(\forall\) n \(\in\) \(\mathbb{N}\) mit n \(\ge\) 4
$$2^n \ge n^2$$

\section{Aufgabe 2}
\label{sec:org6426248}
D(f) := "Definitionsbereich von f"

U(f) := "Inverse von f"

\((f_1 \circ f_2)(x) := f_2(f_1(x))\)

\subsection{a)}
\label{sec:orgc083f07}
\(D(f_1) = \mathbb{R}\)

\(D(f_2) = \mathbb{R}\) $\backslash$ \{-1\}

\subsection{b)}
\label{sec:orgc687b48}
$$(f_1 \circ f_2)(x)$$
$$= f_1(f_2(x))$$
$$= f_1(\frac{1}{x + 1})$$
$$= \frac{1}{(x + 1)^4} - 1$$

\(D(f_1 \circ f_2) = \mathbb{R}\) $\backslash$ \{-1\}

$$(f_2 \circ f_1)(x)$$
$$= f_2(f_1(x))$$
$$= f_2(x^4 - 1)$$
$$= \frac{1}{x^4}$$

\(D(f_2 \circ f_1) = \mathbb{R}\) $\backslash$ \{0\}

\subsection{c)}
\label{sec:org359d51e}
U(f\textsubscript{1})(]0, 15])

= [-2, -1[ \(\cup\) ]1, 2]

\subsection{d)}
\label{sec:org96e730a}
f\textsubscript{1} ist nach unten beschr�nkt mit m = -1.

f\textsubscript{1} ist nicht nach oben beschr�nkt.

\section{Aufgabe 3}
\label{sec:orgc41d136}
\subsection{a)}
\label{sec:org2087d51}
f(x) = y = \(\frac{2x + 3}{x + 2}\)

Beim Inversen von f gilt:
$$x = \frac{2y + 3}{y + 2}$$
$$\Leftrightarrow x(y + 2) = 2y + 3$$
$$\Leftrightarrow xy + 2x = 2y + 3$$
$$\Leftrightarrow xy - 2y = 3 - 2x$$
$$\Leftrightarrow y(x - 2) = 3 - 2x$$
$$\Leftrightarrow y = \frac{3 - 2x}{x - 2}$$

\(\Rightarrow\) U(f)(x) = y = \(\frac{3 - 2x}{x - 2}\), x ungleich 2


D\textsubscript{U(f)} = \(\mathbb{R}\) $\backslash$ \{2\}

\subsection{b)}
\label{sec:org6d0a2b2}
$$(f \circ U(f))(y) = f( \frac{3 - 2y}{y - 2} )$$
$$
  = \frac{ \frac{2(3 - 2y)}{y - 2} + 3 }{ \frac{3 - 2y}{y - 2} + 2 }
  = \frac{ \frac{2(3 - 2y) + 3(y - 2)}{y - 2} }{ \frac{3 - 2y}{y - 2} + 2 }
  = \frac{ \frac{6 - 4y + 3y - 6}{y - 2} }{ \frac{3 - 2y}{y - 2} + 2 }
$$
$$
  = \frac{ \frac{-y}{y - 2} }{ \frac{3 - 2y}{y - 2} + 2 }
  = \frac{ \frac{-y}{y - 2} }{ \frac{3 - 2y + 2(y - 2)}{y - 2} }
  = \frac{ \frac{-y}{y - 2} }{ \frac{-1}{y - 2} }
  = y
$$

\subsection{c)}
\label{sec:org58348ae}
Sei h \(\in\) \(\mathbb{R}\), h > 0, h ist sehr klein

Es gilt f�r den Nenner:
$$(-2 + h) + 2 > 0$$

Daraus folgt f�r den Z�hler:
$$2(-2 + h) + 3 < 0$$
$$\Rightarrow f(-2 + h) < 0$$

Es gilt f�r den Z�hler:
$$2x + 3 = 0 \text{ mit } x \in \mathbb{R}$$
$$\Leftrightarrow x = \frac{3}{2}$$

Daraus folgt f�r den Nenner:
$$\frac{3}{2} + 2 \neq 0$$
$$\Rightarrow f(\frac{3}{2}) = 0$$
$$\Rightarrow -2 + h < \frac{3}{2} \wedge f(-2 + h) < f(\frac{3}{2})$$
\(\Rightarrow\) f ist monoton wachsend

\section{Aufgabe 4}
\label{sec:orgecc18aa}
\subsection{a)}
\label{sec:org5c95cc4}
$$\exp(x^2 - 9) - 1 = 0$$
$$\Leftrightarrow \exp(x^2 - 9) = 1$$
$$\Rightarrow x^2 - 9 = 0$$ nach \(\exp(0) = 1\)
$$\Leftrightarrow x^2 = 9$$
$$\Rightarrow x = -3 \wedge x = 3$$

\subsection{b)}
\label{sec:org5b686f4}
$$f_2(x) = \ln{ \frac{x^2 - 4x + 3}{x^2 - 6x + 8} }$$
$$= \ln{ \frac{(x - 3)(x - 1)}{(x - 2)(x - 4)} }$$
$$=> x \neq 2 \wedge x \neq 4$$

Im folgenden wird untersucht, wann der Funktionswerte Betr�ge annehmen, die
gr��er als 0 sind:


F�r Nenner:

Fall 1 (>0):
$$x - 2 > 0 \wedge x - 4 > 0$$
$$\Rightarrow x > 2 \wedge x > 4$$
$$\Rightarrow I_1 = ]4, \infty[$$


Fall 2 (>0):
$$x - 2 < 0 \wedge x - 4 < 0$$
$$\Rightarrow x < 2 \wedge x < 4$$
$$\Rightarrow I_2 = ]-\infty, 2[$$


Fall 3 (<0):
$$x - 2 < 0 \vee x - 4 < 0$$
$$\Rightarrow x < 2 \vee x < 4$$
$$\Rightarrow I_3 = ]2, 4[$$


F�r Z�hler:

Fall 1 (>0):
$$x - 3 > 0 \wedge x - 1 > 0$$
$$\Rightarrow x > 3 \wedge x > 1$$
$$\Rightarrow I_4 = ]3, \infty[$$


Fall 2 (>0):
$$x - 3 < 0 \wedge x - 1 < 0$$
$$\Rightarrow x < 3 \wedge x < 1$$
$$\Rightarrow I_5 = ]-\infty, 1[$$


Fall 3 (<0):
$$x - 3 < 0 \vee x - 1 < 0$$
$$\Rightarrow x < 3 \vee x < 1$$
$$\Rightarrow I_6 = ]1, 3[$$


Nenner und Z�hler sind > 0:
$$L_1 = (I_1 \cap I_4) \cup (I_2 \cap I_5)$$
$$= ]-\infty, 1[ \cup ]4, \infty[$$


Nenner und Z�hler sind < 0:
$$L_2 = I_3 \cap I_6$$
$$= ]2, 3[$$


$$D(f_2) = L_1 \cup L_2$$
$$= ]-\infty, 1[ \cup ]2, 3[ \cup ]4, \infty[$$
$$= \mathbb{R} \ ([1, 2] \cup [3, 4])$$

\subsection{c)}
\label{sec:orgd1b637d}
Sinus und Cosinus sind beschr�nkt auf Werte im Intervall [-1, 1]

\(\Rightarrow\) der Betrag des Produkts kann auch nur h�chstens 1 sein


f3(-x)

=  cos(-x) sin(-x) "Sinus ist ungerade und Cosinus ist gerade"

=  cos(x) (-sin(x))

= -(cos(x) sin(x))

= -f3(x)

\(\Rightarrow\) f3 ist ungerade

\section{Aufgabe 5}
\label{sec:org6faca73}
\subsection{a)}
\label{sec:org34a51c0}
$$\sin{2x} = \sin{x + x}$$
$$= \sin{x} \cos{x} + \cos{x} \sin{x}$$
$$= 2 \sin{x} \cos{x}$$

\subsection{b)}
\label{sec:orgeddbc89}
$$\cos{\frac{7\pi}{12}} = \cos{(\frac{\pi}{4} + \frac{\pi}{3})}$$
$$= \cos{\frac{\pi}{4}} \cos{\frac{\pi}{3}} - \sin{\frac{\pi}{4}} \sin{\frac{\pi}{3}}$$
$$= \frac{\sqrt{2}}{2} \cdot \frac{1}{2} - \frac{\sqrt{2}}{2} \cdot \frac{\sqrt{3}}{2}$$
$$= \frac{\sqrt{2}}{4} - \frac{\sqrt{6}}{4}$$
$$= \frac{\sqrt{2} - \sqrt{6}}{4}$$
\end{document}