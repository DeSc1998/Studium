% Created 2020-09-01 Di 21:20
% Intended LaTeX compiler: pdflatex
\documentclass[11pt]{article}
\usepackage[latin1]{inputenc}
\usepackage[T1]{fontenc}
\usepackage{graphicx}
\usepackage{grffile}
\usepackage{longtable}
\usepackage{wrapfig}
\usepackage{rotating}
\usepackage[normalem]{ulem}
\usepackage{amsmath}
\usepackage{textcomp}
\usepackage{amssymb}
\usepackage{capt-of}
\usepackage{hyperref}
\author{Moaz Haque, Felix Oechelhaeuser, Leo Pirker, Dennis Schulze}
\date{\today}
\title{Analysis I und Lineare Algebra f�r Ingenieurwissenschaften \large  \\ Hausaufgabe 05 - Geuter 29}
\hypersetup{
 pdfauthor={Moaz Haque, Felix Oechelhaeuser, Leo Pirker, Dennis Schulze},
 pdftitle={Analysis I und Lineare Algebra f�r Ingenieurwissenschaften \large  \\ Hausaufgabe 05 - Geuter 29},
 pdfkeywords={},
 pdfsubject={},
 pdfcreator={Emacs 27.1 (Org mode 9.3.6)}, 
 pdflang={English}}
\begin{document}

\maketitle
\tableofcontents



\section{Aufagbe 1}
\label{sec:org079493a}
\begin{equation*}
  \begin{bmatrix}
    0 & 3 & -2 \\
    1 & 1 & 0 \\
    2 & 0 & 1
  \end{bmatrix}
  \begin{bmatrix}
    x_1 \\
    x_2 \\
    x_3
  \end{bmatrix} =
  \begin{bmatrix}
    8 \\
    4 \\
    3 
  \end{bmatrix}
\Leftrightarrow
  \left[ \begin{array}{@{}ccc|c@{}}
    0 & 3 & -2 & 8 \\
    1 & 1 & 0 & 4 \\
    2 & 0 & 1 & 3
  \end{array} \right]
\end{equation*}

\begin{equation*}
\begin{aligned}
  \xrightarrow{\text{I} \leftrightarrow \text{II}}
  & \left[ \begin{array}{@{}ccc|c@{}}
    1 & 1 & 0 & 4 \\
    0 & 3 & -2 & 8 \\
    2 & 0 & 1 & 3
  \end{array} \right] \\
  \xrightarrow{\text{III} - 2\text{II}}
  & \left[ \begin{array}{@{}ccc|c@{}}
    1 & 1 & 0 & 4 \\
    0 & 3 & -2 & 8 \\
    0 & -2 & 1 & -5
  \end{array} \right] \\
  \xrightarrow{2\text{II}, 3\text{III}}
  & \left[ \begin{array}{@{}ccc|c@{}}
    1 & 1 & 0 & 4 \\
    0 & 6 & -4 & 16 \\
    0 & -6 & 3 & -15
  \end{array} \right] \\
  \xrightarrow{\text{III} + \text{II}}
  & \left[ \begin{array}{@{}ccc|c@{}}
    1 & 1 & 0 & 4 \\
    0 & 6 & -4 & 16 \\
    0 & 0 & -1 & 1
  \end{array} \right] \\
  \xrightarrow{-\text{III}, \frac{1}{6}\text{II}}
  & \left[ \begin{array}{@{}ccc|c@{}}
    1 & 1 & 0 & 4 \\
    0 & 1 & \frac{-4}{6} & \frac{16}{6} \\
    0 & 0 & 1 & -1
  \end{array} \right] \\
  \xrightarrow{\text{II} + \frac{4}{6}\text{III}}
  & \left[ \begin{array}{@{}ccc|c@{}}
    1 & 1 & 0 & 4 \\
    0 & 1 & 0 & 2 \\
    0 & 0 & 1 & -1
  \end{array} \right] \\
  \xrightarrow{\text{I} - \text{II}}
  & \left[ \begin{array}{@{}ccc|c@{}}
    1 & 0 & 0 & 2 \\
    0 & 1 & 0 & 2 \\
    0 & 0 & 1 & -1
  \end{array} \right]
\end{aligned}
\end{equation*}

Daraus folgt \(x_1 = 2\), \(x_2 = 2\) und \(x_3 = -1\).

\pagebreak

\section{Aufgabe 2}
\label{sec:orgb242b2c}
\subsection{a)}
\label{sec:orgce35ed5}
\begin{equation*}
  \begin{bmatrix}
    -2 & 2i & 0 & 2 & 0 \\
    -4 & 4i & -i & 7 & -2 \\
    1 & i & 2 & -1 & 2
  \end{bmatrix}
  \begin{bmatrix}
    x_1 \\
    x_2 \\
    x_3 \\
    x_4 \\
    x_5
  \end{bmatrix} =
  \begin{bmatrix}
    -4 \\
    -8 \\
    4
  \end{bmatrix}
\Leftrightarrow
  \left[ \begin{array}{@{}ccccc|c@{}}
    -2 & 2i & 0 & 2 & 0 & -4 \\    
    -4 & 4i & -i & 7 & -2 & -8 \\
    1 & i & 2 & -1 & 2 & 4
  \end{array} \right]
\end{equation*}

\begin{equation*}
\begin{aligned}
  \xrightarrow{-\frac{1}{2}\text{I}}
  & \left[ \begin{array}{@{}ccccc|c@{}}
    1 & -i & 0 & -1 & 0 & 2 \\    
    -4 & 4i & -i & 7 & -2 & -8 \\
    1 & i & 2 & -1 & 2 & 4
  \end{array} \right] \\
  \xrightarrow{\text{III} - \text{I}}
  & \left[ \begin{array}{@{}ccccc|c@{}}
    1 & -i & 0 & -1 & 0 & 2 \\    
    -4 & 4i & -i & 7 & -2 & -8 \\
    0 & 2i & 2 & 0 & 2 & 2
  \end{array} \right] \\
  \xrightarrow{\text{II} + 4\text{I}}
  & \left[ \begin{array}{@{}ccccc|c@{}}
    1 & -i & 0 & -1 & 0 & 2 \\    
    0 & 0 & -i & 3 & -2 & 0 \\
    0 & 2i & 2 & 0 & 2 & 2
  \end{array} \right] \\
  \xrightarrow{\text{III} \leftrightarrow \text{II}}
  & \left[ \begin{array}{@{}ccccc|c@{}}
    1 & -i & 0 & -1 & 0 & 2 \\    
    0 & 2i & 2 & 0 & 2 & 2 \\
    0 & 0 & -i & 3 & -2 & 0
  \end{array} \right] \\
  \xrightarrow{\frac{-i}{2}\text{II}, i\text{III}}
  & \left[ \begin{array}{@{}ccccc|c@{}}
    1 & -i & 0 & -1 & 0 & 2 \\    
    0 & 1 & -i & 0 & -i & -i \\
    0 & 0 & 1 & 3i & -2i & 0
  \end{array} \right] \\
  \xrightarrow{\text{II} + i\text{III}}
  & \left[ \begin{array}{@{}ccccc|c@{}}
    1 & -i & 0 & -1 & 0 & 2 \\    
    0 & 1 & 0 & -3 & 2-i & -i \\
    0 & 0 & 1 & 3i & -2i & 0
  \end{array} \right] \\
  \xrightarrow{\text{I} + i\text{II}}
  & \left[ \begin{array}{@{}ccccc|c@{}}
    1 & 0 & 0 & -1-3i & 1+2i & 3 \\    
    0 & 1 & 0 & -3 & 2-i    & -i \\
    0 & 0 & 1 & 3i & -2i    & 0
  \end{array} \right] \\
\end{aligned}
\end{equation*}

\subsection{b)}
\label{sec:orge4f33dc}
Aus dem LGS

\begin{equation*}
  \left[ \begin{array}{@{}ccccc|c@{}}
    1 & 0 & 0 & -1-3i & 1+2i & 0 \\    
    0 & 1 & 0 & -3 & 2-i & 0 \\
    0 & 0 & 1 & 3i & -2i & 0
  \end{array} \right]
\end{equation*}

ergeben sich folgende Gleichungen:

\begin{equation*}
\begin{aligned}
\text{(I)  } 0 &= x_1 + (-1-3i)x_4 + (1+2i)x_5 \\
\text{(II)  } 0 &= x_2 - 3 x_4 + (2-i)x_5 \\
\text{(III)  } 0 &= x_2 + 3i x_4 - 2i x_5
\end{aligned}
\end{equation*}

\subsection{c)}
\label{sec:org5e53843}
Zum �berpr�fen werden die die beiden Vektoren in das hergeleitete System aus 2a) eingesetzt. \\
F�r den ersten der Vektor ergibt sich die Gleichung

\begin{equation*}
  \begin{bmatrix}
    1 & 0 & 0 & -1-3i & 1+2i \\
    0 & 1 & 0 & -3 & 2-i \\
    0 & 0 & 1 & 3i & -2i
  \end{bmatrix}
  \begin{bmatrix}
    3 \\ -i \\ 0 \\ 0 \\ 0
  \end{bmatrix} =
  \begin{bmatrix}
    3 \\
    -i \\
    0
  \end{bmatrix}
\end{equation*}

bei der erkennbar ist, dass sie wahr ist.
Damit ist der erste Vektor eine L�sung des LGS.

F�r den zweiten Vektor ergibt sich die Gleichung

\begin{equation*}
\begin{aligned}
  \begin{bmatrix}
    3 \\
    -i \\
    0
  \end{bmatrix} &\neq
  \begin{bmatrix}
    1 & 0 & 0 & -1-3i & 1+2i \\
    0 & 1 & 0 & -3 & 2-i \\
    0 & 0 & 1 & 3i & -2i
  \end{bmatrix}
  \begin{bmatrix}
    2 \\ 0 \\ 1 \\ 0 \\ i
  \end{bmatrix} \\
&= \begin{bmatrix}
     i \\
     1+2i \\
     3
   \end{bmatrix}
\end{aligned}
\end{equation*}

bei der nach Umformung erkennbar wird, dass sie nicht gleich sein kann.
Damit ist der zweite Vektor keine L�sung des LGS.

\subsection{d)}
\label{sec:org6825b18}

\section{Aufgabe 3}
\label{sec:org0c270d5}
\subsection{a)}
\label{sec:orga01a2d9}
F�r \(A_1\): \\
Nur die Spalten 1 und 2 sind voneinander linear abh�ngig, womit gilt \(\text{Rang}(A_1) = 2\). \\

F�r \(A_2\): \\
3 Zeilen der Matrix sind ungleich 0. Die 3. \uline{oder} 4. Spalte kann
als Linearkombination der anderen Spalten dargestellt werden. 
Damit gilt \(\text{Rang}(A_2) = 3\). \\

F�r \(A_3\): \\
Nur 2 Zeilen sind ungleich 0 und beide Spalten sind linear unabh�ngig voneinander.
Damit gilt \(\text{Rang}(A_3) = 2\). \\

F�r \(A_4\): \\
Die beiden Spalten sind linear unabh�ngig voneinander und es gilt damit \(\text{Rang}(A_4) = 2\).

\subsection{b)}
\label{sec:orgc83649d}
F�r \(A_1\): \\
Zwei der Spalten sind linear abh�ngig (\(\begin{bmatrix} -2 \\ 0 \end{bmatrix} = \lambda \begin{bmatrix} 4 \\ 0 \end{bmatrix}\)),
womit sie keine Basis von Bild(\(A_1\)) ist. \\

F�r \(A_2\): \\


\subsection{c)}
\label{sec:org3bc50de}

\subsection{d)}
\label{sec:org74a8195}

\subsection{e)}
\label{sec:org8779f0b}

\section{Aufgabe 4}
\label{sec:org2e666a6}

\section{Aufgabe 5}
\label{sec:org760d188}
\subsection{a)}
\label{sec:orgb19037d}

\subsection{b)}
\label{sec:org8ce4494}

\subsection{c)}
\label{sec:org9b0adbe}

\subsection{d)}
\label{sec:org35d5d1d}
\end{document}