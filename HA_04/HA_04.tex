% Created 2020-08-27 Do 10:48
% Intended LaTeX compiler: pdflatex
\documentclass[11pt]{article}
\usepackage[latin1]{inputenc}
\usepackage[T1]{fontenc}
\usepackage{graphicx}
\usepackage{grffile}
\usepackage{longtable}
\usepackage{wrapfig}
\usepackage{rotating}
\usepackage[normalem]{ulem}
\usepackage{amsmath}
\usepackage{textcomp}
\usepackage{amssymb}
\usepackage{capt-of}
\usepackage{hyperref}
\author{Moaz Haque, Felix Oechelhaeuser, Leo Pirker, Dennis Schulze}
\date{\today}
\title{Analysis I und Lineare Algebra f�r Ingenieurwissenschaften \large  \\ Hausaufgabe 03 - Geuter 29}
\hypersetup{
 pdfauthor={Moaz Haque, Felix Oechelhaeuser, Leo Pirker, Dennis Schulze},
 pdftitle={Analysis I und Lineare Algebra f�r Ingenieurwissenschaften \large  \\ Hausaufgabe 03 - Geuter 29},
 pdfkeywords={},
 pdfsubject={},
 pdfcreator={Emacs 26.3 (Org mode 9.3.6)}, 
 pdflang={English}}
\begin{document}

\maketitle
\tableofcontents


\section{Aufgabe 1}
\label{sec:orgaa0544f}
\subsection{a)}
\label{sec:org953fb79}
\begin{equation*}
\begin{aligned}
p(z) &= z^2 + 3z - 1 + 2i \\
 &= (z^2 - 2z) + (5z - 5) + (4 + 2i) \\
 &= 2 T_2(z) + 5 T_1(z) + (4 + 2i) T_0(z)
\end{aligned}
\end{equation*}

\subsection{b)}
\label{sec:org31ab93b}
\begin{equation*}
\begin{aligned}
\sin\left( x + \frac{\pi}{3} \right) &= 
  \sin(x) \cos\left( \frac{\pi}{3}\right) + \cos(x) \sin\left( \frac{\pi}{3} \right) \\
 &= \sin(x) \frac{1}{2} + \cos(x) \frac{ \sqrt{3} }{2}
\end{aligned}
\end{equation*}

\section{Aufgabe 2}
\label{sec:org6173b87}
\subsection{a)}
\label{sec:org6bf5f6e}
Umformung der Regel von \(T_1\):

\begin{equation*}
\begin{aligned}
0 &= 3x_1 - 2x_2 \\
\Leftrightarrow 2x_2 &= 3x_1 \\
\Leftrightarrow x_2 &= \frac{3}{2}x_1
\end{aligned}
\end{equation*}

Damit gilt f�r \(T_1\):

\begin{equation*}
\begin{aligned}
T_1 &= \left\{ \begin{bmatrix} x_1 \\ \frac{3}{2}x_1 \end{bmatrix} \in \mathbb{R}^2 \right\} \\
 &= \text{span}\left\{ \begin{bmatrix} 1 \\ \frac{3}{2} \end{bmatrix} \right\}
\end{aligned}
\end{equation*}

Damit gelten die Addition und Skalarmultiplikation in \(T_1\). Und daraus folgt:

$$T_1 \subset \mathbb{R}^2$$

Umformung der Regel von \(T_2\):

\begin{equation*}
\begin{aligned}
1 &= x_1 x_2^3 \\
\Leftrightarrow \frac{1}{x_1} &= x_2^3 \\
\Leftrightarrow \frac{1}{\sqrt[3]{x_1}} &= x_2
\end{aligned}
\end{equation*}

Damit gilt f�r \(T_2\):

$$T_2 = \left\{ \begin{bmatrix} x_1 \\ \frac{1}{\sqrt[3]{x_1}} \end{bmatrix} \in \mathbb{R}^2 \right\}$$

Daraus folgt, dass keiner der Vektoren in \(T_2\) als vielfaches eines anderen Vektors aus \(T_2\) dargestellt werden kann.
Ebenso gibt es keine zwei Vektoren \(v, w \in T_2, v \neq -w\) f�r die gilt \(v + w \in T_2\).

Daraus folgt:

$$T_2 \not\subset \mathbb{R}^2$$

\subsection{b)}
\label{sec:orgb40de93}
\(\forall f, g \in T\) gilt \(f + g \in T\), da alle \(f\) und \(g\) an der Stelle \(x = -2\) eine Nullstelle besitzen 
und \(f + g\) muss dem zu folge ebenfalls bei \(x = -2\) eine Nullstelle besitzen.
Analog gilt auch \(\forall f \in T\) mit \(\lambda \in \mathbb{R}\), dass \(\lambda f \in T\). Auch hier hat \(f\) eine Nullstelle bei \(x = -2\),
die bei \(\lambda f\) erhalten bleibt.

Also gilt:

$$T \subset V$$

\section{Aufgabe 3}
\label{sec:orgf6710ad}
\subsection{a}
\label{sec:orgae4114c}
\subsection{b}
\label{sec:orgdd09506}
\subsection{c}
\label{sec:org1255727}
\section{Aufgabe 4}
\label{sec:orgb8c1749}
\subsection{a}
\label{sec:org40706e9}
\subsection{b}
\label{sec:org646f70e}
\subsection{c}
\label{sec:org1b15e21}
\section{Aufgabe 5}
\label{sec:org93df22d}
\subsection{a}
\label{sec:org934b3c8}
\subsection{b}
\label{sec:orgabfca12}
\section{Aufgabe 6}
\label{sec:orgdaf3809}
\subsection{a}
\label{sec:org51ea342}
\subsection{b}
\label{sec:orge07265c}
\end{document}