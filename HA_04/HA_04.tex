% Created 2020-08-27 Do 19:45
% Intended LaTeX compiler: pdflatex
\documentclass[11pt]{article}
\usepackage[latin1]{inputenc}
\usepackage[T1]{fontenc}
\usepackage{graphicx}
\usepackage{grffile}
\usepackage{longtable}
\usepackage{wrapfig}
\usepackage{rotating}
\usepackage[normalem]{ulem}
\usepackage{amsmath}
\usepackage{textcomp}
\usepackage{amssymb}
\usepackage{capt-of}
\usepackage{hyperref}
\author{Moaz Haque, Felix Oechelhaeuser, Leo Pirker, Dennis Schulze}
\date{\today}
\title{Analysis I und Lineare Algebra f�r Ingenieurwissenschaften \large  \\ Hausaufgabe 03 - Geuter 29}
\hypersetup{
 pdfauthor={Moaz Haque, Felix Oechelhaeuser, Leo Pirker, Dennis Schulze},
 pdftitle={Analysis I und Lineare Algebra f�r Ingenieurwissenschaften \large  \\ Hausaufgabe 03 - Geuter 29},
 pdfkeywords={},
 pdfsubject={},
 pdfcreator={Emacs 26.3 (Org mode 9.3.6)}, 
 pdflang={English}}
\begin{document}

\maketitle
\tableofcontents


\section{Aufgabe 1}
\label{sec:org48552e3}
\subsection{a)}
\label{sec:org02a21a2}
\begin{equation*}
\begin{aligned}
p(z) &= z^2 + 3z - 1 + 2i \\
 &= (z^2 - 2z) + (5z - 5) + (4 + 2i) \\
 &= 2 T_2(z) + 5 T_1(z) + (4 + 2i) T_0(z)
\end{aligned}
\end{equation*}

\subsection{b)}
\label{sec:orgfab2617}
\begin{equation*}
\begin{aligned}
\sin\left( x + \frac{\pi}{3} \right) &= 
  \sin(x) \cos\left( \frac{\pi}{3}\right) + \cos(x) \sin\left( \frac{\pi}{3} \right) \\
 &= \sin(x) \frac{1}{2} + \cos(x) \frac{ \sqrt{3} }{2}
\end{aligned}
\end{equation*}

\section{Aufgabe 2}
\label{sec:orgb93a584}
\subsection{a)}
\label{sec:org2d70de7}
Umformung der Regel von \(T_1\):

\begin{equation*}
\begin{aligned}
0 &= 3x_1 - 2x_2 \\
\Leftrightarrow 2x_2 &= 3x_1 \\
\Leftrightarrow x_2 &= \frac{3}{2}x_1
\end{aligned}
\end{equation*}

Damit gilt f�r \(T_1\):

\begin{equation*}
\begin{aligned}
T_1 &= \left\{ \begin{bmatrix} x_1 \\ \frac{3}{2}x_1 \end{bmatrix} \in \mathbb{R}^2 \right\} \\
 &= \text{span}\left\{ \begin{bmatrix} 1 \\ \frac{3}{2} \end{bmatrix} \right\}
\end{aligned}
\end{equation*}

Damit gelten die Addition und Skalarmultiplikation in \(T_1\). Und daraus folgt:

$$T_1 \subset \mathbb{R}^2$$

Umformung der Regel von \(T_2\):

\begin{equation*}
\begin{aligned}
1 &= x_1 x_2^3 \\
\Leftrightarrow \frac{1}{x_1} &= x_2^3 \\
\Leftrightarrow \frac{1}{\sqrt[3]{x_1}} &= x_2
\end{aligned}
\end{equation*}

Damit gilt f�r \(T_2\):

$$T_2 = \left\{ \begin{bmatrix} x_1 \\ \frac{1}{\sqrt[3]{x_1}} \end{bmatrix} \in \mathbb{R}^2 \right\}$$

Daraus folgt, dass keiner der Vektoren in \(T_2\) als vielfaches eines anderen Vektors aus \(T_2\) dargestellt werden kann.
Ebenso gibt es keine zwei Vektoren \(v, w \in T_2, v \neq -w\) f�r die gilt \(v + w \in T_2\).

Daraus folgt:

$$T_2 \not\subset \mathbb{R}^2$$

\subsection{b)}
\label{sec:org4812868}
\(\forall f, g \in T\) gilt \(f + g \in T\), da alle \(f\) und \(g\) an der Stelle \(x = -2\) eine Nullstelle besitzen 
und \(f + g\) muss dem zu folge ebenfalls bei \(x = -2\) eine Nullstelle besitzen.
Analog gilt auch \(\forall f \in T\) mit \(\lambda \in \mathbb{R}\), dass \(\lambda f \in T\). Auch hier hat \(f\) eine Nullstelle bei \(x = -2\),
die bei \(\lambda f\) erhalten bleibt.

Also gilt:

$$T \subset V$$

\section{Aufgabe 3}
\label{sec:org2e768c9}
\subsection{a)}
\label{sec:org95ed2bc}
Ansatz zum �berpr�fen auf lineare Abh�nigkeit:

\begin{equation*}
\begin{aligned}
0 &= a_0 \begin{bmatrix} 3 \\ 0 \\ 2  \end{bmatrix} +
     a_1 \begin{bmatrix} -3 \\ 5 \\ 3 \end{bmatrix} +
     a_2 \begin{bmatrix} 0 \\ 1 \\ -1 \end{bmatrix} \\
 &= \begin{bmatrix} 3 a_0 \\ 0 \\ 2 a_0 \end{bmatrix} +
    \begin{bmatrix} -3 a_1 \\ 5 a_1 \\ 3 a_1 \end{bmatrix} +
    \begin{bmatrix} 0 \\ a_2 \\ -a_2 \end{bmatrix} \\
 &= \begin{bmatrix} 3 (a_0 - a_1) \\ 5 a_1 + a_2 \\ 2 a_0 + 3 a_1 - a_2 \end{bmatrix}
\end{aligned}
\end{equation*}

Daraus folgt \(a_0 = a_1\) aus der ersten Komponente, \(5a_1 = -a_2\) aus der zweiten Komponente.
Die Folgerung aus der dritte Komponente steht im widerspruch zu den ersten beiden,
da diese immer ungleich 0 ist, wenn die ersten beiden Komponenten gleich 0 sind.

Damit gilt nur die triviale L�sung \(a_0 = a_1 = a_2 = 0\) und damit sind die Vektoren linear unabh�ngig.

\subsection{b)}
\label{sec:org33a8f9c}
Ansatz zum �berpr�fen auf lineare Abh�nigkeit:

\begin{equation*}
\begin{aligned}
0 &= a_1 p_1(z) + a_2 p_2(z) + a_3 p_3(z) + a_4 p_4(z) \\
 &=  a_1 + a_2 (3z - 1) + a_3 (z^2 - 1) + a_4 (2z^3 - 3z^2) \\
 &=  a_1 + 3 a_2 z - a_2 + a_3 z^2 - a_3 + 2 a_4 z^3 - 3 a_4 z^2 \\
 &= (a_1 - a_2 - a_3) + 3 a_2 z + (a_3 - 3 a_4) z^2 + 2 a_4 z^3
\end{aligned}
\end{equation*}

Daraus ergeben sich durch Koeffizientenvergleich die Gleichungen:

\begin{equation*}
\begin{aligned}
\text{(I) } 0 &= a_1 - a_2 - a_3 \\
\text{(II) } 0 &= 3 a_2 \\
\text{(III) } 0 &= a_3 - 3 a_4 \\
\text{(IV) } 0 &= 2 a_4
\end{aligned}
\end{equation*}

Aus II und IV folgen jewals \(a_2 = a_4 = 0\). Damit folgt aus III \(a_3 = 0\). Und schlie�lich folgt dann aus I \(a_1 = 0\).

Damit ist die triviale L�sung \(a_1 = a_2 = a_3 = a_4 = 0\) die einzige L�sung und damit sind die Polynome/Vektoren linear unabh�ngig.

\subsection{c)}
\label{sec:orgb90b498}

\section{Aufgabe 4}
\label{sec:orge8a35e6}
\subsection{a)}
\label{sec:org61a8568}
\(\left\{ \begin{bmatrix} 1 \\ -2 \end{bmatrix} \right\}\) ist kein Erzeugendensystem von \(\mathbb{R}^2\),
da die \(\text{span} \left\{ \begin{bmatrix} 1 \\ -2 \end{bmatrix} \right\}\) eine echte Teilmenge von \(\mathbb{R}^2\) ist.
Man kann mit Leichtigkeit ein Element in \(\mathbb{R}^2\) finden, welches nicht in der Span enthalten ist.


Man �berpr�fe, ob sich der Beispielvektor \(\begin{bmatrix} 2 \\ 2 \end{bmatrix}\) in der Span enthalten ist

\begin{equation*}
\begin{aligned}
\begin{bmatrix} 2 \\ 2 \end{bmatrix} &= r \begin{bmatrix} 1 \\ -2 \end{bmatrix} \\
 &= \begin{bmatrix} r \\ -2r \end{bmatrix} \\
\Leftrightarrow 0 &= \begin{bmatrix} r - 2 \\ -2r - 2 \end{bmatrix}
\end{aligned}
\end{equation*}

Aus der ersten Komponente w�rde folgen \(r = 2\) und aus der zweiten folgt \(r = -1\), was ein widerspruch ist.
Daraus folgt \(\begin{bmatrix} 2 \\ 2 \end{bmatrix} \not\in \text{ span} \left\{ \begin{bmatrix} 1 \\ -2 \end{bmatrix} \right\}\), aber
\(\begin{bmatrix} 2 \\ 2 \end{bmatrix} \in \mathbb{R}^2\).

\subsection{b)}
\label{sec:org2086537}
Die Erzeugendensysteme \(E_1 = \left\{ \begin{bmatrix} 1 \\ 0 \end{bmatrix}, \begin{bmatrix} 1 \\ -2 \end{bmatrix} \right\}\) und 
\(E_2 = \left\{ \begin{bmatrix} 1 \\ 0 \end{bmatrix}, \begin{bmatrix} -2 \\ 4 \end{bmatrix} \right\}\) sind die einzigen Basen, bei denen
die beiden Vektoren voneinander linear unabh�ngig sind.

\subsection{c)}
\label{sec:org2a3e7c7}
Ansatz f�r \(E_1\):

\begin{equation*}
\begin{aligned}
\begin{bmatrix} 0 \\ 1 \end{bmatrix} &= a_1 \begin{bmatrix} 1 \\ 0 \end{bmatrix} + a_2 \begin{bmatrix} 1 \\ -2 \end{bmatrix}
  = \begin{bmatrix} a_1 + a_2 \\ -2 a_2 \end{bmatrix} \\
\Leftrightarrow 0 &= \begin{bmatrix} a_1 + a_2 \\ -2 a_2 - 1 \end{bmatrix}
\end{aligned}
\end{equation*}

Daraus ergeben sich die Gleichungen:

\begin{equation*}
\begin{aligned}
\text{(I) } 0 &= a_1 + a_2 \\
\text{(II) } 0 &= -2 a_2 - 1
\end{aligned}
\end{equation*}

Aus II folgt \(a_2 = \frac{-1}{2}\) und damit folgt aus I \(a_1 = \frac{1}{2}\).
Der Koordinatenvektor ist damit \(\begin{bmatrix} \frac{1}{2} \\ \frac{-1}{2} \end{bmatrix}\) f�r \(E_1\).

Ansatz f�r \(E_2\):

\begin{equation*}
\begin{aligned}
\begin{bmatrix} 0 \\ 1 \end{bmatrix} &= a_1 \begin{bmatrix} 1 \\ 0 \end{bmatrix} + a_2 \begin{bmatrix} -2 \\ 4 \end{bmatrix}
 = \begin{bmatrix} a_1 - 2 a_2 \\ 4 a_2 \end{bmatrix} \\
\Leftrightarrow 0 &= \begin{bmatrix} a_1 - 2 a_2 \\ 4 a_2 - 1 \end{bmatrix}
\end{aligned}
\end{equation*}

Daraus ergeben sich die Gleichungen:

\begin{equation*}
\begin{aligned}
\text{(I) } 0 &= a_1 - 2 a_2 \\
\text{(II) } 0 &= 4 a_2 - 1
\end{aligned}
\end{equation*}

Aus II folgt \(a_2 = \frac{1}{4}\) und damit folgt aus I \(a_1 = \frac{1}{2}\).
Der Koordinatenvektor ist damit \(\begin{bmatrix} \frac{1}{2} \\ \frac{1}{4} \end{bmatrix}\) f�r \(E_2\).

\section{Aufgabe 5}
\label{sec:org32413bb}
\subsection{a)}
\label{sec:org73824c2}
$$ A \in \mathbb{C}^{1, 2}, B \in \mathbb{R}^{2, 2}, C \in \mathbb{R}^{2, 2}, D \in \mathbb{C}^{1, 2} $$

Addition \(B + C\) und \(A + D\) sind m�glich, da diese das gleiche Format besitzen.

\begin{equation*}
\begin{aligned}
A + D &= \begin{bmatrix} -i & 2+i \end{bmatrix}
       + \begin{bmatrix} 12-i & -4+2i \end{bmatrix} \\
    &= \begin{bmatrix} -i+12-i & 2+i + (-4) +2i \end{bmatrix} \\
    &= \begin{bmatrix} 12-2i & -2+3i \end{bmatrix}
\end{aligned}
\end{equation*}

\begin{equation*}
\begin{aligned}
B + C &= \begin{bmatrix} 0 & 3 \\ -1 & 0 \end{bmatrix} 
       + \begin{bmatrix} 1 & 13 \\ -1 & 2 \end{bmatrix} \\ 
    &= \begin{bmatrix} 1 & 16 \\ -2 & 2 \end{bmatrix}
\end{aligned}
\end{equation*}

\subsection{b)}
\label{sec:org80e57a8}
$$ A \in \mathbb{C}^{1, 2},
B \in \mathbb{C}^{2, 2},
C \in \mathbb{C}^{3, 2},
D \in \mathbb{C}^{3, 3} $$

Multiplikation von \(A \cdot B\), \(C \cdot B\) und \(D \cdot C\) ist m�glich,
da die Spaltenzahl vom erst genannten mit der Zeilenanzahl vom zweiten �bereinstimmt.

\begin{equation*}
\begin{aligned}
A \cdot B &= \begin{bmatrix} 1 & -2i \end{bmatrix}
             \begin{bmatrix}
                1 & 1-i \\
                i & 0
             \end{bmatrix} \\
  &= \begin{bmatrix} (1)(1) + (-2i)(i) & (1)(1 - i) + (-2i)(0) \end{bmatrix} \\
  &= \begin{bmatrix} 3 & 1 - i \end{bmatrix}
\end{aligned}
\end{equation*}

\begin{equation*}
\begin{aligned}
C \cdot B & = \begin{bmatrix}
        2-i & 3i \\
        0 & i \\
        2+3i & 41
    \end{bmatrix} \begin{bmatrix}
        1 & 1-i \\
        i & 0
    \end{bmatrix} \\ 
    &= \begin{bmatrix}
        (2-i)(1) + (3i)(i) & (2-i)(1-i) + (3i)(0) \\
        (0)(1) + (i)(i) & (0)(1-i) + (i)(0) \\
        (2+3i)(1) + (41)(i) & (2+3i)(1-i) + (41)(0)
    \end{bmatrix} \\
    &= \begin{bmatrix}
        -1-i & 1 - 3i \\
        -1 & 0 \\
        2 + 44i & 5 + i
    \end{bmatrix}
\end{aligned}
\end{equation*}

\begin{equation*}
\begin{aligned}
D \cdot C & = \begin{bmatrix}
        1-i & 2 & -2-3i \\
        -1-3i & 0 & -1 \\
        0 & 1 & 0
    \end{bmatrix} \begin{bmatrix}
        2-i & 3i \\
        0 & i \\
        2+3i & 41 \\
    \end{bmatrix} \\
    & = \begin{bmatrix}
        (1-i)(2-i) + (2)(0) + (-2-3i)(2+3i) & (1-i)(3i) + (2)(i) + (-2-3i)(41) \\
        (-1-3i)(2-i) + (0)(0) + (-1)(2+3i) & (-1-3i)(3i) + (0)(i) + (-1)(41) \\
        0 & i
    \end{bmatrix} \\
    & = \begin{bmatrix}
        (2 - i - 2i + 1) + 0 + (-4 - 6i - 6i + 9) & (3i + 3) + (2i) + (-82 - 123i) \\
        (-2 + i - 6i - 3) + 0 + (-2 - 3i) & (-3i + 9) + 0 - 41 \\
        0 & i
    \end{bmatrix} \\
    & = \begin{bmatrix}
        8 - 15i & -79 - 118i \\
        -7 - 8i & -32 - 3i  \\
        0 & i
    \end{bmatrix}
\end{aligned}
\end{equation*}

\section{Aufgabe 6}
\label{sec:org97e3b06}
\subsection{a)}
\label{sec:orgf6d38e0}

\begin{equation*}
\begin{aligned}
A^2 = A \cdot A & = \begin{bmatrix}
        0 & -1 & 2 \\
        0 & 0 & 1 \\
        0 & 0 & 0 \\
    \end{bmatrix} \begin{bmatrix}
        0 & -1 & 2 \\
        0 & 0 & 1 \\
        0 & 0 & 0 \\
    \end{bmatrix} \\
    & = \begin{bmatrix}
        0 & 0 & -1 \\
        0 & 0 & 0 \\
        0 & 0 & 0 \\
    \end{bmatrix}
\end{aligned}
\end{equation*}

\begin{equation*}
\begin{aligned}
    A^3 = A^2 \cdot A & = \begin{bmatrix}
        0 & 0 & -1 \\
        0 & 0 & 0 \\
        0 & 0 & 0 \\
    \end{bmatrix}
    \begin{bmatrix}
        0 & -1 & 2 \\
        0 & 0 & 1 \\
        0 & 0 & 0 \\
    \end{bmatrix} \\
    & = \begin{bmatrix}
        0 & 0 & 0 \\
        0 & 0 & 0 \\
        0 & 0 & 0 \\
    \end{bmatrix}
\end{aligned}
\end{equation*}

\subsection{b)}
\label{sec:org2f6cbc0}
Alle \(A_n, n > 2\) sind Nullmatritzen.

\begin{equation*}
B = \left\{
        \begin{bmatrix}
            1 & 0 & 0 \\
            0 & 1 & 0 \\
            0 & 0 & 1 \\
        \end{bmatrix}, 
        \begin{bmatrix}
            0 & 1 & 0 \\
            0 & 0 & 0 \\
            0 & 0 & 0 \\
        \end{bmatrix},
        \begin{bmatrix}
            0 & 0 & 1 \\
            0 & 0 & 0 \\
            0 & 0 & 0 \\
        \end{bmatrix}, 
        \begin{bmatrix}
            0 & 0 & 0 \\
            0 & 0 & 1 \\
            0 & 0 & 0 \\
        \end{bmatrix}
     \right\}
\end{equation*}

\(\dim(B) = 4\)
\end{document}