% Created 2020-09-21 Mo 18:58
% Intended LaTeX compiler: pdflatex
\documentclass[a4paper,11pt]{article}
\usepackage[latin1]{inputenc}
\usepackage[T1]{fontenc}
\usepackage{graphicx}
\usepackage{grffile}
\usepackage{longtable}
\usepackage{wrapfig}
\usepackage{rotating}
\usepackage[normalem]{ulem}
\usepackage{amsmath}
\usepackage{textcomp}
\usepackage{amssymb}
\usepackage{capt-of}
\usepackage{hyperref}
\author{Moaz Haque, Felix Oechelhaeuser, Leo Pirker, Dennis Schulze}
\date{\today}
\title{Analysis I und Lineare Algebra f�r Ingenieurwissenschaften \large  \\ Hausaufgabe 08 - Geuter 29}
\hypersetup{
 pdfauthor={Moaz Haque, Felix Oechelhaeuser, Leo Pirker, Dennis Schulze},
 pdftitle={Analysis I und Lineare Algebra f�r Ingenieurwissenschaften \large  \\ Hausaufgabe 08 - Geuter 29},
 pdfkeywords={},
 pdfsubject={},
 pdfcreator={Emacs 27.1 (Org mode 9.4)}, 
 pdflang={English}}
\begin{document}

\maketitle
\tableofcontents

\pagebreak

\section{Aufgabe 1}
\label{sec:orgac0c2a1}
\subsection{a)}
\label{sec:org843efd6}
Da der Grenzwert des Terms bei direktem Einsetzten nicht definiert ist ("\(\frac{0}{0}\)"),
muss die Regel von de l'Hospital angewand werden. Also gilt

\begin{align*}
  \lim_{x \rightarrow 0} \frac{e^x - 1}{\sin(2x)} &= \lim_{x \rightarrow 0} \frac{e^x}{2\cos(2x)} \\
  &= \frac{\lim_{x \rightarrow 0}  e^x}{\lim_{x \rightarrow 0}  2\cos(2x)} \\
  &= \frac{1}{2}
\end{align*}

Damit existiert dessen Grenzwert.

\subsection{b)}
\label{sec:orgc085f5f}
Siehe a) \\
Es gilt

\begin{align*}
  \lim_{x \rightarrow 0} \frac{\cos(x) - 1}{x^2} &= \lim_{x \rightarrow 0} \frac{-\sin(x)}{2x)} \\
  &= \lim_{x \rightarrow 0} \frac{-\cos(x)}{2} \\
  &= \frac{\lim_{x \rightarrow 0}  -\cos(x)}{\lim_{x \rightarrow 0} 2} \\
  &= \frac{-1}{2}
\end{align*}

Damit existiert dessen Grenzwert nach zweifacher Anwendung der Regel.

\section{Aufgabe 2}
\label{sec:orgbe702ff}
\subsection{a)}
\label{sec:orgcfd441e}
\subsection{b)}
\label{sec:org6784b16}
\subsection{c)}
\label{sec:orgb8b75fa}
\section{Aufgabe 3}
\label{sec:org28c6f10}
\section{Aufgabe 4}
\label{sec:org3b78abe}
\section{Aufgabe 5}
\label{sec:org4af11d4}
\subsection{a)}
\label{sec:org60d1e8c}
Es gilt f�r die Ableitung von \(g\)

\begin{align*}
  g'(t) &= \left( (s'(t))^2 + (\omega s(t))^2 \right)' \\
   &= ((s'(t))^2)' + ((\omega s(t))^2)' \\
   &= 2 s'(t) s''(t) + 2 (\omega s(t)) (\omega s'(t)) \\
   &= 2 s'(t) s''(t) + 2 \omega^2 s(t) s'(t) \\
   &= 2 s'(t) (s''(t) + \omega^2 s(t))
\end{align*}

Da \(s''(t) = -\omega^2 s(t)\) gilt, folgt daraus

\begin{align*}
  g'(t) &= 2 s'(t) ((-\omega^2 s(t)) + \omega^2 s(t)) \\
   &= 2 s'(t) \cdot 0 = 0
\end{align*}

Da \(g'(t) \geq 0\) und \(g'(t) \leq 0\) mit \(t \in \mathbb{R}\) gelten, folgt daraus,
dass \(g(t) = c\), mit \(c \in \mathbb{R}\) und f�r alle \(t \in \mathbb{R}\). \\
Daraus folgt, dass \((s'(t))^2 + (\omega s(t))^2\) ebenfalls konstant sein muss.
Daraus folgt wiederum, dass \((s'(t))^2\) und \((\omega s(t))^2\) konstant sind.
Daraus folgt dann, dass \(\omega s(t)\) konstant ist und damit gilt \(s'(t) = 0\) f�r alle \(t \in \mathbb{R}\),
woraus wiederum aus der urspr�ngilchen Annahme folgt \(s(t) = 0\) f�r alle \(t \in \mathbb{R}\).
Damit gilt dann f�r auch \(g(t) = 0\), f�r alle \(t \in \mathbb{R}\).

\subsection{b)}
\label{sec:org4117870}
Es gilt f�r die erste und zweite Ableitung von \(h\)

\begin{align*}
  h'(t) &= \left( s(t) - s(0)\cos(\omega t) - \frac{s'(0)}{\omega}\sin(\omega t) \right)' \\
   &= s'(t) + s(0)\omega \sin(\omega t) - s'(0) \cos(\omega t) \\
  \\
  h''(t) &= \left( s'(t) + s(0)\omega \sin(\omega t) - s'(0) \cos(\omega t) \right)' \\
   &= s''(t) + s(0)\omega^2 \cos(\omega t) + s'(0)\omega \sin(\omega t) \\
   &= -\omega^2 s(t) + s(0)\omega^2 \cos(\omega t) + s'(0)\omega \sin(\omega t) \\
   &= -\omega^2 \left( s(t) - s(0)\cos(\omega t) - \frac{s'(0)}{\omega}\sin(\omega t) \right) \\
   &= -\omega^2 h(t)
\end{align*}

Was in 5a) gezeigt wurde, kann analog f�r \(h\) gezeigt werden. Damit gilt

\begin{align*}
  h(t) = 0 &= s(t) - s(0)\cos(\omega t) - \frac{s'(0)}{\omega}\sin(\omega t) \\
\Leftrightarrow s(t) &= s(0)\cos(\omega t) + \frac{s'(0)}{\omega}\sin(\omega t)
\end{align*}

was \textbf{eine} L�sung der urspr�nglichen Differenzialgleichung ist.

\section{Aufgabe 6}
\label{sec:org0453766}
\section{Aufgabe 7}
\label{sec:org0a5f378}
\end{document}