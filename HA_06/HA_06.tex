% Created 2020-09-04 Fr 23:56
% Intended LaTeX compiler: pdflatex
\documentclass[a4paper,11pt]{article}
\usepackage[latin1]{inputenc}
\usepackage[T1]{fontenc}
\usepackage{graphicx}
\usepackage{grffile}
\usepackage{longtable}
\usepackage{wrapfig}
\usepackage{rotating}
\usepackage[normalem]{ulem}
\usepackage{amsmath}
\usepackage{textcomp}
\usepackage{amssymb}
\usepackage{capt-of}
\usepackage{hyperref}
\usepackage{tikz}
\usepackage{tkz-euclide}
\usetikzlibrary{arrows.meta,arrows}
\author{Moaz Haque, Felix Oechelhaeuser, Leo Pirker, Dennis Schulze}
\date{\today}
\title{Analysis I und Lineare Algebra f�r Ingenieurwissenschaften \large  \\ Hausaufgabe 06 - Geuter 29}
\hypersetup{
 pdfauthor={Moaz Haque, Felix Oechelhaeuser, Leo Pirker, Dennis Schulze},
 pdftitle={Analysis I und Lineare Algebra f�r Ingenieurwissenschaften \large  \\ Hausaufgabe 06 - Geuter 29},
 pdfkeywords={},
 pdfsubject={},
 pdfcreator={Emacs 27.1 (Org mode 9.3.6)}, 
 pdflang={English}}
\begin{document}

\maketitle
\tableofcontents


\section{Aufgabe 1}
\label{sec:orgadce55f}
\subsection{a)}
\label{sec:org34c381a}
\begin{align}
    f_{B_1,B_1} = K_{B_1}(f(K_{B_1}^{-1}(e_i))) = K_{B_1}(f(b_i)) \\
    K_{B_1} = \mathbb{R}^3 \rightarrow \mathbb{R}^3, \begin{bmatrix}
        a \\
        b \\
        c
    \end{bmatrix} \rightarrow \begin{bmatrix}
        \lambda_1 \\
        \lambda_2 \\
        \lambda_3
    \end{bmatrix}
    \intertext{da:}
    \lambda_1 \begin{bmatrix}
        1 \\
        0 \\
        0
    \end{bmatrix} + \lambda_2 \begin{bmatrix}
        0 \\
        1 \\
        0
    \end{bmatrix} + \lambda_3 \begin{bmatrix}
        0 \\
        0 \\
        1
    \end{bmatrix} = \begin{bmatrix}
        a \\
        b \\
        c
    \end{bmatrix} \Leftrightarrow \left[ \begin{array}{@{}ccc|c@{}}
        1 & 0 & 0 & a \\
        0 & 1 & 0 & b \\
        0 & 0 & 1 & c
    \end{array} \right]
    \intertext{1. Spalte:}
    f_{B_1,B_1}(e_1) = K_{B_1}\left(f\left(\begin{bmatrix}
        1 \\
        0 \\
        0
    \end{bmatrix}\right)\right) = K_{B_1}\left(\begin{bmatrix}
        1 \\
        0 \\
        2 \\
    \end{bmatrix}\right) = \begin{bmatrix}
        1 \\
        0 \\
        2 \\
    \end{bmatrix}
    \intertext{2. Spalte:}
    f_{B_1,B_1}(e_2) = K_{B_1}\left(f\left(\begin{bmatrix}
        0 \\
        1 \\
        0
    \end{bmatrix}\right)\right) = K_{B_1}\left(\begin{bmatrix}
        -2 \\
        0 \\
        1
    \end{bmatrix}\right) = \begin{bmatrix}
        -2 \\
        0 \\
        1
    \end{bmatrix}
    \intertext{3. Spalte:}
    f_{B_1,B_1}(e_3) = K_{B_1}\left(f\left(\begin{bmatrix}
        0 \\
        0 \\
        1
    \end{bmatrix}\right)\right) = K_{B_1}\left(\begin{bmatrix}
        -2 \\
        1 \\
        3
    \end{bmatrix}\right) = \begin{bmatrix}
        -2 \\
        1 \\
        3
    \end{bmatrix}
    \\
    f_{B_1,B_1} = \begin{bmatrix}
        e_1 & e_2 & e_3
    \end{bmatrix} = \begin{bmatrix}
        1 & -2 & -2 \\
        0 & 0 & 1 \\
        2 & 1 & 3
    \end{bmatrix}
\end{align}

\subsection{b)}
\label{sec:org190778a}
\begin{align}
    id_{B_2,B_1} & = K_{B_1}(K_{B_2}^{-1}(e_i)) = K_{B_1}(v_i)
    \intertext{1. Spalte:}
    id_{B_2,B_1}(e_1) & = K_{B_1}(v_1) = K_{B_1}\left(\begin{bmatrix}
        1 \\
        0 \\
        -1
    \end{bmatrix}\right) = \begin{bmatrix}
        1 \\
        0 \\
        -1
    \end{bmatrix}
    \intertext{2. Spalte:}
    id_{B_2,B_1}(e_2) & = K_{B_1}(v_2) = K_{B_1}\left(\begin{bmatrix}
        2 \\
        1 \\
        -1
    \end{bmatrix}\right) = \begin{bmatrix}
        2 \\
        1 \\
        -1
    \end{bmatrix}
    \intertext{3. Spalte:}
    id_{B_2,B_1}(e_3) & = K_{B_1}(v_3) = K_{B_1}\left(\begin{bmatrix}
        -2 \\
        1 \\
        4
    \end{bmatrix}\right) = \begin{bmatrix}
        -2 \\
        1 \\
        4
    \end{bmatrix}
    \\
    \Rightarrow id_{B_2,B_1} & = \begin{bmatrix}
        1 & 2 & -2 \\
        0 & 1 & 1 \\
        -1 & -1 & 4
    \end{bmatrix}
\end{align}

\subsection{c)}
\label{sec:orge9e216e}
\begin{align}
    id_{B_1,B_2} = & (id_{B_2,B_1})^{-1} \\
    \begin{bmatrix}
        id_{B_2,B_1} & I_3
    \end{bmatrix} = & \left[ \begin{array}{@{}ccc|ccc@{}}
        1 & 2 & -2 & 1 & 0 & 0 \\
        0 & 1 & 1 & 0 & 1 & 0 \\
        -1 & -1 & 4 & 0 & 0 & 1
    \end{array} \right] \xrightarrow{III+I} \left[ \begin{array}{@{}ccc|ccc@{}}
        1 & 2 & -2 & 1 & 0 & 0 \\
        0 & 1 & 1 & 0 & 1 & 0 \\
        0 & 1 & 2 & 1 & 0 & 1
    \end{array} \right] \\
    \xrightarrow{III-II}
    & \left[ \begin{array}{@{}ccc|ccc@{}}
        1 & 2 & -2 & 1 & 0 & 0 \\
        0 & 1 & 1 & 0 & 1 & 0 \\
        0 & 0 & 1 & 1 & -1 & 1 
    \end{array} \right]
    \xrightarrow{II-III}
    \left[ \begin{array}{@{}ccc|ccc@{}}
        1 & 2 & -2 & 1 & 0 & 0 \\
        0 & 1 & 0 & -1 & 2 & -1 \\
        0 & 0 & 1 & 1 & -1 & 1 
    \end{array} \right]
    \\
    \xrightarrow{I-2II}
    & \left[ \begin{array}{@{}ccc|ccc@{}}
        1 & 0 & -2 & 3 & -4 & 2 \\
        0 & 1 & 0 & -1 & 2 & -1 \\
        0 & 0 & 1 & 1 & -1 & 1 
    \end{array} \right]
    \\
    \xrightarrow{I+2II}
    & \left[ \begin{array}{@{}ccc|ccc@{}}
        1 & 0 & 0 & 5 & -6 & 4 \\
        0 & 1 & 0 & -1 & 2 & -1 \\
        0 & 0 & 1 & 1 & -1 & 1 
    \end{array} \right] = \begin{bmatrix}
        I_3 & (id_{B_2,B_1})^{-1}
    \end{bmatrix}
    \\
    \Rightarrow id_{B_1,B_2} = (id_{B_2,B_1})^{-1} = & \begin{bmatrix}
        5 & -6 & 4 \\
        -1 & 2 & -1 \\
        1 & -1 & 1
    \end{bmatrix}
\end{align}

\subsection{d)}
\label{sec:orgba945d3}
\begin{align*}
    f_{B_2,B_2} &= id_{B_1,B_2} f_{B_1,B_1}(id_{B_1,B_2})^{-1} \\
    & = id_{B_1,B_2} f_{B_1,B_1}id_{B_2,B_1} \\
    & \begin{bmatrix}
        5 & -6 & 4 \\
        -1 & 2 & -1 \\
        1 & -1 & 1
    \end{bmatrix} \begin{bmatrix}
        1 & -2 & -2 \\
        0 & 0 & 1 \\
        -2 & 1 & 3 
    \end{bmatrix} \begin{bmatrix}
        1 & 2 & -2 \\
        0 & 1 & 1 \\
        -1 & -1 & 4
    \end{bmatrix}
    \\
    & = \begin{bmatrix}
        5 & -6 & 4 \\
        -1 & 2 & -1 \\
        1 & -1 & 1
    \end{bmatrix} \begin{bmatrix}
        1+2 & 2-2+2 & -2-2-8 \\
        -1 & -1 & 4 \\
        -2-3 & -4+1-3 & 4+1+12
    \end{bmatrix}
    \\
    & = \begin{bmatrix}
        5 & -6 & 4 \\
        -1 & 2 & -1 \\
        1 & -1 & 1
    \end{bmatrix} \begin{bmatrix}
        3 & 2 & -12 \\
        -1 & -1 & 4 \\
        -5 & -5 & 17
    \end{bmatrix}
    \\
    & = \begin{bmatrix}
        15+6-20 & 10+6-24 & -60-24+68 \\
        -3-2+5 & -2-2+6 & 12+8-17 \\
        3+1-5 & 2+1-6 & -12-4+17
    \end{bmatrix}
    = \begin{bmatrix}
        1 & -8 & -16 \\
        0 & 2 & 3 \\
        -1 & -3 & 1
    \end{bmatrix}
\end{align*}

\section{Aufgabe 2}
\label{sec:org9042c32}
\subsection{a)}
\label{sec:orge1eb92a}
Da \(f\) eine lineare Abbildung ist, gelten

$$f(x + y) = f(x) + f(y) \text{ und } f(\lambda x) = \lambda f(x)$$

und damit gilt

\begin{equation*}
\begin{aligned}
f(4x) &= 2 f(2x + 3) - 3 f(2) \\
 &= 2(-x + 1) - 3(3x + 2) \\
 &= -2x + 2 - 9x - 6 \\
 &= -11x - 4
\end{aligned}
\end{equation*}

\subsection{b)}
\label{sec:orgd3d0f36}
Ansatz f�r \(\text{Bild}(f)\):

\begin{equation*}
\begin{aligned}
  f(x) &= \frac{1}{4} f(4x) = \frac{-11}{4}x - 1 \\
  f(1) &= \frac{1}{2} f(2) = \frac{3}{2}x + 1
\end{aligned}
\end{equation*}

Pr�fe, ob \(f(x)\) und \(f(1)\) linear abh�ngig sind:

Seien a\textsubscript{1}, a\textsubscript{2} \(\in\) \(\mathbb{R}\) damit gelte

\begin{equation*}
\begin{aligned}
 0 &= a_1 f(1) + a_2 f(x) \\
 &= a_1 (\frac{3}{2}x + 1) + a_2 (\frac{-11}{4}x - 1) \\
 &= \frac{3 a_1}{2}x + a_1 + \frac{-11 a_2}{4}x - a_2 \\
 &= a_1 - a_2 + \frac{6 a_1 - 11 a_2}{4}x
\end{aligned}
\end{equation*}

Durch Koeffizientenvergleich ergeben sich die Gleichungen

\begin{equation*}
\begin{aligned}
\text{(I)  } 0 &= a_1 - a_2 \\
\text{(II)  } 0 &= \frac{1}{4}(6 a_1 - 11 a_2) \\
  \Leftrightarrow 0 &= 6 a_1 - 11 a_2
\end{aligned}
\end{equation*}

wo aus I folgt \(a_1 = a_2\) und aus II folgt \(a_1 = \frac{11}{6} a_2\), was I widersprechen w�rde,
wenn gilt \(a_1 \neq 0 \neq a_2\). Daruas folgt \(a_1 = a_2 = 0\) und damit sind \(f(x)\) und \(f(1)\)
linear unabh�ngig.

Damit gilt \(\text{Bild}(f) = \text{span}\{f(1), f(x)\}\), woraus folgt \(\dim(\text{Bild}(f)) = 2\).

\subsection{c)}
\label{sec:org91d4671}
Es gelten \(\dim(\text{Bild}(f)) = 2\) und \(\dim(\mathbb{R}[x]_{\leq 1}) = 2\). \\
Damit ergibt sich aus der Dimensionsformel

\begin{equation*}
\begin{aligned}
 \dim(\mathbb{R}[x]_{\leq 1}) &= \dim(\text{Kern}(f)) + \dim(\text{Bild}(f)) \\
\Leftrightarrow \dim(\text{Kern}(f)) &= \dim(\mathbb{R}[x]_{\leq 1}) - \dim(\text{Bild}(f)) = 2 - 2 = 0
\end{aligned}
\end{equation*}

Damit gilt f�r die Basis \(B_K\) vom Kern(\(f\))

$$ B_K = \{ \} $$

\subsection{d)}
\label{sec:org43d722d}
Bestimmen von \(K_{B_1}\) mit a, b, \(\lambda\)\textsubscript{1}, \(\lambda\)\textsubscript{2} \(\in\) \(\mathbb{R}\):

\begin{equation*}
\begin{aligned}
ax + b \mapsto \begin{bmatrix} \lambda_1 \\ \lambda_2 \end{bmatrix} \\
\text{mit  } \lambda_1 (2x + 3) + \lambda_2 2 &= ax + b \\
  2\lambda_1 x + \lambda_1 3 + \lambda_2 2 &= ax + b
\end{aligned}
\end{equation*}

Durch Koeffizientenvergleich ergibt sich:

\begin{equation*}
\begin{aligned}
\text{(I)  } a &= 2\lambda_1 \\ 
\text{(II)  } b &= \lambda_1 3 + \lambda_2 2
\end{aligned}
\end{equation*}

Daraus ergeben sich einmal nach I \(\lambda_1 = \frac{a}{2}\) \\
und anschlie�end nach II \(\lambda_2 = \frac{2b - 3a}{4}\).

Daraus folgt

$$K_{B_1}: ax + b \mapsto \begin{bmatrix} \frac{a}{2} \\ \frac{2b - 3a}{4} \end{bmatrix}$$


Bestimmen von \(K_{B_2}\) mit c, d, \(\beta\)\textsubscript{1}, \(\beta\)\textsubscript{2} \(\in\) \(\mathbb{R}\):

\begin{equation*}
\begin{aligned}
cx + d \mapsto \begin{bmatrix} \beta_1 \\ \beta_2 \end{bmatrix} \\
\text{mit  } \beta_1 (-x + 1) + \beta_2 (3x + 2) &= cx + d \\
  -\beta_1 x + \beta_1 + 3\beta_2 x + 2\beta_2 &= cx + d \\
   (3\beta_2 - \beta_1)x + \beta_1 + 2\beta_2 &= cx + d
\end{aligned}
\end{equation*}

Durch Koeffizientenvergleich ergibt sich:

\begin{equation*}
\begin{aligned}
  \begin{bmatrix} c \\ d \end{bmatrix} &=
  \begin{bmatrix} -1 & 3 \\ 1 & 2 \end{bmatrix} 
  \begin{bmatrix} \beta_1 \\ \beta_2 \end{bmatrix} \\
e &= A \beta
\end{aligned}
\end{equation*}

was umgeschrieben werden kann zu

\begin{equation*}
\begin{aligned}
  \left[A|e\right] = &\left[ \begin{array}{@{}cc|c@{}}
    -1 & 3 & c \\
    1 & 2 & d
  \end{array} \right]\\
\xrightarrow{\text{II} + \text{I}}
  &\left[ \begin{array}{@{}cc|c@{}}
    -1 & 3 & c \\
    0 & 5 & c+d
  \end{array} \right]
\xrightarrow{-\text{I}}
  \left[ \begin{array}{@{}cc|c@{}}
    1 & -3 & -c \\
    0 & 5 & c+d
  \end{array} \right] \\
\xrightarrow{5\text{I} + 3\text{II}}
  &\left[ \begin{array}{@{}cc|c@{}}
    5 & 0 & -2c+3d \\
    0 & 15 & 3c+3d
  \end{array} \right]
\xrightarrow{\frac{1}{5}\text{I}, \frac{1}{15}\text{II}}
  \left[ \begin{array}{@{}cc|c@{}}
    1 & 0 & \frac{-2c+3d}{5} \\
    0 & 1 & \frac{c+d}{5}
  \end{array} \right]
\end{aligned}
\end{equation*}

und daraus folgt \(\beta_1 = \frac{-2c+3d}{5}\) und \(\beta_2 = \frac{c+d}{5}\).

Damit ergibt sich

$$K_{B_2}: cx + d \mapsto \begin{bmatrix} \frac{-2c+3d}{5} \\ \frac{c+d}{5} \end{bmatrix}$$

Daraus folgt mit b\textsubscript{n} \(\in\) \(B_1\)

\begin{equation*}
\begin{aligned}
f_{B_1, B_2}(e_n) &= K_{B_2}(f(K_{B_1}^{-1}(e_n))) \text{ mit } n \in \left\{ 1, 2 \right\} \\
\Leftrightarrow f_{B_1, B_2}(e_n) &= K_{B_2}(f(b_n))
\end{aligned}
\end{equation*}

Einsetzen der basen:

\begin{equation*}
\begin{aligned}
K_{B_2}(f(b_1)) &= K_{B_2}(f(2x+3)) &= K_{B_2}(-x+1) &= \begin{bmatrix} 1 \\ 0 \end{bmatrix}\\
K_{B_2}(f(b_2)) &= K_{B_2}(f(2)) &= K_{B_2}(3x+2) &= \begin{bmatrix} 0 \\ 1 \end{bmatrix}
\end{aligned}
\end{equation*}

Damit gilt

\begin{equation*}
f_{B_1, B_2} = \begin{bmatrix} 1 & 0 \\ 0 & 1 \end{bmatrix}
\end{equation*}

\section{Aufgabe 3}
\label{sec:org1d35fca}
Behauptung: \\
Die Zahlenfolgt \((x_n)_n\) konvergiert gegen 0, wenn \(n\) gegen \(\infty\) strebt. \\

\uline{zu zeigen}: \(\forall\) \(\epsilon\) > 0 gibt es ein \(N_{\epsilon} \in \mathbb{N}\), so dass gilt

$$|x_n - 0| = \left| \frac{2}{n + 1} - 0 \right| < \epsilon, \forall n \geq N_{\epsilon}$$

\pagebreak

Es gilt

\begin{equation*}
\begin{aligned}
\left| \frac{2}{n + 1} - 0 \right| &< \epsilon \\
\Leftrightarrow \frac{2}{n + 1} &< \epsilon \\
\Leftrightarrow \frac{2}{\epsilon} &< n + 1 \\
\Leftrightarrow \frac{2}{\epsilon} - 1 &< n
\end{aligned}
\end{equation*}

Man w�hlt nun ein \(N_{\epsilon} \in \mathbb{N}\) mit \(N_{\epsilon} > \frac{2}{\epsilon} - 1\).
Dann gilt \(\forall\) \(n\) \(\ge\) \(N_{\epsilon}\):

$$\left| \frac{2}{n + 1} - 0 \right| = \frac{2}{n + 1} \leq \frac{2}{N_{\epsilon} + 1} < \epsilon$$

Als Beispiel:

\begin{equation*}
\begin{aligned}
\epsilon = \frac{1}{100} \Leftrightarrow \frac{1}{\epsilon} &= 100 \\
\Rightarrow N_{\epsilon} &> \frac{2}{\epsilon} - 1 = 199 \\
\Rightarrow N_{\epsilon} &= 200
\end{aligned}
\end{equation*}

\section{Aufgabe 4}
\label{sec:org815ef01}
Ansatz f�r PBZ mit \(A\), \(B\) \(\in\) \(\mathbb{R}\):

\begin{equation*}
\frac{3}{(k + 1)(k + 2)} = \frac{A}{k + 1} + \frac{B}{k + 2}
\end{equation*}

Umgeformt nach \(A\) mit \(k = -1\):

$$ A = \frac{3}{(-1) + 2} = 3 $$


Umgeformt nach \(B\) mit \(k = -2\):

$$ B = \frac{3}{(-2) + 1} = -3 $$

Dann gilt mit \(n \in \mathbb{N}\)

\begin{equation*}
\begin{aligned}
\sum_{k=0}^{n} \frac{3}{(k + 1)(k + 2)} &= \sum_{k=0}^{n} \left( \frac{3}{k + 1} + \frac{-3}{k + 2} \right) \\
 &= \sum_{k=0}^{n} \frac{3}{k + 1} - \sum_{k=0}^{n} \frac{3}{k + 2} \\
 &= 3 + \frac{3}{n + 2}
\end{aligned}
\end{equation*}

und nun muss noch gezeigt werden, dass der Term \(\frac{3}{n + 2}\) konvergiert.

\uline{Behauptung}:
$$\lim_{n \rightarrow \infty} \frac{3}{n + 2} = 0$$

Sei \(\epsilon > 0\) beliebig. \\
Dann gilt

\begin{equation*}
\begin{aligned}
  \left| \frac{3}{n + 2} - 0 \right| &= \frac{3}{n + 2} < \epsilon \\
\Leftrightarrow \frac{3}{\epsilon} &< n + 2 \Leftrightarrow \frac{3}{\epsilon} - 2 < n
\end{aligned}
\end{equation*}

Man w�hlt nun ein \(N_{\epsilon} \in \mathbb{N}\) mit \(N_{\epsilon} > \frac{3}{\epsilon} - 2\). \\
Dann gilt

$$ \left| \frac{3}{n + 2} - 0 \right| = \frac{3}{n + 2} \leq \frac{3}{N_{\epsilon} + 2} < \epsilon $$

und damit gilt

\begin{equation*}
\lim_{n \rightarrow \infty} \left( \sum_{k=0}^{n} \frac{3}{(k + 1)(k + 2)} \right) = 3
\end{equation*}

\section{Aufgabe 5}
\label{sec:orga2305c7}
\subsection{a)}
\label{sec:org90777a9}
Untersuchung von \(a_n\): \\
Es gilt

\begin{equation*}
\begin{aligned}
a_n &= \frac{42n^2 - 7n + 3}{n^2 + 2n + 10} \\
 &= \frac{n^2}{n^2} \cdot \frac{42 - \frac{7}{n} + \frac{3}{n^2}}{1 + \frac{2}{n} + \frac{10}{n^2}} \\
 &= \frac{42 - \frac{7}{n} + \frac{3}{n^2}}{1 + \frac{2}{n} + \frac{10}{n^2}}
\end{aligned}
\end{equation*}

\pagebreak

Dann gilt f�r den Grenzwert von \(a_n\)

\begin{equation*}
\begin{aligned}
  \lim_{n \rightarrow \infty} a_n &= \lim_{n \rightarrow \infty} \left( \frac{42 - \frac{7}{n} + \frac{3}{n^2}}{1 + \frac{2}{n} + \frac{10}{n^2}} \right) \\
 &= \frac{ \lim_{n \rightarrow \infty} 42 - 7\lim_{n \rightarrow \infty} \frac{1}{n} + 3\lim_{n \rightarrow \infty} \frac{1}{n^2} }
         { \lim_{n \rightarrow \infty} 1 + 2\lim_{n \rightarrow \infty} \frac{1}{n} + 10\lim_{n \rightarrow \infty} \frac{1}{n^2} } \\
 &= \frac{42 - 7 \cdot 0 + 3 \cdot 0 \cdot 0}{1 + 2 \cdot 0 + 10 \cdot 0 \cdot 0} \\
 &= \frac{42}{1} = 42
\end{aligned}
\end{equation*}


Untersuchung von \(b_n\): \\
Es gilt \(\forall r \in \mathbb{R}\)

$$\left| \sin(r) \right| \leq 1 \text{ und ist divergent, wenn } r \rightarrow \pm \infty$$

daraus folgt

\begin{equation*}
\begin{aligned}
  \lim_{n \rightarrow \infty} b_n &= \lim_{n \rightarrow \infty} \left( \frac{1}{n^2} \sin\left( \frac{n \pi^2}{3} \right) \right) \\
  &= \lim_{n \rightarrow \infty} \frac{1}{n^2} \cdot \lim_{n \rightarrow \infty} \sin\left( \frac{n \pi^2}{3} \right) \\
  &= 0
\end{aligned}
\end{equation*}


Untersuchung von \(c_n\): \\

\subsection{b)}
\label{sec:org0d679f0}
Untersuchung der Folge \(x_n\): \\
Es gilt

\begin{equation*}
\begin{aligned}
\lim_{n \rightarrow \infty} \frac{2n - 1}{n} &= \lim_{n \rightarrow \infty} \frac{n}{n} \cdot (2 - \frac{1}{n}}) \\
 &= \lim_{n \rightarrow \infty} (2 - \frac{1}{n}})
 &= 2 - 0 = 2
\end{aligned}
\end{equation*}

und es gilt

\begin{equation*}
\begin{aligned}
\lim_{n \rightarrow \infty} \frac{2n + 2}{n} &= \lim_{n \rightarrow \infty} \frac{n}{n} \cdot (2 + \frac{2}{n}}) \\
 &= \lim_{n \rightarrow \infty} (2 + \frac{2}{n}})
 &= 2 - 0 = 2
\end{aligned}
\end{equation*}

damit gilt

\begin{equation*}
\lim_{n \rightarrow \infty} x_n = 2
\end{equation*}


Untersuchung von Folge \(y_n\): \\
Es gilt f�r \(\sqrt{x}, x \in \mathbb{R}, x \geq 0\), wenn

$$ x_1 < x_2 \text{ dann ist } \sqrt{x_1} < \sqrt{x_2} \text{ mit } x_1 \geq 0, x_2 > 0 $$

daruas folgt, dass \(\sqrt{n}\) streng monoton steigend ist, woraus folgt, dass sie bestimmt divergent ist. \\
Damit ist auch die Folge \(y_n\) bestimmt divergent, da gilt

$$ \sqrt{n} < y_n $$

\subsection{c)}
\label{sec:org022014f}
Es gilt

\begin{equation*}
\begin{aligned}
\lim_{n \rightarrow \infty} a_n &= \lim_{n \rightarrow \infty} \left( 3 \left( \frac{2}{3} - \frac{1}{n+1} \right) + \left( \frac{1}{3} \right)^n \right) \\
 &= \lim_{n \rightarrow \infty} \left( 2 - \frac{3}{n+1} + \left( \frac{1}{3} \right)^n \right) \\
 &= \lim_{n \rightarrow \infty} \left( 2 - \frac{n}{n} \cdot \frac{\frac{3}{n}}{1+\frac{1}{n}} + \left( \frac{1}{3} \right)^n \right) \\
 &= \lim_{n \rightarrow \infty} 2 - \lim_{n \rightarrow \infty} \frac{\frac{3}{n}}{1+\frac{1}{n}} + \lim_{n \rightarrow \infty} \left( \frac{1}{3} \right)^n \\
 &= 2 - 0 + 0 = 2
\end{aligned}
\end{equation*}

\section{Aufgabe 6}
\label{sec:orgec8d39a}
\subsection{a)}
\label{sec:orga25ac70}
\begin{align*}
    \intertext{Induktionsanfang:}
    x_0 = 1 < 4
    \intertext{Induktionsvoraussetzung:}
    x_n < 4
    \intertext{Induktionsbehauptung:}
    x_n + 1 < 4
    \intertext{Induktionsschluss:}
    x_n < 4 \Leftrightarrow \frac{x}{4} < 1 \Leftrightarrow 4 > \frac{x}{4} + 3 = x_{n+1}
    \intertext{Aus $4 > \frac{x}{4} + 3$ folgt, dass jedes beliebiges $x_{n+1} < 4$ ist,
    womit die Bahauptung $x_n < 4$ bewiesen ist.}
\end{align*}

\subsection{b)}
\label{sec:orgc9f6914}
\begin{align*}
    x_n < 4 
    \Leftrightarrow x_n -4 < 0
    \Leftrightarrow 3x_n - 12 < 0
    \Leftrightarrow 4x_n - 12 < 1x_n
    \\
    \Leftrightarrow x_n - 3 < \frac{x_n}{4}
    \Leftrightarrow x_n < \frac{x_n}{4} + 3 = x_{n+1}
    \\
    \Rightarrow \intertext{Da $x_n < x_{n+1}$ ist die Folge $x_n$ streng monoton wachsend}
\end{align*}

\subsection{c)}
\label{sec:org1a27b84}
\begin{align*}
    \intertext{$x_n$ ist konvergent, da die Folge beschr�nkt ist (aus Aufgabe 6.a folgt: $x_{n+1} < 4$) 
    und die Monotonie folgt aus 6.b.}
\end{align*}

\subsection{d)}
\label{sec:orgf7de911}
\begin{align*}
    x & = \lim \limits_{n \rightarrow \infty} x_{n+1}
    = \lim \limits_{n \rightarrow \infty} \frac{x_n}{4} + 3
    = \frac{x}{4} + 3
    \\
    \Rightarrow x & = \frac{x}{4} + 3
    \Leftrightarrow \frac{3x}{4} = 3
    \Leftrightarrow 3x = 12
    \Leftrightarrow x = 4
    \intertext{Der Grenzwert der Folge $x_n$ ist 4}
\end{align*}
\end{document}