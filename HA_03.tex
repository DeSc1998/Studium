% Created 2020-08-22 Sa 20:11
% Intended LaTeX compiler: pdflatex
\documentclass[11pt]{article}
\usepackage[latin1]{inputenc}
\usepackage[T1]{fontenc}
\usepackage{graphicx}
\usepackage{grffile}
\usepackage{longtable}
\usepackage{wrapfig}
\usepackage{rotating}
\usepackage[normalem]{ulem}
\usepackage{amsmath}
\usepackage{textcomp}
\usepackage{amssymb}
\usepackage{capt-of}
\usepackage{hyperref}
\author{Moaz Haque, Felix Oechelhaeuser, Leo Pirker, Dennis Schulze}
\date{\today}
\title{Analysis I und Lineare Algebra f�r Ingenieurwissenschaften \large  \\ Hausaufgabe 03}
\hypersetup{
 pdfauthor={Moaz Haque, Felix Oechelhaeuser, Leo Pirker, Dennis Schulze},
 pdftitle={Analysis I und Lineare Algebra f�r Ingenieurwissenschaften \large  \\ Hausaufgabe 03},
 pdfkeywords={},
 pdfsubject={},
 pdfcreator={Emacs 26.3 (Org mode 9.3.6)}, 
 pdflang={English}}
\begin{document}

\maketitle
\tableofcontents


\section{Aufgabe 1}
\label{sec:org5f7ae06}
\subsection{a)}
\label{sec:orgad34fd6}
\subsubsection{z\textsubscript{1}}
\label{sec:orgd397a45}
\begin{equation}
\begin{aligned}
z_1 &= 4e^{-\frac{\pi}{6}i}
= 4(\cos(-\frac{\pi}{6}) + i\sin(-\frac{\pi}{6})) \\
&= 4(\frac{\sqrt{3}}{2} - \frac{1}{2}i) \\
&= 2\sqrt{3} - 2i
\end{aligned}
\end{equation}

\subsubsection{z\textsubscript{2}}
\label{sec:org87df7c1}
\begin{equation}
\begin{aligned}
z_2 &= 2e^{\frac{16}{3}\pi i}
= 2e^{(\frac{4}{3} + 4)\pi i}
= 2e^{\frac{4}{3}\pi i} \\
&= -2e^{\frac{1}{3}\pi i} \\
&= -2(\cos(\frac{\pi}{3}) + i\sin(\frac{\pi}{3})) \\
&= -2(\frac{\sqrt{2}}{2} + \frac{\sqrt{2}}{2}i) \\
&= -\sqrt{2} - \sqrt{2}i
\end{aligned}
\end{equation}

\subsection{b)}
\label{sec:org946ea5f}
\subsubsection{z\textsubscript{1}}
\label{sec:org05c6e16}
$$r^2 = (-3)^2 + (-3)^2 = 18$$
$$\tan(\phi) = \frac{-3}{-3}$$
Der Nenner ist kleiner als 0:
$$
  \Rightarrow \phi = \arctan(\frac{-3}{-3}) + \pi
  = \frac{\pi}{4} + \pi = \frac{5\pi}{4}
$$
Damit ist z\textsubscript{1} in Eulerdarstellung:
$$z_1 = \sqrt{18}e^{\frac{5\pi}{4}i}$$

\subsubsection{z\textsubscript{2}}
\label{sec:org55d18e2}
$$r^2 = 2^2 + (\sqrt{12})^2 = 4 + 12 = 16$$
$$\Leftrightarrow r = 4$$

$$\tan(\phi) = \frac{\sqrt{12}}{2}$$
Der Nenner ist gr��er 0:
$$\Rightarrow \phi = \arctan(\frac{\sqrt{12}}{2}) = \frac{\pi}{3}$$
Damit ist z\textsubscript{2} in Eulerdarstellung:
$$z_2 = 4e^{\frac{\pi}{3}i}$$

\section{Aufgabe 2}
\label{sec:org70f6f25}
\subsection{a)}
\label{sec:orgd0eb92b}

\begin{equation}
\begin{aligned}
&z^3 + \sqrt{2} - \sqrt{2}i = 0 \\
\Leftrightarrow &z^3 + 2e^{\frac{-\pi}{4}i} = 0 \\
\Leftrightarrow &z^3 = -2e^{\frac{-\pi}{4}i} \\
\Leftrightarrow &z = \sqrt[3]{-2}e^{\frac{-\pi}{12}i} = z_1
\end{aligned}
\end{equation}

Die anderen L�sungen z\textsubscript{2} und z\textsubscript{3} lassen sich finden, wenn man den Term \(\frac{2k\pi}{3}\) mit k \(\in\) \{1, 2\}
hinzu addiert. Somit erh�lt man:

\begin{equation}
\begin{aligned}
&z_k = \sqrt[3]{-2}e^{(\frac{-\pi}{12} + \frac{2k\pi}{3})i}, k \in {0, 1, 2} \\
\Leftrightarrow &z_k = \sqrt[3]{-2}e^{(\frac{-\pi}{12} + \frac{8k\pi}{12})i} \\
\Leftrightarrow &z_k = \sqrt[3]{-2}e^{\frac{(8k - 1)\pi}{12}i}
\end{aligned}
\end{equation}

\subsection{b)}
\label{sec:orgce307fc}
\begin{equation}
\begin{aligned}
z^5 = \frac{1 + i}{1 - i} &\Leftrightarrow z^5 = \frac{(1 + i)^2}{2} \\
&\Leftrightarrow z^5 = \frac{2i}{2} \Leftrightarrow z^5 = i \\
&\Leftrightarrow z^5 = e^{\frac{\pi}{2}i} \\
&\Leftrightarrow z = e^{\frac{\pi}{10}i}
\end{aligned}
\end{equation}

Analog zu 2a) erh�lt man die anderen L�sungen mit den Term \(\frac{2k\pi}{5}\) mit k \(\in\) \{1, 2, 3, 4\}.
Womit man erh�lt:

\begin{equation}
\begin{aligned}
z_k &= e^{(\frac{\pi}{10} + \frac{2k\pi}{5})i}, k \in {0, 1, 2, 3, 4} \\
\Leftrightarrow z_k &= e^{(\frac{\pi}{10} + \frac{4k\pi}{10})i} \\
\Leftrightarrow z_k &= e^{\frac{(1 + 4k)\pi}{10}i}
\end{aligned}
\end{equation}

\section{Aufgabe 3}
\label{sec:orgd05e79b}
\subsection{a)}
\label{sec:org97398e9}
$$\deg(p(z)) = 4$$
Die Nullstellen von p sind \{4, \(i\), -5\}, wobei \(z_2 = i\) eine zwei-fache Nullstelle ist.

\subsection{b)}
\label{sec:orga269c14}
$$p(z) = \frac{z^5 + 4z^4 - 3z^3 + 2z^2 - z + 1}{z^2 - 3}$$

Die folgenden "Gleichungen" sind nicht als solche zu betrachten.
Sie zeigen lediglich die einzelnen Schritte der Polynomdivision.

\begin{equation}
\begin{aligned}
(z^5 + 4z^4 - 3z^3 + 2z^2 - z + 1) / (z^2 - 3) &= 0 \\
(4z^4 - 6z^3 + 2z^2 - z + 1) / (z^2 - 3) &= z^3 \\
(-6z^3 - 10z^2 - z + 1) / (z^2 - 3) &= z^3 + 4z^2 \\
(-10z^2 + 17z + 1) / (z^2 - 3) &= z^3 + 4z^2 - 6z \\
(17z + 31) / (z^2 - 3) &= z^3 + 4z^2 - 6z - 10
\end{aligned}
\end{equation}

Das Polynom nach der Division:

$$p(z) = z^3 + 4z^2 - 6z - 10 + \frac{17z + 31}{z^2 - 3}$$

\section{Aufgabe 4}
\label{sec:org4913f47}
\subsection{a)}
\label{sec:orgb9597c0}

\begin{equation}
\begin{aligned}
p(2i) &= (2i)^4 + (2i)^3 + 2(2i)^2 + 4(2i) - 8 \\
&= 16 - 8i - 8 + 8i - 8 \\
&= 0
\end{aligned}
\end{equation}

\subsection{b)}
\label{sec:orgbe3c87e}
$$\frac{p(z)}{q(z)} = \frac{z^4 + z^3 + 2z^2 + 4z - 8}{z^2 + 4}$$

Wie in 3b) sind die "Gleichungen" nur repr�sentativ f�r die einzelnen Schritte der Polynomdivision.

\begin{equation}
\begin{aligned}
(z^4 + z^3 + 2z^2 + 4z - 8) / (z^2 + 4) &= 0 \\
(z^3 - 2z^2 + 4z - 8) / (z^2 + 4) &= z^2 \\
(-2z^2 - 8) / (z^2 + 4) &= z^2 + z \\
(0) / (z^2 + 4) &= z^2 + z - 2
\end{aligned}
\end{equation}

Das Polynom ist dann nach der Division:

$$\frac{p(z)}{q(z)} = z^2 + z - 2$$

\subsection{c)}
\label{sec:org84f4b0c}
Da q | p (siehe 4b) sind die Nullstellen von q auch gleichzeitig die Nullstellen von p.
Die Nullstellen z\textsubscript{1, 2} von q sind:

$$q(z) = z^2 + 4 = (z + 2i)(z - 2i)$$
$$\Rightarrow z_{1, 2} = \pm 2i$$

Die anderen beiden Nullstellen von p erh�lt man vom Polynom \(z^2 + z - 2\) aus 4b:

\begin{equation}
\begin{aligned}
0 &= z^2 + z - 2 \\
\Rightarrow z_{3, 4} &= \frac{-1}{2} \pm \sqrt{\frac{1}{4} + \frac{8}{4}} \\
\Leftrightarrow z_{3, 4} &= \frac{-1}{2} \pm \frac{3}{2}
\end{aligned}
\end{equation}

Damit kann man p jewals komplex und reell zerlegen zu

\begin{equation}
\begin{aligned}
p(z) &= z^4 + z^3 + 2z^2 + 4z - 8 \\
 &= (z - 2i)(z + 2i)(z + 2)(z - 1) \\
 &= (z^2 + 4)(z + 2)(z - 1)
\end{aligned}
\end{equation}

mit den Nullstellen z\textsubscript{0} \(\in\) \{-2, -2i, 2i, 1\}.

\section{Aufgabe 5}
\label{sec:org2b13f58}
\subsection{a)}
\label{sec:org8e07420}
Bestimmen der Nullstellen des Nennerpolynoms \(q(z) = z^2 + 2z + 2\):

\begin{equation}
\begin{aligned}
0 &= z^2 + 2z + 2 \\
\Rightarrow z_{1, 2} &= -1 \pm \sqrt{1 - 2} \\
\Leftrightarrow z_{1, 2} &= -1 \pm i
\end{aligned}
\end{equation}

Die Linearfaktoren von q sind dann \((z + 1 - i)\) und \((z + 1 + i)\).
Daraus ergibt sich dann der Ansatz:

\begin{equation}
\begin{aligned}
f(z) &= \frac{z + 7}{(z + 1 - i)(z + 1 + i)}
 &= \frac{A}{(z + 1 - i)} + \frac{B}{(z + 1 + i)}
\end{aligned}
\end{equation}

Durch die Zuhaltemethode erh�lt man dann:

\begin{equation}
\begin{aligned}
A &= \frac{(-1 + i) + 7}{(-1 + i) + 1 + i} \\
 &= \frac{6 + i}{2i} \\
\Leftrightarrow A &= \frac{(6 + i)(-2i)}{2i(-2i)} = \frac{2 - 12i}{4} = \frac{1 - 6i}{2} \\
\end{aligned}
\end{equation}

\begin{equation}
\begin{aligned}
B &= \frac{(-1 - i) + 7}{(-1 - i) + 1 - i} \\
 &= \frac{6 - i}{-2i} \\
\Leftrightarrow B &= \frac{(6 - i)2i}{(-2i)2i} = \frac{2 - 12i}{4} = \frac{1 - 6i}{2}
\end{aligned}
\end{equation}

\subsection{b)}
\label{sec:org3d6c42a}
Ansatz f�r die reelle Zerlegung:
\begin{equation}
\begin{aligned}
f(x) &= \frac{2 + i}{x - (1 + 2i)} + \frac{2 - i}{x - (1 - 2i)} \\
 &= \frac{Cx + D}{(x - (1 + 2i))(x - (1 - 2i))} = \frac{Cx + D}{x^2 - 2x + 5} = \frac{Cx + D}{(x - 1)^2 + 4} \\
 &= \frac{(2 + i)(x - (1 - 2i)) + (2 - i)(x - (1 + 2i))}{(x - (1 + 2i))(x - (1 - 2i))} \\
 &= \frac{(2x - (2 - 4i) + xi - (i - 2i^2)) + (2x - (2 + 4i) - xi + (i + 2i^2))}{(x - 1)^2 + 4} \\
 &= \frac{2x - 2 + 2 + 2x - 2 - 2}{(x - 1)^2 + 4} \\
 &= \frac{2x - 2 + 2x - 2}{(x - 1)^2 + 4} \\
 &= \frac{4x - 4}{(x - 1)^2 + 4}
\end{aligned}
\end{equation}

Daraus folgt \(D = -4\) und \(C = 4\).

\section{Aufgabe 6}
\label{sec:orgc64d43c}
\subsection{a)}
\label{sec:orgc8ecb5e}

\subsection{b)}
\label{sec:org9ddddbc}
\end{document}